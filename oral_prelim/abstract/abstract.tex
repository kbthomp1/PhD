%        File: abstract.tex
%     Created: Tue Apr 12 03:00 PM 2016 E
% Last Change: Tue Apr 12 03:00 PM 2016 E
%
\documentclass[a4paper]{article}

\usepackage{newlfont}
\usepackage{titlesec}

  \title{Aerothermodynamic Design Sensitivities for a Reacting Gas Flow Solver
  on an Unstructured Mesh Using a Discrete Adjoint Formulation}

  \date{\vspace{-5ex}}
%\date{}  % Toggle commenting to test

 \author{
  Kyle B. Thompson%
    \thanks{NASA Pathways Intern, Aerothermodynamics Branch} \\
  {\normalsize\itshape NASA Langley Research Center, Hampton, VA} \\
 }

\begin{document}

\maketitle

\begin{abstract}

An approach is described to efficiently compute design sensitivities in an
inviscid, reacting gas flow solver, using a discrete adjoint formulation.  In
both the primal and adjoint flow solver the species continuity equations are
decoupled from the mixture continuity, momentum, and total energy equations.
This decoupling simplifies the implicit system, so that both solvers can be made
significantly more efficient, with very little penalty on overall scheme
robustness.  The computational cost of the point implicit relaxation in the
primal flow solver is shown to scale linearly with the number of species for the
decoupled system, whereas the fully coupled approach scales quadratically. To
demonstrate this capability, a hypersonic vehicle with a rcs jet configuration
is proposed as a design problem in which drag and surface temperature are sought
to be be optimized by a variety of design parameters related to the rcs jet
conditions and geometry.

\end{abstract}

\end{document}


