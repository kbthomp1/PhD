\documentclass{article}   	% use "amsart" instead of "article" for AMSLaTeX format

%Predefined definitions
\usepackage{common}

\usepackage{geometry}
%\geometry{letterpaper}                   		% ... or a4paper or a5paper or ... 
\usepackage{graphicx}
\usepackage{mathtools}
\usepackage{amsmath}
\usepackage{amssymb}
\usepackage{newlfont}
\usepackage{caption}
\usepackage{subcaption}
\usepackage{environ}

\NewEnviron{myequation}{%
    \begin{equation}
    \scalebox{1.1}{$\BODY$}
    \end{equation}
    }

\title{Quadratic Interpolation Between Thermodynamic Curve Fits}
\author{Kyle B. Thompson}

\begin{document}
\maketitle
%%% OUTPUT:

We seek to blend the two thermodynamic curve fits in such a way that we maintain $c_0$ continuity in both specific heat ($C_p$) and enthalpy ($h$).  To accomplish this, a quadratic function must be used, of the form
\begin{equation}
  a T^2 + b T + c = C_p
  \label{generic_form}
\end{equation}
The coefficients $a$, $b$, and $c$ are determined by solving the system that results from the boundary value problem
\begin{equation}
  \begin{cases}
    a {T_1}^{2} + b T_1 +c = C_{p_1} \\
    a {T_2}^{2} + b T_2 +c = C_{p_2} \\
    a \frac{\left( {T_2}^{3} - {T_1}^{3}\right) }{3} + b\frac{ \left( {T_2}^{2} - {T_1}^{2}\right) }{2} + c \left( T_2 - T_1\right) = h_2-h_1
  \end{cases}
\end{equation}
Where the $x_1$ and $x_2$ subscripts describe the left and right states, respectively.  Solving the linear system, the coefficients are
%%%%%%%%%%%%%%%
\begin{myequation}
  \begin{cases}
    a=\frac{3\left( C_{p_2}+ C_{p_1}\right) }{(T_2-T_1)^{2}} - \frac{6 \left(h_2 - h_1\right)}{(T_2-T_1)^{3}}\\ \\
    b=-\frac{2\left[(C_{p_2} + 2C_{p_1})T_2 + (2C_{p_2} + C_{p_1})T_1\right]}{(T_2-T_1)^{2}} + \frac{6(T_2+T_1)(h_2 - h_1)}{(T_2 - T_1)^3}\\ \\
    c=\frac{C_{p_1} T_2 (T_2 + 2T_1) + C_{p_2} T_1 (T_1 + 2 T_2)}{(T_2-T_1)^2} - \frac{6 T_1 T_2 (h_2 - h_1)}{(T_2 - T_1)^3}
  \end{cases}
\end{myequation}
This can be simplified to
\begin{gather}
  \begin{cases}
    a=3B - A \\ \\
    b=\frac{-2(C_{p_1} T_2 + C_{p_2}T_2)}{(T_2 - T_1)^2} +(T_2+T_1) (A - 2B) \\ \\
    c=\frac{C_{p_1} {T_2}^2 + C_{p_2} {T_1}^2}{(T_2-T_1)^2} + T_1 T_2 (2B - A)
  \end{cases} \\
  A = \frac{6(h_2 - h_1)}{(T_2 - T_1)^3} \\
  B = \frac{C_{p_2} + C_{p_1}}{(T_2 - T_1)^2}
\end{gather}

%\noindent
%%%%%%%%%%%%%%%%
%%%% INPUT:
%\begin{minipage}[t]{8ex}{\color{red}\bf
%\begin{verbatim}
%(%i12) 
%\end{verbatim}}
%\end{minipage}
%\begin{minipage}[t]{\textwidth}{\color{blue}
%\begin{verbatim}
%soln : solve(eqns,[a,b,c]);
%\end{verbatim}}
%\end{minipage}
%%%% OUTPUT:
%\definecolor{labelcolor}{RGB}{100,0,0}
%\begin{math}\displaystyle
%\parbox{8ex}{\color{labelcolor}(\%o12) }
%[[a=\frac{-3\,C_{p_2}\,T_2-3\,C_{p_1}\,T_2+\left( 3\,C_{p_2}+3\,C_{p_1}\right) \,T_1+6\,h_2-6\,h_1}{-{T_2}^{3}+3\,T_1\,{T_2}^{2}-3\,{T_1}^{2}\,T_2+{T_1}^{3}},b=-\frac{-2\,C_{p_2}\,{T_2}^{2}-4\,C_{p_1}\,{T_2}^{2}+T_1\,\left( -2\,C_{p_2}\,T_2+2\,C_{p_1}\,T_2+6\,h_2-6\,h_1\right) +\left( 6\,h_2-6\,h_1\right) \,T_2+\left( 4\,C_{p_2}+2\,C_{p_1}\right) \,{T_1}^{2}}{-{T_2}^{3}+3\,T_1\,{T_2}^{2}-3\,{T_1}^{2}\,T_2+{T_1}^{3}},c=\frac{-C_{p_1}\,{T_2}^{3}+T_1\,\left( -2\,C_{p_2}\,{T_2}^{2}-C_{p_1}\,{T_2}^{2}+\left( 6\,h_2-6\,h_1\right) \,T_2\right) +{T_1}^{2}\,\left( C_{p_2}\,T_2+2\,C_{p_1}\,T_2\right) +C_{p_2}\,{T_1}^{3}}{-{T_2}^{3}+3\,T_1\,{T_2}^{2}-3\,{T_1}^{2}\,T_2+{T_1}^{3}}]]
%\end{math}
%%%%%%%%%%%%%%%%

\end{document}  
