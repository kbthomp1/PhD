\documentclass[11pt,          % font size: 11pt or 12pt
               phd,           % degree:    ms or phd
               onehalfspacing % spacing: onehalfspacing or doublespacing
               ]{ncsuthesis}

%%----------------------------------------------------------------------------%%
%%------------------------------ Import Packages -----------------------------%%
%%----------------------------------------------------------------------------%%

\usepackage{booktabs}  % professionally typeset tables
\usepackage{amsmath}
\usepackage{amssymb}
\usepackage{textcomp}  % better copyright sign, among other things
\usepackage{xcolor}
\usepackage{lipsum}    % filler text
\usepackage{mathtools}
\usepackage{newlfont}
\usepackage{caption}
\usepackage{subcaption}
\usepackage{titlesec}
\usepackage[export]{adjustbox}
\usepackage[position=b]{subcaption}
\usepackage{url}
\urlstyle{same}


%%----------------------------------------------------------------------------%%
%%---------------------------- Formatting Options ----------------------------%%
%%----------------------------------------------------------------------------%%
%%

%% -------------------------------------------------------------------------- %%
%% Disposition format -- any titles, headings, section titles
%%  These formatting commands affect all headings, titles, headings,
%%  so sizing commands should not be used here.
%%  Formatting options to consider are
%%     +  \sffamily - sans serif fonts.  Dispositions are often typeset in
%%                    sans serif, so this is a good option. 
%%     +  \rmfamily - serif fonts
%%     +  \bfseries - bold face
%\dispositionformat{\sffamily\bfseries}   % bold and sans serif
\dispositionformat{\bfseries}            % bold and serif

%% -------------------------------------------------------------------------- %%
%% Formatting for centered headings - Abstract, Dedication, etc. headings
%%  This is where one might put a sizing command.
%%  \MakeUppercase can be used to typeset all headings in uppercase.
\headingformat{\large\MakeUppercase}   % All letters uppercase
%\headingformat{\large}                % Not all uppercase
%\headingformat{\Large\scshape}        % Small Caps, used with serif fonts.

%% Typographers recommend using a normal inter-word space after
%% sentences. TeX's default is to add an wider space, but \frenchspacing
%% gives a normal spacing. Comment out the following line if you prefer
%% wider spaces between sentences.
\frenchspacing


%% -------------------------------------------------------------------------- %%
%%  Optional packages
%%    A number of compatible packages to improve the look and feel of
%%    your document are available in the file optional.tex 
%%    (For example, hyperlinks, fancy chapter headings, and fonts)
%% To use these options, uncomment the next line and see optional.tex
%%%  Optional Packages to consider.   These packages are compatible with
%%    ncsuthesis.  

%% -------------------------------------------------------------------------- %%
%% Fancy chapter headings
%%  available options: Sonny, Lenny, Glenn, Conny, Rejne, Bjarne
%\usepackage[Sonny]{fncychap}

%%----------------------------------------------------------------------------%%
%% Hyperref package creates PDF metadata and hyperlinks in Table of Contents
%%  and citations.  Based on feedback from the NCSU thesis editor, 
%%  the links are not visually distinct from normal text (i.e. no change
%%  in color or extra boxes).
\usepackage[
  pdfauthor={Kyle B. Thompson},
  pdftitle={The Title},
  pdfcreator={pdftex},
  pdfsubject={NC State ETD Thesis},
  pdfkeywords={aerospace, adjoint, cfd},
  colorlinks=true,
  linkcolor=black,
  citecolor=black,
  filecolor=black,
  urlcolor=black,
]{hyperref}


%% -------------------------------------------------------------------------- %%
%% Microtype - If you use pdfTeX to compile your thesis, you can use
%%              the microtype package to access advanced typographic
%%              features.  By default, using the microtype package enables
%%              character protrusion (placing glyphs a hair past the right 
%%              margin to make a visually straighter edge)
%%              and font expansion (adjusting font width slightly to get 
%%              more favorable justification).
%%              Using microtype should decrease the number of lines
%%              ending in hyphens.
%\usepackage{microtype}


%%----------------------------------------------------------------------------%%
%% Fonts 

%% ETD guidelines don't specify the font.  You can enable the fonts
%%  by uncommenting the appropriate lines.  Using the default Computer 
%%  Modern fonts is *not* required.  A few common choices are below.
%%  See http://www.tug.dk/FontCatalogue/ for more options.

%% Serif Fonts -------------------------------------------------
%%  The four serif fonts listed here (Utopia, Palatino, Kerkis,
%%  and Times) all have math support.


%% Utopia
%\usepackage[T1]{fontenc}
%\usepackage[adobe-utopia]{mathdesign}

%% Palatino
%\usepackage[T1]{fontenc}
%\usepackage[sc]{mathpazo}
%\linespread{1.05}

%% Kerkis
%\usepackage[T1]{fontenc}
%\usepackage{kmath,kerkis}

%% Times
%\usepackage[T1]{fontenc}
%\usepackage{mathptmx}


%% Sans serif fonts -------------------------

%\usepackage[scaled]{helvet}  % Helvetica
%\usepackage[scaled]{berasans} % Bera Sans



%%----------------------------------------------------------------------------%%
%%---------------------------- Content Options -------------------------------%%
%%----------------------------------------------------------------------------%%
%% Size of committee: 3, 4, 5, or 6 -- this number includes the chair
\committeesize{4}

%% Members of committee
%%  Each of the following member commands takes an optional argument
%%   to specify their role on the committee.
%%  For co-chairs, use the commands:
%%      \cochairI{Doug Dodd}
%%      \cochairII{Chris Cox}
%%
\cochairI{Hassan Hassan}
\cochairII{Peter Gnoffo}
\memberI{Jack Edwards}
\memberII{John Griggs}


%% Student writing thesis, \student{First Middle}{Last}
\student{Kyle B.}{Thompson} % a middle initial

%% Degree program
\program{Aerospace Engineering}

%% Thesis Title
%%  Keep in mind, according to ETD guidelines:
%%    +  Capitalize first letter of important words.
%%    +  Use inverted pyramid shape if title spans more than one line.
%%
%%  Note: To break the title onto multiple lines, use \break instead of \\.
\thesistitle{A North Carolina State University Sample \LaTeX{} Thesis \break 
with a Title So Long it Needs a Line Break}

%% Degree year.  Necessary if your degree year doesn't equal the current year.
%\degreeyear{1995}


%%----------------------------------------------------------------------------%%
%%---------------------------- Personal Macros -------------------------------%%
%%----------------------------------------------------------------------------%%

%% A central location to add your favorite macros.

%% A few examples to get you started.
\newcommand{\uv}[1]{\ensuremath{\mathbf{\hat{#1}}}}
\newcommand{\bo}{\ensuremath{\boldsymbol{\Omega}}}
\newcommand{\eref}[1]{Eq.~\ref{#1}}
\newcommand{\fref}[1]{Figure~\ref{#1}}
\newcommand{\tref}[1]{Table~\ref{#1}}

%%---------------------------------------------------------------------------%%
\begin{document}

%%---------------------------------------------------------------------------%%
\frontmatter

%% ------------------------------ Abstract ---------------------------------- %%
\begin{abstract}

\lipsum[1-6]


\end{abstract}


%% ---------------------------- Copyright page ------------------------------ %%
%% Comment the next line if you don't want the copyright page included.
\makecopyrightpage

%% -------------------------------- Title page ------------------------------ %%
\maketitlepage

%% -------------------------------- Dedication ------------------------------ %%
\begin{dedication}
 \centering To my parents.
\end{dedication}

%% -------------------------------- Biography ------------------------------- %%
\begin{biography}
The author was born in a small town \ldots
\end{biography}

%% ----------------------------- Acknowledgements --------------------------- %%
\begin{acknowledgements}
I would like to thank my advisor for his help.
\end{acknowledgements}


\thesistableofcontents

\thesislistoftables

\thesislistoffigures


%%---------------------------------------------------------------------------%%
\mainmatter

\chapter{Introduction}
\label{chapter-one}

In the last decades computational fluid dynamics (CFD) codes have matured to the
point that it is possible to obtain high-fidelity design sensitivities that can
be coupled with optimization packages to enable design optimization for a
variety of design inputs\cite{baysal1992aerodynamic, balagangadhar2001design}.
In recent years, the benefits of using an adjoint-based formulation to compute
sensitivities have been realized and implemented in many compressible CFD
codes\cite{mavriplis-2006, nemec-aftosmis-adjoint, nielsen2002recent}, because
of the ability compute all sensitivities at the cost of a single extra adjoint
solution, instead of an additional flow solution for each design variable.
Reacting gas CFD codes have lagged in adopting this adjoint-based
approach, but interest has increased in recent years\cite{Copeland, Barcelona,
lockwood2010parameter}.  The relatively small number of reacting gas CFD codes
that employ adjoint-based sensitivity analysis is likely due to the significant
jump in complexity of the linearizations required, especially in regard to the
chemical source term and the dissipation term in the Roe flux difference
splitting (FDS) scheme\cite{roe}.

In the reacting gas solver community, there is a split in the methodology used
to compute adjoint-based sensitivity derivatives, specifically whether to
implement a continuous or discrete version of adjoint equations.  In a
continuous formulation of the adjoint equations, the flow solver equations are
linearized and then discretized, whereas in discrete formulation, the flow
solver equations are discretized and then linearized.  Copeland et
al.\cite{Copeland} have adopted a continuous formulation of the adjoint equations
for flows in chemical and thermal non-equilibrium, and has implemented this
formulation in the from Stanford University CFD code
SU2\cite{palacios2013stanford}.  The reasoning cited for choosing a continuous
formulation of the adjoint equations, was that, among other things, the adjoint
system was not suited to being solved by the same methods suited for solving the
flow solver equations and the resulting adjoint solution could lead to
non-physical oscillations that fail to capture the continuous adjoint solution.
Lockwood et al.\cite{lockwood2010uncertainty, lockwood2010parameter}, on the
other hand, have published favorable results for a discrete adjoint formulation
for both parameter sensitivity and uncertainty quantification.

For the work presented here, a discrete adjoint formulation is chosen over a
continuous one.  The reason for this decision is that merits a of continuous
adjoint formulation do not outweigh the testing and practical benefits of a
discrete adjoint formulation.  Because the flow solver equations are discretized
before linearizing them in a discrete formulation, the adjoint solution can be
verified using finite-difference derivatives.  This is not possible with a
continuous formulation, as the solution to the continuous problem will not
exactly match the discretization; therefore, from a practical standpoint, the
testability of a continuous adjoint solution is much more ambiguous than a
discrete adjoint solution.  The claim that non-equilibrium flow solvers are not
guaranteed to solve the adjoint system is disputed here, as Nielsen et
al.\cite{nielsen2004implicit} proved that a discrete adjoint scheme can be made
dual exact, where the discrete adjoint approach matches the direct
differentiation approach exactly for each time step.

An additional obstacle that may prevent adjoint-based sensitivity analysis in
reacting gas solvers is the extreme problem size associated with high energy
physics.  The additional equations required in reacting gas simulations lead to
large Jacobians that scale quadratically in size to the number of governing
equations.  This leads to a significant increase in the memory required to store
the flux linearizations and the computational cost of the point solver.  As
reacting gas CFD solvers are used to solve increasingly more complex problems,
this onerous quadratic scaling of computational cost and Jacobian size will
ultimately surpass the current limits of hardware and time constraints on
achieving a flow solution\cite{fischer}.

To mitigate this scaling issue, Candler et al.\cite{candler} proposed a scheme
to for a modified form of the Steger-Warming flux vector splitting
scheme\cite{MacCormack,Steger}. In that work, it was shown that quadratic
scaling between the cost of solving the implicit system and adding species mass
equations can be reduced from quadratic to linear scaling by decoupling the
species mass equations from the mixture mass, momentum, and energy equations and
solving the two systems sequentially.  This work extends the aforementioned work
from the modified form of the Steger-Warming flux vector splitting
method to the Roe flux difference splitting (FDS) scheme.  Candler et
al.\cite{candler2013analysis} later analyzed stability issues associated with
the decoupled formulation in \cite{candler}.  This work also presents a chemical
source term scaling procedure that mitigates these stability concerns.

As its primary focus, this work demonstrates that a variation of this decoupled
point-implicit scheme can be applied to both the flow solver and adjoint solver
in a reacting gas CFD code, and significantly improve the efficiency in both
computational cost and memory required.  Additionally, the implementation of
exact linearizations is shown to significantly improve performance and
robustness in the flow solver over common approximations used when linearizing
the Roe FDS scheme.

An axi-symmetric hypersonic re-entry vehicle with an annular jet is chosen as
the main demonstration problem in this work.  This design was previously
examined by Gnoffo et al.\cite{gnoffo2016tapping} for a perfect gas as a means
of increasing the time-averaged vehicle drag via unsteady interactions between
the annular jet plume and the bow shock.  A steady solution was found for that
geometry and it was postulated that the annular jet plume could be used as a
active cooling mechanism to augment or replace the passive thermal protection
system (TPS) on the vehicle.  This work extends the previous work to inviscid
flow in chemical non-equilibrium, and applies sensitivities computed by the
adjoint solver to minimize the vehicle outer surface temperature and mass flow
rate through the annular jet plenum by changing the conditions at the plenum.



%%---------------------------------------------------------------------------%%
%%  Bibliography 

%%  You can use the bibitem list.
%\bibliographystyle{unsrt}
%\begin{thebibliography}{99}
%\bibitem{cb02}
%Casella, G. and Berger, R.L. (2002)
%\newblock {\it Statistical Inference, Second Edition.}
%Duxbury Press, Belmont, CA.
%
%\bibitem{t06}
%Tsiatis, A.A. (2006)
%\newblock {\it Semiparametric Theory and Missing Data.}
%Springer, New York.
%
%\end{thebibliography}

%% or use BibTeX
\bibliography{KyleThompson-thesis}{}
\bibliographystyle{plain}

%%---------------------------------------------------------------------------%%
% Appendices
\appendix

\chapter{Derivations}
\label{derivations}

\section{Decoupled Flux Derivation}
\label{decoupled-flux-derivation}

For the Roe flux difference splitting scheme, the species mass fluxes are given by
%------------------------------------------------------------------------------%
\begin{equation}
	F_{\rho_s} = \frac{\rho_s^L\overline{U}^L+\rho_s^R\overline{U}^R}{2}
	-\frac{\tilde{c}_s(\lambda_1 dv_1 + \lambda_2 dv_2)+\lambda_3 dv_{3_s}}{2}
  \label{species-mass} \\
\end{equation}
\begin{align}	
		dv_1 &= \frac{p^R-p^L+\tilde{\rho} \tilde{a} (\overline{U}^R-\overline{U}^L)}{\tilde{a}^2} \\
		dv_2 &= \frac{p^R-p^L-\tilde{\rho} \tilde{a} (\overline{U}^R-\overline{U}^L)}{\tilde{a}^2} \\
		dv_{3_s} &= \frac{\tilde{a}^2 (\rho_s^R-\rho_s^L)- \tilde{c}_s (p^R-p^L)}{\tilde{a}^2}
\end{align}
\begin{align}
	\lambda_1 = \mid\mathbf{\overline{U}}+\tilde{a} \mid,\quad 
	\lambda_2 = \mid \mathbf{\overline{U}}-\tilde{a} \mid,\quad 
	\lambda_3 =  \mid \mathbf{\overline{U}} \mid
\end{align}
%------------------------------------------------------------------------------%
where the $\tilde{}$ notation signifies a Roe-averaged quantity, given by:
%------------------------------------------------------------------------------%
\begin{gather}
	\mathbf{\tilde{U}} =\roe \mathbf{\tilde{U}}^L+(1-\roe)\mathbf{\tilde{U}}^R \\
	\roe = \frac{\tilde{\rho}}{\tilde{\rho}+\rho^R} \\
	\tilde{\rho} = \sqrt{\rho^R\rho^L}
\end{gather}
%------------------------------------------------------------------------------%
The species mass fluxes must sum to the total mass flux; thus, the total mixture
mass flux is given as
%------------------------------------------------------------------------------%
\begin{equation}
\label{total-mass}
	F_\rho = \sum\limits_{s}{F_{\rho_s}} = \frac{\rho^L\overline{U}^L+\rho^R\overline{U}^R}{2}
	-\frac{\lambda_1 dv_1 + \lambda_2 dv_2 +\lambda_3 dv_3}{2}
\end{equation}
\begin{equation}
	dv_3 = \frac{\tilde{a}^2 (\rho^R-\rho^L)-(p^R-p^L)}{\tilde{a}^2}
\end{equation}
%------------------------------------------------------------------------------%
Multiplying \eref{total-mass} by the Roe-averaged mass fraction and
substituting it into \eref{species-mass} results in:
%------------------------------------------------------------------------------%
\begin{equation}
\label{unsimp-sp-flux}
	F_{\rho_s} =\tilde{c}_s F_\rho + \frac{(c_s^L-\tilde{c}_s)\rho^L(\overline{U}^L+\mid \tilde{U}\mid)}{2}
	+ \frac{(c_s^R-\tilde{c}_s)\rho^R(\overline{U}^R-\mid \tilde{U}\mid)}{2}
\end{equation}
%------------------------------------------------------------------------------%
It should be noted here that the Roe-averaged normal velocity, $\tilde{U}$,
requires an entropy correction in the presence of strong shocks\cite{harten}.
This correction has a dependence on the roe-averaged speed of sound, and
therefore has a dependence on the species mass fractions; however,
through numerical experiments it has been determined that omitting this
dependence does not adversely affect convergence.  The notation can be further
simplified by defining the normal velocities as follows:
%------------------------------------------------------------------------------%
\begin{equation}
  \lambda^+ = \frac{\overline{U}^L+\mid \tilde{U}\mid}{2}, \quad 
  \lambda^- = \frac{\overline{U}^R-\mid \tilde{U}\mid}{2} 
  \label{lambda-pm} 
\end{equation}
%------------------------------------------------------------------------------%
Finally, substituting \eref{lambda-pm} into \eref{unsimp-sp-flux} yields the
final result for calculating the species flux in the decoupled system:
%------------------------------------------------------------------------------%
\begin{equation}
  F_{\rho_s} =
  \tilde{c}_s F_\rho 
  + (c_s^L-\tilde{c}_s)\rho^L\lambda^+ 
  + (c_s^R-\tilde{c}_s)\rho^R\lambda^-
  \label{sp-flux} 
\end{equation}
%------------------------------------------------------------------------------%
Forming the convective contributions to the Jacobians is straightforward.
Because the $\mathbf{U}'$ level variables are constant, only the left, right,
and Roe-averaged state mass fractions vary.  Differentiating \eref{sp-flux}
with respect to the mass fraction, $c_s$, the left and right state contributions
are
%------------------------------------------------------------------------------%
\begin{gather}
  \pd{F_{\rho_s}}{c^L_s} = \roe F_\rho+(1-\roe)\rho^L\lambda^+ - \roe \rho^R\lambda^- \\
  \pd{F_{\rho_s}}{c^R_s} = (1-\roe) F_\rho+(\roe-1)\rho^L\lambda^+ + \roe\rho^R\lambda^- 
  \label{dc-sp-linearizations}
\end{gather}
%------------------------------------------------------------------------------%
Because there is no dependence between species in decoupled convective
formulation, the Jacobian block elements are purely diagonal for the convective
contributions, of the form
%------------------------------------------------------------------------------%
\begin{equation} 
  \begin{pmatrix} 
    \pd{F_{\rho_1}}{c_1} & & 0 \\ 
    & \ddots & \\ 
    0 & & \pd{F_{\rho_{ns}}}{c_{ns}}
  \end{pmatrix}
  \label{dc-diag-Jacobian}
\end{equation}
%------------------------------------------------------------------------------%

\section{Quadratic Interpolation Between Thermodynamic Curve Fits}
\label{sec:quad-cp-blending}

We seek to blend the two thermodynamic curve fits in such a way that we maintain
$c_0$ continuity in both specific heat ($C_p$) and enthalpy ($h$).  To
accomplish this, a quadratic function must be used, of the form
%------------------------------------------------------------------------------%
\begin{equation}
  a T^2 + b T + c = C_p
  \label{generic_form}
\end{equation}
%------------------------------------------------------------------------------%
The coefficients $a$, $b$, and $c$ are determined by solving the system that
results from the boundary value problem
%------------------------------------------------------------------------------%
\begin{equation}
  \begin{cases}
    a {T_1}^{2} + b T_1 +c = C_{p_1} \\
    a {T_2}^{2} + b T_2 +c = C_{p_2} \\
    a \frac{\left( {T_2}^{3} - {T_1}^{3}\right) }{3} + b\frac{ \left( {T_2}^{2} - {T_1}^{2}\right) }{2} + c \left( T_2 - T_1\right) = h_2-h_1
  \end{cases}
\end{equation}
%------------------------------------------------------------------------------%
Where the $x_1$ and $x_2$ subscripts describe the left and right states,
respectively.  Solving the linear system, the coefficients are
%------------------------------------------------------------------------------%
\begin{equation}
  \begin{cases}
    a=\frac{3\left( C_{p_2}+ C_{p_1}\right) }{(T_2-T_1)^{2}} - \frac{6 \left(h_2 - h_1\right)}{(T_2-T_1)^{3}}\\ \\
    b=-\frac{2\left[(C_{p_2} + 2C_{p_1})T_2 + (2C_{p_2} + C_{p_1})T_1\right]}{(T_2-T_1)^{2}} + \frac{6(T_2+T_1)(h_2 - h_1)}{(T_2 - T_1)^3}\\ \\
    c=\frac{C_{p_1} T_2 (T_2 + 2T_1) + C_{p_2} T_1 (T_1 + 2 T_2)}{(T_2-T_1)^2} - \frac{6 T_1 T_2 (h_2 - h_1)}{(T_2 - T_1)^3}
  \end{cases}
\end{equation}
%------------------------------------------------------------------------------%
This can be simplified to
%------------------------------------------------------------------------------%
\begin{gather}
  \begin{cases}
    a=3B - A \\ \\
    b=\frac{-2(C_{p_1} T_2 + C_{p_2}T_2)}{(T_2 - T_1)^2} +(T_2+T_1) (A - 2B) \\ \\
    c=\frac{C_{p_1} {T_2}^2 + C_{p_2} {T_1}^2}{(T_2-T_1)^2} + T_1 T_2 (2B - A)
  \end{cases} \\
  A = \frac{6(h_2 - h_1)}{(T_2 - T_1)^3} \\
  B = \frac{C_{p_2} + C_{p_1}}{(T_2 - T_1)^2}
\end{gather}
%------------------------------------------------------------------------------%
Note that this does not ensure that entropy will be continuous across curve
fits.

\section{Change of Variable Sets}
\label{change-of-var-section}

The decoupled scheme developed by Candler et. al\cite{candler} is based upon the
change of variables
%------------------------------------------------------------------------------%
\begin{equation}
  \mU = \begin{pmatrix}
    \rho_1 \\
    \vdots \\
    \rho_{ns} \\
    \rho \vu \\
    \rho E
  \end{pmatrix}
  \rightarrow
  \mv = \begin{pmatrix}
    c_1 \\
    \vdots \\
    c_{ns} \\
    \rho \\
    \rho \vu \\
    \rho E
  \end{pmatrix}
  \label{var-sets}
\end{equation}
%------------------------------------------------------------------------------%
To avoid confusion between variable sets, we re-write the variable vectors,
$\mU$ and $\mv$, in a more generic sense
%------------------------------------------------------------------------------%
\begin{equation}
  \mU = \begin{pmatrix}
    u_1 \\
    \vdots \\
    u_{ns + 2}
  \end{pmatrix}
  \rightarrow
  \mv = \begin{pmatrix}
    v_1 \\
    \vdots \\
    v_{ns + 3}
  \end{pmatrix}
  \label{generic-var-sets}
\end{equation}
%------------------------------------------------------------------------------%
For simplicity, consider a system with two species, $\rho_1$ and $\rho_2$.
Using the relationship $\rho_s = c_s \rho$, then the original variable vector,
$\mU$ can be rewritten in terms of the new variables, $\mv$ as
%------------------------------------------------------------------------------%
\begin{equation}
  \mU = \begin{pmatrix}
    u_1 \\
    u_2 \\
    u_3 \\
    u_4
  \end{pmatrix}
  =
  \begin{pmatrix}
    v_1 v_3 \\
    v_2 v_3 \\
    v_4 \\
    v_5
  \end{pmatrix}
  \label{u-to-v}
\end{equation}
%------------------------------------------------------------------------------%
This allows the derivation of the Jacobian 
%------------------------------------------------------------------------------%
\begin{equation}
  \pd{\mU}{\mv} = 
  \begin{pmatrix}
    \pd{u_1}{v_1} & \pd{u_1}{v_2} & \pd{u_1}{v_3} & \pd{u_1}{v_4} & \pd{u_1}{v_5} \\ \\
    \pd{u_2}{v_1} & \pd{u_2}{v_2} & \pd{u_2}{v_3} & \pd{u_2}{v_4} & \pd{u_2}{v_5} \\ \\
    \pd{u_3}{v_1} & \pd{u_3}{v_2} & \pd{u_3}{v_3} & \pd{u_3}{v_4} & \pd{u_3}{v_5} \\ \\
    \pd{u_4}{v_1} & \pd{u_4}{v_2} & \pd{u_4}{v_3} & \pd{u_4}{v_4} & \pd{u_4}{v_5}
  \end{pmatrix}
  =
  \begin{pmatrix}
    v_3 & 0   & v_1 & 0 & 0 \\ \\
    0   & v_3 & v_2 & 0 & 0 \\ \\
    0   & 0   & 0   & 1 & 0 \\ \\
    0   & 0   & 0   & 0 & 1
  \end{pmatrix}
  \label{dudv-jac}
\end{equation}
%------------------------------------------------------------------------------%
At this point, it is important to note that the Jacobian in \eref{dudv-jac} has
two psuedo-inverse matricies, that correspond to the right and left inverse.
The right inverse, $\pdr{\mv}{\mU}{R}$, can be constructed based on the previously
defined steps
%------------------------------------------------------------------------------%
\begin{equation}
  \pdr{\mv}{\mU}{R} = 
  \begin{pmatrix}
    \pd{v_1}{u_1} & \pd{v_1}{u_2} & \pd{v_1}{u_3} & \pd{v_1}{u_4} \\ \\
    \pd{v_2}{u_1} & \pd{v_2}{u_2} & \pd{v_2}{u_3} & \pd{v_2}{u_4} \\ \\
    \pd{v_3}{u_1} & \pd{v_3}{u_2} & \pd{v_3}{u_3} & \pd{v_3}{u_4} \\ \\
    \pd{v_4}{u_1} & \pd{v_4}{u_2} & \pd{v_4}{u_3} & \pd{v_4}{u_4} \\ \\
    \pd{v_5}{u_1} & \pd{v_5}{u_2} & \pd{v_5}{u_3} & \pd{v_5}{u_4} 
  \end{pmatrix}
  =
  \begin{pmatrix}
    \frac{1-v_1}{v_3} & \frac{-v_1}{v_3}  & 0 & 0 \\ \\
    \frac{-v_2}{v_3}  & \frac{1-v_2}{v_3} & 0 & 0 \\ \\
    1                 & 1                 & 0 & 0 \\ \\
    0                 & 0                 & 1 & 0 \\ \\
    0                 & 0                 & 0 & 1
  \end{pmatrix}
  \label{dvdu-jac-right}
\end{equation}
%------------------------------------------------------------------------------%
It is easily verified that the matrix product of
\erefs{dudv-jac}{dvdu-jac-right} produces identity
%------------------------------------------------------------------------------%
\begin{equation}
  \pd{\mU}{\mv} \pdr{\mv}{\mU}{R} = 
  \begin{pmatrix}
    v_3 & 0   & v_1 & 0 & 0 \\ \\
    0   & v_3 & v_2 & 0 & 0 \\ \\
    0   & 0   & 0   & 1 & 0 \\ \\
    0   & 0   & 0   & 0 & 1
  \end{pmatrix}
  \begin{pmatrix}
    \frac{1-v_1}{v_3} & \frac{-v_1}{v_3}  & 0 & 0 \\ \\
    \frac{-v_2}{v_3}  & \frac{1-v_2}{v_3} & 0 & 0 \\ \\
    1                 & 1                 & 0 & 0 \\ \\
    0                 & 0                 & 1 & 0 \\ \\
    0                 & 0                 & 0 & 1
  \end{pmatrix}
  =
  \begin{pmatrix}
    1 & 0 & 0 & 0 \\ \\
    0 & 1 & 0 & 0 \\ \\
    0 & 0 & 1 & 0 \\ \\
    0 & 0 & 0 & 1
  \end{pmatrix}
  \label{dvdu-jac-right-I}
\end{equation}
%------------------------------------------------------------------------------%
however, \eref{dvdu-jac-right-I} is not associative
%------------------------------------------------------------------------------%
\begin{equation}
  \begin{aligned}
    \pdr{\mv}{\mU}{R} \pd{\mU}{\mv} &= 
    \begin{pmatrix}
      \frac{1-v_1}{v_3} & \frac{-v_1}{v_3}  & 0 & 0 \\ \\
      \frac{-v_2}{v_3}  & \frac{1-v_2}{v_3} & 0 & 0 \\ \\
      1                 & 1                 & 0 & 0 \\ \\
      0                 & 0                 & 1 & 0 \\ \\
      0                 & 0                 & 0 & 1
    \end{pmatrix}
    \begin{pmatrix}
      v_3 & 0   & v_1 & 0 & 0 \\ \\
      0   & v_3 & v_2 & 0 & 0 \\ \\
      0   & 0   & 0   & 1 & 0 \\ \\
      0   & 0   & 0   & 0 & 1
    \end{pmatrix} \\
    &= 
    \begin{pmatrix}
      1-v_1 & -v_1  & \frac{-(v_1)^2 - v_1 v_2 + v_1}{v_3} & 0 & 0 \\ \\
      -v_2  & 1-v_2 & \frac{-(v_2)^2 - v_1 v_2 + v_2}{v_3} & 0 & 0 \\ \\
      0     & 0     & 0                                    & 1 & 0 \\ \\
      0     & 0     & 0                                    & 0 & 1
    \end{pmatrix}
  \end{aligned}
  \label{dvdu-jac-right-I-bad}
\end{equation}
%------------------------------------------------------------------------------%
to correctly compute identity, the property of matrix transpose multiplication
is used
%------------------------------------------------------------------------------%
\begin{equation}
  \begin{aligned}
    \left( \pd{\mU}{\mv} \pdr{\mv}{\mU}{R} \right)^T &= 
    {\pdr{\mv}{\mU}{R}}^T {\pd{\mU}{\mv}}^T \\ &=
    \begin{pmatrix}
      \frac{1-v_1}{v_3} & \frac{-v_2}{v_3}  & 1 & 0 & 0 \\ \\
      \frac{-v_1}{v_3}  & \frac{1-v_2}{v_3} & 1 & 0 & 0 \\ \\
      0                 & 1                 & 0 & 1 & 0 \\ \\
      0                 & 0                 & 0 & 0 & 1
    \end{pmatrix}
    \begin{pmatrix}
      v_3 & 0   & 0   & 0 \\ \\
      0   & v_3 & 0   & 0 \\ \\
      v_1 & v_2 & 0   & 0 \\ \\
      0   & 0   & 1   & 0 \\ \\
      0   & 0   & 0   & 1 
    \end{pmatrix} \\
    &= 
    \begin{pmatrix}
      1 & 0 & 0 & 0 \\ \\
      0 & 1 & 0 & 0 \\ \\
      0 & 0 & 1 & 0 \\ \\
      0 & 0 & 0 & 1
    \end{pmatrix}
  \end{aligned}
  \label{dvdu-jac-tranpose}
\end{equation}
%------------------------------------------------------------------------------%
This is critical in understanding the relationships needed to switch transform
variables sets.  For linearizations of the residual, $\mr$, the correct
transformation from the variable set $\mU$ to the variable set $\mv$ is
%------------------------------------------------------------------------------%
\begin{equation}
  \pd{\mr}{\mv} = \pd{\mr}{\mU} \pd{\mU}{\mv}
  \label{r-u-to-v}
\end{equation}
%------------------------------------------------------------------------------%
which is intuitively understood; however, the transformation from the variable
set $\mv$ to the variable set $\mU$ must follow \eref{dvdu-jac-tranpose}
%------------------------------------------------------------------------------%
\begin{equation}
  \pd{\mr}{\mU} = \left( \pd{\mr}{\mv}^{T} \pd{\mv}{\mU}^{T} \right)^{T}
  \label{r-v-to-u}
\end{equation}
%------------------------------------------------------------------------------%
The transposition in \eref{r-u-to-v} is critical, as the linearizations will be
incorrect if the multiplication is done without it.  Fortunately,
\eref{r-u-to-v} is rarely seen in practice, as most linearizations are done for
the fully-coupled system that requires \eref{r-v-to-u} to transform the
linearizations
%------------------------------------------------------------------------------%
\begin{equation}
  \begin{aligned}
    \pd{\mr}{\mv} = \pd{\mr}{\mU} \pd{\mU}{\mv} =
    \begin{pmatrix}
      \rdiff{\rho_1}{\rho_1}    & \dots  & \rdiff{\rho_1}{\rho_{ns}}    & \rdiff{\rho_1}{\rho \vu}    & \rdiff{\rho_1}{\rho E}    \\ \\
      \vdots                    & \ddots & \vdots                       & \vdots                      & \vdots                    \\ \\
      \rdiff{\rho_{ns}}{\rho_1} & \dots  & \rdiff{\rho_{ns}}{\rho_{ns}} & \rdiff{\rho_{ns}}{\rho \vu} & \rdiff{\rho_{ns}}{\rho E} \\ \\
      \rdiff{\rho \vu}{\rho_1}  & \dots  & \rdiff{\rho \vu}{\rho_{ns}}  & \rdiff{\rho \vu}{\rho \vu}  & \rdiff{\rho \vu}{\rho E}  \\ \\
      \rdiff{\rho E}{\rho_1}    & \dots  & \rdiff{\rho E}{\rho_{ns}}    & \rdiff{\rho E}{\rho \vu}    & \rdiff{\rho E}{\rho E}
    \end{pmatrix}
    \begin{pmatrix}
      \rho   & \dots  & 0      & c_1     & 0      & 0      \\ \\
      \vdots & \ddots & \vdots & \vdots  & \vdots & \vdots \\ \\
      0      & \dots  &\rho    & c_{ns}  & 0      & 0      \\ \\
      0      & \dots  &0       & 0       & 1      & 0      \\ \\
      0      & \dots  &0       & 0       & 0      & 1
    \end{pmatrix}
  \end{aligned}
  \label{drdu-to-drdv}
\end{equation}
%------------------------------------------------------------------------------%
likewise, in the adjoint the transformation is applied to the tranpose of the 
Jacobian
%------------------------------------------------------------------------------%
\begin{equation}
  \begin{aligned}
    \pd{\mr}{\mU}^{T} = \pd{\mU}{\mv}^{T} \pd{\mr}{\mU}^{T} =
    \begin{pmatrix}
      \rho   & \dots  & 0      &  0      & 0      \\ \\
      \vdots & \ddots & \vdots &  \vdots & \vdots \\ \\
      0      & \dots  &\rho    &  0      & 0      \\ \\
      c_1    & \dots  & c_{ns} &  0      & 0      \\ \\
      0      & \dots  & 0      &  1      & 0      \\ \\
      0      & \dots  & 0      &  0      & 1
    \end{pmatrix}
    \begin{pmatrix}
      \rdiff{\rho_1}{\rho_1}    & \dots  & \rdiff{\rho_{ns}}{\rho_1}    & \rdiff{\rho \vu}{\rho_1}    & \rdiff{\rho E}{\rho_1} \\ \\
      \vdots                    & \ddots & \vdots                       & \vdots                      & \vdots                   \\ \\
      \rdiff{\rho_1}{\rho_{ns}} & \dots  & \rdiff{\rho_{ns}}{\rho_{ns}} & \rdiff{\rho \vu}{\rho_{ns}} & \rdiff{\rho E}{\rho_{ns}} \\ \\
      \rdiff{\rho_1}{\rho \vu}  & \dots  & \rdiff{\rho_{ns}}{\rho \vu}  & \rdiff{\rho \vu}{\rho \vu}  & \rdiff{\rho E}{\rho \vu} \\ \\
      \rdiff{\rho_1}{\rho E}    & \dots  & \rdiff{\rho_{ns}}{\rho E}    & \rdiff{\rho \vu}{\rho E}    & \rdiff{\rho E}{\rho E}
    \end{pmatrix}
  \end{aligned}
  \label{drdu-to-drdv-t}
\end{equation}
%------------------------------------------------------------------------------%
Since the tranformation is left-multiplied, the matrix vector products of the 
exact Jacobian with costate variables, $\adjlam{}$, in the adjoint linear system
can be done first, and the transformation can then be applied to the system
%------------------------------------------------------------------------------%
\begin{equation}
  \begin{aligned}
    \pd{\mr}{\mU}^{T} \adjlam{} =
    \begin{pmatrix}
      \rho   & \dots  & 0      &  0      & 0      \\ \\
      \vdots & \ddots & \vdots &  \vdots & \vdots \\ \\
      0      & \dots  &\rho    &  0      & 0      \\ \\
      c_1    & \dots  & c_{ns} &  0      & 0      \\ \\
      0      & \dots  & 0      &  1      & 0      \\ \\
      0      & \dots  & 0      &  0      & 1
    \end{pmatrix}
    \begin{pmatrix}
      \rlprod{\rho_1}{\rho_1}    & \dots  &+& \rlprod{\rho_{ns}}{\rho_1}    &+& \rlprod{\rho \vu}{\rho_1}    &+& \rlprod{\rho E}{\rho_1} \\ \\
      \vdots                     & \ddots & & \vdots                        & & \vdots                       & & \vdots                   \\ \\
      \rlprod{\rho_1}{\rho_{ns}} & \dots  &+& \rlprod{\rho_{ns}}{\rho_{ns}} &+& \rlprod{\rho \vu}{\rho_{ns}} &+& \rlprod{\rho E}{\rho_{ns}} \\ \\
      \rlprod{\rho_1}{\rho \vu}  & \dots  &+& \rlprod{\rho_{ns}}{\rho \vu}  &+& \rlprod{\rho \vu}{\rho \vu}  &+& \rlprod{\rho E}{\rho \vu} \\ \\
      \rlprod{\rho_1}{\rho E}    & \dots  &+& \rlprod{\rho_{ns}}{\rho E}    &+& \rlprod{\rho \vu}{\rho E}    &+& \rlprod{\rho E}{\rho E}
    \end{pmatrix}
  \end{aligned}
  \label{adj-drdv}
\end{equation}
%------------------------------------------------------------------------------%
This indicates the important point that the transformation of the adjoint
residual is not dependent on the number of equations solved, but only the number
of dependent variables the equations are linearized with respect to, namely
$\mU$.

\section{Change of Equation Sets}
\label{sec:change-of-equations}

In addition to a change of variables, the decoupled scheme also changes the
equations being solved, effectively solving is one more equation than the
fully-coupled scheme.  The residual vector for the decoupled scheme, $\rv{}$,
and the residual vector for the fully coupled scheme, $\ru{}$, can be written as
%------------------------------------------------------------------------------%
\begin{equation}
  \ru{} =
  \begin{pmatrix}
    \res{\rho_1} \\ \\
    \vdots \\ \\
    \res{\rho_{N_s}} \\ \\
    \res{\rho \vu} \\ \\
    \res{\rho E}
  \end{pmatrix}
  \rightarrow
  \rv{} =
  \begin{pmatrix}
    \res{\rho_1} - c_1 \resrho \\ \\
    \vdots \\ \\
    \res{\rho_{N_s}} - c_{N_s} \resrho \\ \\
    \resrho \\ \\
    \res{\rho \vu} \\ \\
    \res{\rho E}
  \end{pmatrix}
  \label{fc-dc-res}
\end{equation}
%------------------------------------------------------------------------------%
\eref{fc-dc-res} shows that $\rv{}$ is completely comprised of components from
$\ru{}$; therefore, a relationship between these equation sets can be derived in
a similar fashion to Appendix \ref{change-of-var-section}.  Rewriting $\ru{}$ in
terms of $\rv{}$
%------------------------------------------------------------------------------%
\begin{equation}
  \ru{} =
  \begin{pmatrix}
    \rv{i} + c_i \left(\rv{N_s+1} \right) \\
    \rv{N_s+2} \\
    \rv{N_s+3}
  \end{pmatrix}
  \label{ru-to-rv}
\end{equation}
%------------------------------------------------------------------------------%
It is possible to form the transformation
%------------------------------------------------------------------------------%
\begin{equation}
  \pd{\ru{}}{\rv{}} =
  \begin{pmatrix}
    1      & \dots  & 0 & c_1        & 0 & 0 \\
    \vdots & \ddots & \vdots            & \vdots & \vdots & \vdots & \vdots \\
    0      & \dots  & 1 & c_{N_{ns}} & 0 & 0 \\
    0      & \dots  & 0 & 0          & 1 & 0 \\
    0      & \dots  & 0 & 0          & 0 & 1 \\
  \end{pmatrix}
  \label{drudrv}
\end{equation}
%------------------------------------------------------------------------------%
The adjoint costate variable vector $\adjlam{}$ is effectively defined as
%------------------------------------------------------------------------------%
\begin{equation}
  \adjlam{} = \pd{f}{\mr}
  \label{adj-lam-dfdr}
\end{equation}
%------------------------------------------------------------------------------%
Thus, there are two different costate variable vectors for the equation sets
defined in \eref{ru-to-rv}
%------------------------------------------------------------------------------%
\begin{align}
  \adjlam{\mU} = \pd{f}{\ru{}}
  \label{adj-lam-ru} \\[12pt]
  \adjlam{\mv} = \pd{f}{\rv{}}
  \label{adj-lam-rv}
\end{align}
%------------------------------------------------------------------------------%
based on \erefs{adj-lam-ru}{adj-lam-rv}, it clear that the transformation
defined in \eref{drudrv} is the mapping between these costate variable sets
%------------------------------------------------------------------------------%
\begin{equation}
  \adjlam{\mU} = \pd{\ru{}}{\rv{}} \adjlam{\mv}
  \label{adj-ru-to-rv}
\end{equation}
%------------------------------------------------------------------------------%
\eref{adj-ru-to-rv} enables a right preconditioning of the adjoint system of
equations, where the costate variables associated with the decoupled system of
equations can be transformed into the costate variables associated with fully
coupled system of equations, as post-processing step.  This also decreases the
length of the solution vector storage required, since the fully coupled system
has one less equation than the decoupled system.



%%---------------------------------------------------------------------------%%
\backmatter

\end{document}
