\documentclass[11pt,          % font size: 11pt or 12pt
               phd,           % degree:    ms or phd
               onehalfspacing % spacing: onehalfspacing or doublespacing
               ]{ncsuthesis}

%%----------------------------------------------------------------------------%%
%%------------------------------ Import Packages -----------------------------%%
%%----------------------------------------------------------------------------%%

\usepackage{booktabs}  % professionally typeset tables
\usepackage{amsmath}
\usepackage{amssymb}
\usepackage{textcomp}  % better copyright sign, among other things
\usepackage{xcolor}
\usepackage{lipsum}    % filler text
\usepackage{mathtools}
\usepackage{geometry}
\usepackage{graphicx}
\usepackage{relsize}
\usepackage{environ}
\usepackage{newlfont}
\usepackage{caption}
\usepackage{subcaption}
\usepackage{titlesec}
\usepackage[export]{adjustbox}
\usepackage[position=b]{subcaption}
\usepackage{url}
\urlstyle{same}


%%----------------------------------------------------------------------------%%
%%---------------------------- Formatting Options ----------------------------%%
%%----------------------------------------------------------------------------%%
%%

%% -------------------------------------------------------------------------- %%
%% Disposition format -- any titles, headings, section titles
%%  These formatting commands affect all headings, titles, headings,
%%  so sizing commands should not be used here.
%%  Formatting options to consider are
%%     +  \sffamily - sans serif fonts.  Dispositions are often typeset in
%%                    sans serif, so this is a good option. 
%%     +  \rmfamily - serif fonts
%%     +  \bfseries - bold face
%\dispositionformat{\sffamily\bfseries}   % bold and sans serif
\dispositionformat{\bfseries}            % bold and serif

%% -------------------------------------------------------------------------- %%
%% Formatting for centered headings - Abstract, Dedication, etc. headings
%%  This is where one might put a sizing command.
%%  \MakeUppercase can be used to typeset all headings in uppercase.
\headingformat{\large\MakeUppercase}   % All letters uppercase
%\headingformat{\large}                % Not all uppercase
%\headingformat{\Large\scshape}        % Small Caps, used with serif fonts.

%% Typographers recommend using a normal inter-word space after
%% sentences. TeX's default is to add an wider space, but \frenchspacing
%% gives a normal spacing. Comment out the following line if you prefer
%% wider spaces between sentences.
\frenchspacing


%% -------------------------------------------------------------------------- %%
%%  Optional packages
%%    A number of compatible packages to improve the look and feel of
%%    your document are available in the file optional.tex 
%%    (For example, hyperlinks, fancy chapter headings, and fonts)
%% To use these options, uncomment the next line and see optional.tex
%\include{optional}


%%----------------------------------------------------------------------------%%
%%---------------------------- Content Options -------------------------------%%
%%----------------------------------------------------------------------------%%
%% Size of committee: 3, 4, 5, or 6 -- this number includes the chair
\committeesize{5}

%% Members of committee
%%  Each of the following member commands takes an optional argument
%%   to specify their role on the committee.
%%  For co-chairs, use the commands:
%%      \cochairI{Doug Dodd}
%%      \cochairII{Chris Cox}
%%
\cochairI{Hassan Hassan}
\cochairII{Peter Gnoffo}
\memberI{Jack Edwards}
\memberII{Hong Luo}
\memberIII{John Griggs}


%% Student writing thesis, \student{First Middle}{Last}
\student{Kyle B.}{Thompson} % a middle initial

%% Degree program
\program{Aerospace Engineering}

%% Thesis Title
%%  Keep in mind, according to ETD guidelines:
%%    +  Capitalize first letter of important words.
%%    +  Use inverted pyramid shape if title spans more than one line.
%%
%%  Note: To break the title onto multiple lines, use \break instead of \\.
\thesistitle{Aerothermodynamic Design Sensitivities for a Reacting Gas Flow Solver \break
on an Unstructured Mesh Using a 
Discrete Adjoint Formulation}

%% Degree year.  Necessary if your degree year doesn't equal the current year.
%\degreeyear{1995}


%%----------------------------------------------------------------------------%%
%%---------------------------- Personal Macros -------------------------------%%
%%----------------------------------------------------------------------------%%

\newcommand{\uv}[1]{\ensuremath{\mathbf{\hat{#1}}}}
\newcommand{\bo}{\ensuremath{\boldsymbol{\Omega}}}
\newcommand{\fref}[1]{Figure~\ref{#1}}
\newcommand{\tref}[1]{Table~\ref{#1}}
\newcommand{\eref}[1]{Eq.~\ref{#1}}
\newcommand{\erefs}[2]{Eq.s~(\ref{#1}-\ref{#2})}
\newcommand{\pd}[2]{\frac{\partial #1}{\partial #2}}
\newcommand{\pnd}[3]{\frac{\partial^{#3} #1}{\partial #2^{#3}}}

% residual and its derivatives
\newcommand{\res}[1]{\vR_{#1}}
\newcommand{\rdiff}[2]{\pd{\vR_{#1}}{#2}}
\newcommand{\rtdiff}[2]{\pd{\vR_{#1}}{#2}^{\mathsmaller T}}

% Sum with limits on top and bottom of sigma
\newcommand{\lsum}[3]{\sum\limits_{#1}^{#2}{#3}}

% adjoint lambda with subscript
\newcommand{\adjlam}[1]{\Lambda_{#1}}

\newcommand{\real}{\mathbf R}
\newcommand{\rnn}{\real^{N \times N}}
\newcommand{\rmn}{\real^{M \times N}}
\newcommand{\diag}{\mbox{diag}}
\newcommand{\vf}{\mathbf f}
\newcommand{\vF}{\mathbf F}
\newcommand{\vx}{\mathbf x}
\newcommand{\vc}{\mathbf c}
\newcommand{\vd}{\mathbf d}
\newcommand{\vy}{\mathbf y}
\newcommand{\vz}{\mathbf z}
\newcommand{\ve}{\mathbf e}
\newcommand{\me}{\mathbf E}
\newcommand{\mg}{\mathbf G}
\newcommand{\vw}{\mathbf w}
\newcommand{\mw}{\mathbf W}
\newcommand{\mr}{\mathbf R}
\newcommand{\mh}{\mathbf H}
\newcommand{\mc}{\mathbf C}
\newcommand{\ms}{\mathbf S}
\newcommand{\mi}{\mathbf I}
\newcommand{\vb}{\mathbf b}
\newcommand{\vq}{\mathbf q}
\newcommand{\va}{\mathbf a}
\newcommand{\vv}{\mathbf v}
\newcommand{\vu}{\mathbf u}
\newcommand{\vU}{\mathbf U}
\newcommand{\vR}{\mathbf R}
\newcommand{\Up}{\mathbf{U}'}
\newcommand{\Uhat}{\mathbf{\hat{U}}}
\newcommand{\ma}{\mathbf A}
\newcommand{\mb}{\mathbf B}
\newcommand{\md}{\mathbf D}
\newcommand{\vr}{\mathbf r}
\newcommand{\mm}{\mathbf M}
\newcommand{\ml}{\mathbf L}
\newcommand{\mU}{\mathbf U}
\newcommand{\ul}{\mathbf{U^L}}
\newcommand{\ur}{\mathbf{U^R}}
\newcommand{\mv}{\mathbf V}
\newcommand{\mP}{\mathbf P}
\newcommand{\mq}{\mathbf Q}
\newcommand{\mx}{\mathbf X}
\newcommand{\pbar}{\overline p}

%%---------------------------------------------------------------------------%%
\begin{document}

%%---------------------------------------------------------------------------%%
\frontmatter

%% ------------------------------ Abstract ---------------------------------- %%
\begin{abstract}

  An algorithm is described to efficiently compute aerothermodynamic design
sensitivities using a decoupled variable set.  In a conventional approach to
computing design sensitivities for reacting flows, the species continuity
equations are fully coupled to the conservation laws for momentum and energy. In
this algorithm, the species continuity equations are solved separately from the
mixture continuity, momentum, and total energy equations. This decoupling
simplifies the implicit system, so that the flow solver can be made
significantly more efficient, with very little penalty on overall scheme
robustness.  Most importantly, the computational cost of the point implicit
relaxation is shown to scale linearly with the number of species for the
decoupled system, whereas the fully coupled approach scales quadratically. Also,
the decoupled method significantly reduces the cost in wall time and memory in
comparison to the fully coupled approach. 

This decoupled approach for computing design sensitivities with the adjoint
system is demonstrated for inviscid flow in chemical non-equilibrium around a
re-entry vehicle with a retro-firing annular nozzle. The sensitivities of the
surface temperature and mass flow rate through the nozzle plenum are computed
with respect plenum conditions and verified against complex-variable
finite-difference.  The decoupled scheme significantly improved the
computational time and memory required to complete the optimization, making this
an attractive method for high-fidelity design of hypersonic vehicles.

\end{abstract}


%% ---------------------------- Copyright page ------------------------------ %%
%% Comment the next line if you don't want the copyright page included.
\makecopyrightpage

%% -------------------------------- Title page ------------------------------ %%
\maketitlepage

%% -------------------------------- Dedication ------------------------------ %%
\begin{dedication}
 \centering To my parents and friends.
\end{dedication}

%% -------------------------------- Biography ------------------------------- %%
\begin{biography}
The author was born on December 11th, 1990, in Willow Springs, North Carolina.
He recieved a B.S. in Aerospace Engineering from North Carolina State University
in 2012, and subsequently recieved his M.S. in Aerospace Engineering from North
Carolina State University in 2014.  After recieving his M.S., he began work in
the Aerothermodynamics branch of NASA Langley Research Center, via the Pathways
program.  He completed this disseration on adjoint-based optimization while
working at NASA Langley Research Center.
\end{biography}

%% ----------------------------- Acknowledgements --------------------------- %%
\begin{acknowledgements}
First, I would like to thank Dr. Peter Gnoffo, who has been an unending source
of support during my time NASA Langley Research Center.  I would like thank my
parents for all of their encouragement over the years, and their commitment to
seeing I be given the best opportunity to make the most of myself.  I would like
to thank my advisor, Dr. Hassan, for instilling in me a diligence to improve
myself and for mentoring me through many challenges.  I would like to thank Dr.
Jeff White in the Computational Aerosciences Branch at NASA Langley Research
Center for his advice on reacting flow simulation. I would like to thank the
Entry Systems Modeling Project within NASA's Game Changing Development Program
for their funding and support of this research.  Finally, I would like to
recognize the FUN3D team at NASA Langley Research Center, for their support and
availability to discuss many challenging problems I have encountered during the
course of my time at NASA.
\end{acknowledgements}

\thesistableofcontents

\thesislistoftables

\thesislistoffigures


%%---------------------------------------------------------------------------%%
\mainmatter

\chapter{Introduction}
\label{chapter-one}

In the last decades computational fluid dynamics (CFD) codes have matured to the
point that it is possible to obtain high-fidelity design sensitivities that can
be coupled with optimization packages to enable design optimization for a
variety of design inputs\cite{baysal1992aerodynamic, balagangadhar2001design}.
In recent years, the benefits of using an adjoint-based formulation to compute
sensitivities have been realized and implemented in many compressible CFD
codes\cite{mavriplis-2006, nemec-aftosmis-adjoint, nielsen2002recent}, because
of the ability compute all sensitivities a the cost of a single extra adjoint
solution, instead of an additional flow solution for each design variable.
Reacting gas CFD codes have lagged significantly in adopting this adjoint-based
approach, with only a small number of codes having published
results\cite{Copeland, Barcelona}.  This is likely due to the significant
jump in complexity of the linearizations required, especially in regard to the
chemical source term and the dissipation term in the Roe flux difference
splitting (FDS) scheme\cite{roe}.

An additional obstacle that may prevent adjoint-based sensitivity analysis in
reacting gas solvers is the extreme problem size associated with high energy
physics.  The additional equations required in reacting gas simulations lead to
large Jacobians that scale quadratically in size to the number of governing
equations.  This leads to a significant increase in the memory required to store
the flux linearizations and the computational cost of the point solver.  As
reacting gas CFD solvers are used to solve increasingly more complex problems,
this onerous quadratic scaling of computational cost and Jacobian size will
ultimately surpass the current limits of hardware and time constraints on
achieving a flow solution\cite{fischer}.

To mitigate this scaling issue, Candler et al.\cite{candler} proposed a scheme
to for a modified form of the Steger-Warming flux vector splitting
scheme\cite{MacCormack,Steger}. In that work, it was shown that quadratic
scaling between the cost of solving the implicit system and adding species mass
equations can be reduced from quadratic to linear scaling by decoupling the
species mass equations from the mixture mass, momentum, and energy equations and
solving the two systems sequentially.  This work extends the aforementioned work
from the modified form of the Steger-Warming flux vector splitting
method to the Roe flux difference splitting (FDS) scheme.

The work presented demonstrates that this decoupling can be applied to both the
flow solver and adjoint solver in a reacting gas CFD code, and significantly
improve the efficiency in both computational cost and memory required.
Additionally the implementation of exact linearizations is shown to
significantly improve performance and robustness in the flow solver over common
approximations used when linearizing the Roe FDS scheme.  The formulation and
linearization of the fluxes for the Roe FDS scheme are presented here, and
design optimization for an inviscid reacting flow is conducted for inviscid,
reacting flow around an axi-symmetric hypersonic re-entry vehicle with an
annular jet.


\chapter{Adjoint Solver}
\label{chapter-two}

\section{Adjoint Derivation}

The FUN3D adjoint derivation is given in Eric Nielson's PhD Thesis.  Starting
rith forming the Lagrangian as
%------------------------------------------------------------------------------%
\begin{equation}
  L(\md,\mq,\mx,\mathbf{\Lambda})=f(\md,\mq,\mx)
  +\mathbf{\Lambda}^T\mr(\md,\mq,\mx)
\end{equation}
%------------------------------------------------------------------------------%
Where $\mr$ is the residual of the flow equations.  Differentiating with respect
to the design variables $\md$ yields
%------------------------------------------------------------------------------%
\begin{equation}
  \frac{\partial L}{\partial \md}=
  \Bigg\{\frac{\partial f}{\partial \md}+\bigg[\frac{\partial \mx}{\partial \md}\bigg]^T \frac{\partial f}{\partial \mx}\Bigg\}
  +\bigg[\frac{\partial \mq}{\partial \md}\bigg]^T
  \Bigg\{\frac{\partial f}{\partial \mq}+\bigg[\frac{\partial \mr}{\partial \mq}\bigg]^T \mathbf{\Lambda}\Bigg\}
  +\Bigg\{\bigg[\frac{\partial \mr}{\partial \md}\bigg]^T
  +\bigg[\frac{\partial \mx}{\partial \md}\bigg]^T\bigg[\frac{\partial \mr}{\partial \mx}\bigg]^T\Bigg\}\mathbf{\Lambda}
  \label{dL}
\end{equation}
%------------------------------------------------------------------------------%
To eliminate the dependence of conserved variables $\mq$ on the design
variables, we solve the adjoint equation
%------------------------------------------------------------------------------%
\begin{equation}
  \bigg[\frac{\partial \mr}{\partial \mq}\bigg]^T\mathbf{\Lambda}=
  -\frac{\partial f}{\partial \mq}
  \label{adjoint-main}
\end{equation}
%------------------------------------------------------------------------------%
Where the Lagrange multipliers (also known as costate variables),
$\mathbf{\Lambda}$ are the cost function dependence on the residual
%------------------------------------------------------------------------------%
\begin{equation}
  \mathbf{\Lambda}=-\frac{\partial f}{\partial \mr}
\end{equation}
%------------------------------------------------------------------------------%
This can ultimately be used to error estimation and sensitivity analysis for
design optimization.  With the second term in eq.~(\ref{dL}) eliminated, the
derivative of the Lagrangian becomes
%------------------------------------------------------------------------------%
\begin{equation}
  \frac{\partial L}{\partial \md}=
  \Bigg\{\frac{\partial f}{\partial \md}+\bigg[\frac{\partial \mx}{\partial
  \md}\bigg]^T \frac{\partial f}{\partial \mx}\Bigg\}
  +\Bigg\{\bigg[\frac{\partial \mr}{\partial \md}\bigg]^T
  +\bigg[\frac{\partial \mx}{\partial \md}\bigg]^T\bigg[\frac{\partial
  \mr}{\partial \mx}\bigg]^T\Bigg\}\mathbf{\Lambda}
  \label{obj_function}
\end{equation}
%------------------------------------------------------------------------------%
By solving the adjoint equation in \eref{adjoint-main}) to obtain the costate
variable vector, $\mathbf{\Lambda}$, we can now use a non-linear optimizer to
determine the optimum set of design variables, $\md^*$. This can be done using
{\bf PORT} or {\bf KSOPT} in FUN3D, as well as a host of other non-linear
optimizers.

\section{Decoupled Adjoint}

The purpose of this is to show the relationship between the costate variable for total density $\lambda_\rho$ to the costate variables for species densities $\lambda_{\rho_s}$. Beginning with the definition of the Adjoint Equations:
%------------------------------------------------------------------------------%
\begin{equation}
  \left(\frac{\partial R}{\partial Q}\right)^T\lambda = \frac{\partial F}{\partial Q}
  \label{adj_eqn}
\end{equation}
%------------------------------------------------------------------------------%
Where the $R$ is the residual of the governing equations, $Q$ is the vector of conserved variables, and $F$ is the cost function (i.e. lift, drag, etc.). Note that the first term is simply the transpose of the jacobian multiplied by the costate variable vector $\lambda$, and can be written as:
%------------------------------------------------------------------------------%
\begin{equation}
  \left(\frac{\partial R}{\partial Q}\right)_i^T \lambda 
  = \sum_{j=1}^{N_{eq}}{
    \left(\frac{\partial R_j}{\partial Q_i}\right) \lambda_j}
  \label{lhs_sum}
\end{equation}
%------------------------------------------------------------------------------%
Suppose we define the system of equations in two different ways.  The first system, which we'll call the ``meanflow system'', consists of 5 equations:
%------------------------------------------------------------------------------%
\begin{equation}
  R = \begin{pmatrix} 
        R_{\rho} \\ R_{\rho u} \\ R_{\rho v} \\ R_{\rho w} \\ R_{\rho E}
      \end{pmatrix}, \quad
      \lambda = \begin{pmatrix}
        \lambda_\rho \\ \lambda_{\rho u} \\ \lambda_{\rho v} \\ \lambda_{\rho w} \\
        \lambda_{\rho E}
      \end{pmatrix}
  \label{5x5}
\end{equation}
%------------------------------------------------------------------------------%
The second system consists of the full system of equations:
%------------------------------------------------------------------------------%
\begin{equation}
  R = \begin{pmatrix} 
        R_{\rho_1} \\ \vdots \\ R_{\rho_s} \\ R_{\rho u} \\
        R_{\rho v} \\ R_{\rho w} \\ R_{\rho E}
      \end{pmatrix}, \quad
      \lambda = \begin{pmatrix}
        \lambda_{\rho_1} \\ \vdots \\ \lambda_{\rho_s} \\
        \lambda_{\rho u} \\ \lambda_{\rho v} \\ \lambda_{\rho w} \\
        \lambda_{\rho E}
      \end{pmatrix}
  \label{full_sys}
\end{equation}
%------------------------------------------------------------------------------%
By making the approximation that the mass fraction $c_s$ is constant, we can show that the full system of equations reduces to the meanflow system.  By this approximation the derivatives with respect to species density become:
%------------------------------------------------------------------------------%
\begin{equation}
  \frac{\partial R}{\partial \rho} =
  \frac{\partial R}{\partial \rho_s} 
  \frac{\partial \rho_s}{\partial \rho} =
  c_s\left(\frac{\partial R}{\partial \rho_s}\right)
  \label{mass_frac_approx}
\end{equation}
%------------------------------------------------------------------------------%
Thus, for a single row of the full system:
%------------------------------------------------------------------------------%
\begin{equation}
  \left(\frac{\partial R}{\partial Q}\right)_{\rho_s}^T \lambda 
  = \sum_{j=1}^{N_{full}}{
    \left(\frac{\partial R_j}{\partial \rho}\right) \frac{\lambda_j}{c_s}}
    = \frac{1}{c_s}\left(\frac{\partial F}{\partial \rho}\right)
  \label{full_reduction}
\end{equation}
%------------------------------------------------------------------------------%
After cancelling the mass fractions, this allows the first row of the full system to be equated to the first row of the meanflow system:
%------------------------------------------------------------------------------%
\begin{equation}
  \sum_{j=1}^{N_{full}}{
    \left(\frac{\partial R_j}{\partial \rho}\right) \lambda_j}
  = \sum_{j=1}^{N_{meanflow}}{
    \left(\frac{\partial R_j}{\partial \rho}\right) \lambda_j}
  \label{eq_mean_full}
\end{equation}
%------------------------------------------------------------------------------%
Expanding this out, it becomes clear many terms cancel:
\begin{multline}
  \frac{\partial R_{\rho_1}}{\partial \rho}\lambda_{\rho_1} +
  \dots +
  \frac{\partial R_{\rho_s}}{\partial \rho}\lambda_{\rho_s} +
  \frac{\partial R_{\rho u}}{\partial \rho}\lambda_{\rho u} +
  \frac{\partial R_{\rho v}}{\partial \rho}\lambda_{\rho v} +
  \frac{\partial R_{\rho w}}{\partial \rho}\lambda_{\rho w} +
  \frac{\partial R_{\rho E}}{\partial \rho}\lambda_{\rho E} = \\
  \frac{\partial R_{\rho}}{\partial \rho}\lambda_{\rho} +
  \frac{\partial R_{\rho u}}{\partial \rho}\lambda_{\rho u} +
  \frac{\partial R_{\rho v}}{\partial \rho}\lambda_{\rho v} +
  \frac{\partial R_{\rho w}}{\partial \rho}\lambda_{\rho w} +
  \frac{\partial R_{\rho E}}{\partial \rho}\lambda_{\rho E}
  \label{expand_row}
\end{multline}
%------------------------------------------------------------------------------%
\begin{equation}
  \frac{\partial R_{\rho_1}}{\partial \rho}\lambda_{\rho_1} +
  \dots +
  \frac{\partial R_{\rho_s}}{\partial \rho}\lambda_{\rho_s} =
  \frac{\partial R_{\rho}}{\partial \rho}\lambda_{\rho}
  \label{simpl_ex_row}
\end{equation}
%------------------------------------------------------------------------------%
Finally, because the individual species mass fluxes must sum to the total mass flux:
%------------------------------------------------------------------------------%
\begin{equation}
  \sum_{s=1}^{N_{species}}{R_{\rho_s}} = R_{\rho}
  \label{sp_sum}
\end{equation}
%------------------------------------------------------------------------------%
Eqn (\ref{simpl_ex_row}) can be rewritten as:
%------------------------------------------------------------------------------%
\begin{equation}
  \frac{\partial R_{\rho_1}}{\partial \rho}\lambda_{\rho_1} +
  \dots +
  \frac{\partial R_{\rho_s}}{\partial \rho}\lambda_{\rho_s} =
  \frac{\partial R_{\rho_1}}{\partial \rho}\lambda_{\rho} +
  \dots +
  \frac{\partial R_{\rho_s}}{\partial \rho}\lambda_{\rho}
  \label{near_final}
\end{equation}
%------------------------------------------------------------------------------%
Which implies that the species mass costate variables are all equal to the total mass costate variable, yielding:
%------------------------------------------------------------------------------%
\begin{align}
  \lambda_{\rho} &= \lambda_{\rho_s} \\
  d \lambda_{\rho} &= d \lambda_{\rho_s}
  \label{final_result}
\end{align}
%------------------------------------------------------------------------------%
\pagebreak

\chapter{Numerical Solution of Flow Equations}
\label{chapter-three}

In this section, the discretization and method of solution of the governing
equations from \cref{chapter-two} is derived.  Traditionally, the reacting gas
path of FUN3D\cite{biedron2016fun3d} has employed an implicit, fully coupled
scheme in the flow solver.  In addition to providing details of this fully
coupled scheme, a decoupled scheme based on the Roe FDS scheme\cite{roe} is
derived to improve the computational efficiency of the flow solver and decrease
the relative memory required.  Details on the reconstruction scheme used in
FUN3D to extend the baseline finite-volume scheme to higher-order accuracy are
also provided in this section.

\section{Fully-Coupled Point Implicit Method}

The governing equations presented in \erefs{species-cons}{tot-energy-cons} can
be recast in vector form as
%------------------------------------------------------------------------------%
\begin{equation}
	\label{inv_flux_vec}
  \pd{\mU}{t}\vol + \nabla \cdot \vF = \mw \vol
\end{equation}
%------------------------------------------------------------------------------%
 or, in semi-discrete form,
%------------------------------------------------------------------------------%
\begin{equation}
	\label{inv_flux_fv}
  \pd{\mU}{t}\vol + \lsum{f}{}{(\vF \cdot \Norm)^f} = \mw \vol
 \end{equation}
%------------------------------------------------------------------------------%
summing over all faces, $f$, in the domain, where $\vol$ is the cell volume, 
$\mathbf{W}$ is the chemical source term vector, and $\mathbf{N}$ is the face
outward normal vector.  The vectors of conserved variables and fluxes are:
%------------------------------------------------------------------------------%
\begin{equation}
	\begin{matrix}
	\mathbf{U}=\begin{pmatrix}
   		\rho_1\\
		\vdots \\
		\rho_{ns} \\
		\rho \vu \\
		\rho E \\
	\end{pmatrix},      &
 	\mathbf{F} = \begin{pmatrix}
		\rho_1  \overline{U} \\
		\vdots \\
		\rho_{ns} \overline{U} \\
		\rho \vu \overline{U} + p \norm\\
		(\rho E + p) \overline{U} \\
	\end{pmatrix}
	\end{matrix}
  \label{fc-variables}
 \end{equation}
%------------------------------------------------------------------------------%
where $\overline{U}$ is the outward pointing normal velocity, $E$ is
the total energy of the mixture per unit mass as defined in
\eref{tot-energy-def}.  The flux and species chemical
source term at the next time level can be approximated as
%------------------------------------------------------------------------------%
\begin{align}
  \vF^{n+1} &\approx \vF^n + \pd{\vF}{\mU} \Delta \mU^n \\
  \mw^{n+1} &\approx \mw^n + \pd{\mw}{\mU} \Delta \mU^n
  \label{fc-fluxes-timelevel}
\end{align}
%------------------------------------------------------------------------------%
where $\Delta \mU^n = \mU^{n+1} - \mU^{n}$.  Using an implicit time integration,
the implicit scheme becomes:
%------------------------------------------------------------------------------%
\begin{equation}
	  \frac{\vol}{\Delta t} \Delta \mU^n
    +\lsum{f}{}{
      \left(\pd{\vF^f}{\ul} \Delta \ul + \pd{\vF^f}{\ul} \Delta \ur \right)^n 
      \Norm^f
    } - \vol \pd{\mw}{\mU}
  \Delta \mU^n
  = - \lsum{f}{}{\left( \vF^f \cdot \Norm^f \right)^n} + \vol \mw^n
  \label{fc-explicit}
\end{equation}
%------------------------------------------------------------------------------%
\eref{fc-explicit} results in a global Jacobian comprised of block Jacobians
from the system
%------------------------------------------------------------------------------%
\begin{equation}
  \underbrace{
    \left[ 
      \frac{\vol}{\Delta t} \mi + 
      \begin{pmatrix}
        \rdiff{\rho_1}{\rho_1}     & \dots  & \rdiff{\rho_1}{\rho_{N_s}}     & \rdiff{\rho_1}{\rho \vu}     & \rdiff{\rho_1}{\rho E}      \\
        \vdots                     & \ddots & \vdots                         & \vdots                       & \vdots                      \\
        \rdiff{\rho_{N_s}}{\rho_1} & \dots  & \rdiff{\rho_{N_s}}{\rho_{N_s}} & \rdiff{\rho_{N_s}}{\rho \vu} &  \rdiff{\rho_{N_s}}{\rho E} \\
        \rdiff{\rho \vu}{c_1}      & \dots  & \rdiff{\rho \vu}{c_{N_s}}      & \rdiff{\rho \vu}{\rho \vu}   &  \rdiff{\rho \vu}{\rho E}   \\
        \rdiff{\rho E}{c_1}        & \dots  & \rdiff{\rho E}{c_{N_s}}        & \rdiff{\rho E}{\rho \vu}     &  \rdiff{\rho E}{\rho E}
      \end{pmatrix}
    \right]
  }_\text{$(N_s + 4) \times (N_s + 4)$}
  \underbrace{
    \begin{pmatrix}
      \Delta \mU_{\rho_1}     \\
      \vdots          \\
      \Delta \mU_{\rho_{N_s}} \\
      \Delta \mU_{\rho \vu}   \\
      \Delta \mU_{\rho E}
    \end{pmatrix}
  }_\text{$(N_s + 4) \times 1$}
  =
  \underbrace{
    \begin{pmatrix}
      \res{\rho_1}     \\
      \vdots           \\
      \res{\rho_{N_s}} \\
      \res{\rho \vu}   \\
      \res{\rho E}
    \end{pmatrix}
  }_\text{$(N_s + 4) \times 1$}
  \label{drdu-fc-flow}
\end{equation}
%------------------------------------------------------------------------------%
with global system of equations written generically as
%------------------------------------------------------------------------------%
\begin{equation}
  \left( \frac{\vol}{\Delta t} \mi + \rdiff{\mU}{\mU} \right)\Delta \mU = \ru{}
  \label{fc-global}
\end{equation}
%------------------------------------------------------------------------------%
where $\ru{}$ is global residual vector, corresponding to the RHS of
\eref{fc-explicit}, and $\rdiff{\mU}{\mU}$ is global Jacobian matrix,
corresponding the linearizations in the LHS of \eref{fc-explicit}. This
non-linear system can be relaxed to steady state, where $\ru{} \approx 0$, by
means of point implicit relaxation.  The LHS of \eref{fc-global}, $\ma_{\mU}$,
is split into its diagonal and off-diagonal elements, with the latter moved to
the RHS:
%------------------------------------------------------------------------------%
\begin{equation}
  \left( \frac{\vol}{\Delta t} \mi + \rdiff{\mU}{\mU} \right) = 
  \ma_{\mU} = O+D
  \label{decomp-jac}
\end{equation}
%------------------------------------------------------------------------------%
As shown in \eref{drdu-fc-flow}, each block matrix element is a square
$(N_s+4)\times(N_s+4)$ matrix.  This system can be solved iteratively by using a
multi-color matrix ordering to perform a series
of Gauss-Seidel sweeps. The computational work for the Gauss-Seidel scheme is
dominated by matrix-vector multiplications of elements of $O$ with
$\Delta \mU$, which are $O((N_s + 4)^2)$ operations.  In the next
section, it is shown that decoupling the system reduces these matrix-vector
multiplications to $O(5^2 + N_s)$ operations.

\section{Decoupled Point Implicit Method}

If the species mass equations are replaced by a single mixture mass equation,
the mixture equations can be separated from the species mass
equations and the conserved variables become
%------------------------------------------------------------------------------%
\begin{equation}
	\mU' =
  \begin{pmatrix}
		\rho \\
		\rho \vu \\
		\rho E
	\end{pmatrix} \quad
	\Uhat =
  \begin{pmatrix}
		\rho_1 \\
		\vdots \\
		\rho_{ns}
  \end{pmatrix}
  \label{dc-variables}
\end{equation}
%------------------------------------------------------------------------------%
with corresponding flux vectors
%------------------------------------------------------------------------------%
\begin{equation}
 	\vF' = 
  \begin{pmatrix}
		\rho \overline{U} \\
		\rho \vu \overline{U} + p \norm \\
		(\rho E + p) \overline{U} \\
	\end{pmatrix} \quad
 	\Fhat = 
  \begin{pmatrix}
		\rho_1  \overline{U} \\
		\vdots \\
		\rho_{ns} \overline{U} \\
	\end{pmatrix}
  \label{dc-fluxes}
\end{equation}
%------------------------------------------------------------------------------%
Solving the flux vectors is performed in two sequential steps.  The mixture
fluxes, $\vF'$, are first solved as
%------------------------------------------------------------------------------%
\begin{equation}
  \pd{\Up}{t} \vol + \lsum{f}{}{\left(\vF' \cdot \Norm\right)^f} = 0
  \label{dc-flux-part-1}
\end{equation}
%------------------------------------------------------------------------------%
followed by the species fluxes, $\Fhat$, as
%------------------------------------------------------------------------------%
\begin{equation}
  \pd{\Uhat}{t} \vol + \lsum{f}{}{\left(\Fhat \cdot \Norm\right)^f} = \vol \mw
  \label{dc-flux-part-2}
\end{equation}
%------------------------------------------------------------------------------%
The same point-implicit relaxation that uses multi-color Gauss-Seidel sweeps is
used to update the conserved variables in $\mU '$, and all associated
auxiliary variables, such as temperature, pressure, speed of sound, etc. are
updated to be consistent with the new state of $\mU '$.  This update is done
holding the mass fraction state constant, and will always result in the
relaxation of a five-equation system.  Decoupling the variable sets in
\eref{dc-variables} does trade an implicit relationship between the mixture and
species equations for an explicit one; thus, this decoupling can have an impact
on the stability of the scheme, especially due to the non-linearity of the
chemical source term\cite{park}.
 
The solution of the species mass equations takes a different form.  Based on the
work of Candler et al.\cite{candler}, the decoupled variables can be rewritten
in terms of mass fraction, as follows:
%------------------------------------------------------------------------------%
\begin{equation} 
  \delta \Uhat^n 
  = \rho^{n+1} \Vhat^{n+1} - \rho^n\Vhat^n 
  = \rho^{n+1} \delta \Vhat^n + \Vhat^n \delta \rho^n
  \label{du-to-dv-mass-frac}
\end{equation}
%------------------------------------------------------------------------------%
where $\mathbf{\hat{V}}=(c_1,\hdots,c_{ns})^T$, and $c_s=\rho_s/\rho$ is the
mass fraction of species $s$.  While the derivation of the species mass
equations is different for the Roe FDS scheme from that of Steger-Warming
proposed by Candler et al.\cite{candler}, the final result takes a similar form: 
%------------------------------------------------------------------------------%
\begin{gather}
  \hat{F}_{\rho_s} 
  = c_s F'_\rho+(c_s^L-\tilde{c}_s)\rho^L\lambda^+
  + (c_s^R-\tilde{c}_s)\rho^R\lambda^-
  \label{dc-flux}
\end{gather}
%------------------------------------------------------------------------------%
where $F_\rho'$ is the total mass flux computed previously using all
$\mathbf{U}'$ variables, $\tilde{}$ denotes a Roe-averaged quantity, and
$\lambda^{+/-}$ are functions of the Roe averaged eigenvalues.  Likewise,
linearizing the species mass fluxes with respect to the $\mathbf{\hat{V}}$
variables yields
%------------------------------------------------------------------------------%
\begin{align} 
  \Fhat^{n+1} &\approx
  \Fhat^n 
  + \pd{\Fhat}{\Vhat^L} \Delta \Vhat^L 
  + \pd{\Fhat}{\Vhat^R} \Delta \Vhat^R
  \label{flux-dc-taylor} \\
  \pd{\Fhat}{\Vhat^L} &= 
  \roe F_\rho+(1-\roe)\rho^L\lambda^+ - \roe \rho^R\lambda^- 
  \label{dfdvl} \\
  \pd{\Fhat}{\Vhat^R} &= 
  ( 1-\roe )F_\rho+(\roe -1)\rho^L\lambda^+ + \roe \rho^R\lambda^- 
  \label{dfdvr}
\end{align}
%------------------------------------------------------------------------------%
A full derivation of \erefs{dc-flux}{dfdvr}, along with the definition of
$\roe$, is included in \aref{decoupled-flux-derivation}. The chemical source
term is linearized in the same manner as the fully coupled scheme; however, the
updated $\mathbf{U}'$ variables are used to evaluate the Jacobian, and the chain
rule is applied to linearize $\mathbf{\hat{W}}$ with respect to the species mass
fractions:
%------------------------------------------------------------------------------%
\begin{equation} 
  \What^{n+1} \approx \What^n + 
  \pd{\What}{\mU} \bigg|_{\mU '} 
  \pd{\mU}{\Vhat}
  \label{source-term-linearization}
\end{equation}
%------------------------------------------------------------------------------%
For simplicity of notation, we define
%------------------------------------------------------------------------------%
\begin{equation} 
  C = \pd{\What}{\mU} \bigg|_{\mathbf{U}'} \pd{\mU}{\Vhat}
  \label{c-source-term}
\end{equation}
%------------------------------------------------------------------------------%
The decoupled system to be solved becomes:
%------------------------------------------------------------------------------%
\begin{equation} 
  \begin{split}
    \frac{\vol}{\Delta t} \rho^{n+1} \Delta\Vhat + 
    \lsum{f}{}{ \left( 
      \pd{\Fhat^f}{\Vhat^L} \cdot \Norm^f \Delta \Vhat^L 
    + \pd{\Fhat^f}{\Vhat^R} \cdot \Norm^f \Delta \Vhat^R 
    \right)^{n,n+1}} 
    - \vol C^{n,n+1} \Delta \Vhat^n \\ 
    = -\lsum{f}{}{
      \left( \Fhat^{n,n+1} \cdot \Norm \right)^f 
    } 
    + \vol \mw^{n, n+1} + R_\rho \Vhat^n
  \end{split}
  \label{decoupled-no-scaling}
\end{equation}
%------------------------------------------------------------------------------%
where
%------------------------------------------------------------------------------%
\begin{equation}
  R_\rho =
  \lsum{f}{}{\lsum{s=1}{N_s}{(\hat{F}_{\rho_s}^{n,n+1}\cdot\mathbf{N})}} 
  \label{dc-constraint}
\end{equation}
%------------------------------------------------------------------------------%
is included to preserve the constraint that the mass fractions
sum to unity, i.e., $\sum\limits_{s}{c_s}=1$, $\sum\limits_{s}{\delta c_s}=0$

Candler et al.\cite{candler2013analysis} later found that the decoupled scheme,
using a modified Steger-Warming flux splitting, was less stable than the fully
coupled scheme when large reaction rates were present, particularly for
exothermic reaction.  This same instability was observed in numerical
experiments for \eref{decoupled-no-scaling}.  To mitigate this issue, Candler et
al. proposed an alternative decoupling of variable sets that was more stable,
particularly if endothermic reaction dominated the flow.  Instead of
implementing this alternative decoupled scheme, \eref{decoupled-no-scaling} is
modified with a scalar scaling factor, $\omega_r$, on the chemical source term
$\mw^{n,n+1}$
%------------------------------------------------------------------------------%
\begin{equation} 
  \begin{split}
    \frac{\vol}{\Delta t} \rho^{n+1} \Delta\Vhat + 
    \lsum{f}{}{ \left( 
      \pd{\Fhat^f}{\Vhat^L} \cdot \Norm^f \Delta \Vhat^L 
    + \pd{\Fhat^f}{\Vhat^R} \cdot \Norm^f \Delta \Vhat^R 
    \right)^{n,n+1}} 
    - \vol C^{n,n+1} \Delta \Vhat^n \\ 
    = -\lsum{f}{}{
      \left( \Fhat^{n,n+1} \cdot \Norm \right)^f 
    } 
    + (\vol)(\omega_r)(\mw^{n, n+1}) + R_\rho \Vhat^n
  \end{split}
  \label{decoupled-with-scaling}
\end{equation}
%------------------------------------------------------------------------------%
Ramping this scaling factor from zero to one over the course of timestepping the
solution to steady-state preserves robustness of the decoupled scheme, and
numerical experiments show that it has minimal impact on the convergence rate.
Typically, the large reaction rates are only present during the transients early
on in the simulation, and damping the source term during this phase does not
affect the final result, provided the ramping is completed before the end of the
simulation.

In this decoupled scheme, the mixture and species continuity equations for
single point in the global system of the decoupled flow solve can be written as
%------------------------------------------------------------------------------%
\begin{equation}
  \begin{gathered}
    \text{\bf{Mixture Equations}}: \\
    \left[ 
    \frac{V}{\Delta t}\mi + 
    \begin{pmatrix}
      \rdiff{\rho}{\rho} & \rdiff{\rho}{\rho \vu} & \rdiff{\rho}{\rho E} \\
      \rdiff{\rho \vu}{\rho} & \rdiff{\rho \vu}{\rho \vu} & \rdiff{\rho \vu}{\rho E} \\
      \rdiff{\rho E}{\rho} & \rdiff{\rho E}{\rho \vu} & \rdiff{\rho E}{\rho E}
    \end{pmatrix}
    \right]
  \begin{pmatrix}
    \Delta \rho \\
    \Delta \rho \vu \\
    \Delta \rho E
  \end{pmatrix}
  =
  \begin{pmatrix}
    \resrho \\
    \res{\rho \vu} \\
    \res{\rho E}
  \end{pmatrix}
  \end{gathered}
  \label{approx-jac}
\end{equation}
%------------------------------------------------------------------------------%
\begin{equation}
  \begin{gathered}
    \text{\bf{Species Continuity Equations}}: \\
    \left[
      \frac{\rho V}{\Delta t}\mi + 
      \begin{pmatrix}
        \rdiff{\rho_1}{c_1} & \cdots & \rdiff{\rho_{1}}{c_{ns}} \\
        \vdots & \ddots & \vdots \\
        \rdiff{\rho_{ns}}{c_1} & \cdots & \rdiff{\rho_{ns}}{c_{ns}}
      \end{pmatrix}
    \right]
    \begin{pmatrix}
      \Delta c_1 \\
      \vdots \\
      \Delta c_{ns}
    \end{pmatrix}
    =
    \begin{pmatrix}
      \res{\rho_1} - c_1 \resrho \\
      \vdots \\
      \res{\rho_{N_s}} - c_{N_s} \resrho
    \end{pmatrix}
  \end{gathered}
  \label{approx-jac-dc}
\end{equation}
%------------------------------------------------------------------------------%
It should be noted that the approximations made in \erefs{dfdvl}{dfdvr} result
in no interdependence between species; so, the only off-diagonal terms for the
Jacobian matrix in \eref{approx-jac-dc} come from the the chemical source term
linearizations.

\section{Predicted Cost and Memory Savings of the Decoupled Implicit Problem}
\label{sec:predicted-cost-mem-savings}

In decoupling the species equations, the most significant savings comes from the
source term linearization being purely node-based\cite{gnoffo-tp}.  Solving the
mixture equations in is conducted in the same manner as the fully coupled
system.  The global Jacobian for the mixture equations, $A_m$ consists of block
$5 \times 5$ Jacobian matricies of the form shown in \eref{approx-jac}. In the
decoupled species continuity equations, all entries in the global Jacobian,
$A_d$, are $N_s \times N_s$ block matrices of the form in \eref{approx-jac-dc}.
Because there is no interdependence of species, except through the chemical
source term, all contributions due to linearizing the convective flux are purely
diagonal.  Via \eref{decomp-jac}, we decompose $A_d$ into its diagonal and
off-diagonal elements, resulting in the following linear system:
%------------------------------------------------------------------------------%
\begin{equation}
  \label{dc_sys} 
  \begin{pmatrix} 
    \Box & & & & \\ & \ddots & & & \\ & & \Box \\ & & & \ddots & \\ & & & & \Box
  \end{pmatrix}
  \begin{pmatrix}
    \delta \mathbf{\hat{V}}_1 \\ \vdots \\ \delta \mathbf{\hat{V}}_i \\ 
    \vdots \\ \delta \mathbf{\hat{V}}_{nodes}
  \end{pmatrix}
  =
  \begin{pmatrix}
    \hat{b}_1 \\ \vdots \\ \hat{b}_i \\ \vdots \\ \hat{b}_{nodes} 
  \end{pmatrix}
  -
  \begin{pmatrix}
    (\sum_{j=1}^{N_{nb}}{[\diagdown] \delta\mathbf{\hat{V}}_{j}})_1 \\ \vdots \\
    (\sum_{j=1}^{N_{nb}}{[\diagdown] \delta\mathbf{\hat{V}}_{j}})_i \\ \vdots \\
    (\sum_{j=1}^{N_{nb}}{[\diagdown] \delta\mathbf{\hat{V}}_{j}})_{nodes}
  \end{pmatrix} 
\end{equation} 
%------------------------------------------------------------------------------%
where $\Box$ represents a dense $N_s \times N_s$ matrix, $[\diagdown]$ represents
a diagonal matrix, and $\delta \mathbf{\hat{V}}_j$  is the decoupled variable
update on the node $j$ that neighbors node $i$, where $N_{nb}$ is the number of
nodes neighboring node $i$.  Thus, the non-zero entries in the off-diagonal
matrix can be reduced from diagonal matrices to vectors.  This results in
significant savings in both computational cost and memory, as the only expensive
operation left in solving the implicit system is dealing with the diagonal
entries in the Jacobian.  An LU decomposition of an $N \times N$ matrix
requires $\sim N^3/3$ operations, whereas the the matrix vector
products of equivalent dimension requires $N^2$ operations and the vector inner
products cost.  Including the LU decomposition operations, the relative cost of
the linear solves in the decoupled scheme compared to the fully coupled scheme
is approximately
%------------------------------------------------------------------------------%
\begin{equation}
  \text{Relative Computational Cost} = 
  \frac{
    \frac{\left( N_s + 4 \right)^3}{3} N_{nodes} + \left( N_s + 4 \right)^2(S_{GS})(N_{nz})
  }{
    \frac{(N_s)^3 + (5)^3}{3} N_{nodes} + \left( N_s + 5^2\right)(S_{GS})(N_{nz})
  }
  \label{relative-lu-gs-cost}
\end{equation}
%------------------------------------------------------------------------------%
where $S_{GS}$ is the number of multi-color Gauss-Seidel sweeps, $N_{nodes}$ is
the number of nodes, and $N_{nz}$ is the number of non-zero off-diagonal entries
stored using compressed row storage\cite{George}.  Because $(S_{GS)(N_nz} >>
N_{nodes}$, the LU decomposition cost is negligible except when the number of
species is extremely large.  $N_{nb}(N_s)$ operations.  If the LU decomposition
costs are dropped from \eref{relative-lu-gs-cost}, the relative cost of the
linear solve becomes
%------------------------------------------------------------------------------%
\begin{equation}
  \text{Relative Computational Cost} = 
  \frac{
   \left( N_s + 4 \right)^2
  }{
    \left( N_s + 5^2\right)
  }
  \label{relative-no-lu-gs-cost}
\end{equation}
%------------------------------------------------------------------------------%
and can therefore expect nearly linear speedup in the linear solver cost with the
number of species when using the decoupled system over the fully coupled system.
It should be noted that cost predicted by operation count of an LU decomposition
does not usually manifest equivalently in computational cost.  Significant
compiler optimizations are possible if a storage scheme is adopted where memory
collocation of the diagonal block jacobians is maintained.  These optimizations
decrease the cost of the LU decomposition relative to the Gauss-Seidel sweeps
further, and thus the LU decomposition has never been seen as a dominant cost
for hypersonic simulations.

Using compressed row storage, the relative memory savings in the limit of a
large number of species for the Jacobian is given by
%------------------------------------------------------------------------------%
\begin{equation}
  \label{mem_req_eq}
  \begin{split} 
    Relative\ Memory\ Cost &=
    \frac{size(A_d)}{size(A)} \\ &= \lim_{N_s\to\infty}
    \frac{(N_s^2+5^2)(N_{nodes})+(N_s+5^2)(N_{nz})}{(N_s+4)^2(N_{nodes}+N_{nz})} \\
    &= \frac{N_{nodes}}{N_{nodes} + N_{nz}}
  \end{split}
\end{equation}
%------------------------------------------------------------------------------%
For a three-dimensional structured grid, each node has six neighbors,
i.e., $N_{nz} = 6N_{nodes}$; therefore, we can expect the Jacobian memory
required to decrease by a factor of seven using this decoupled scheme.
Interestingly, for a grid that is not purely hexahedra, $N_{nz} > 6N_{nodes}$;
thus, this decoupled scheme provides higher relative memory savings on
unstructured grids than structured grids when using compressed row storage.

\section{Higher Order Reconstruction}
\label{sec:2nd-order-reconstruction}

To achieve higher-order accuracy, the FUN3D solver uses an extension of the
unstructured MUSCL (U-MUSCL) reconstruction scheme developed by Burg et
al\cite{burg2005higher,burg2003verification}, which is itself an extension of
the Monotonic Upstream-centered Scheme for Conservation Laws (MUSCL) scheme
developed by Van Leer\cite{van1979towards}.  The implementation is a combination
of central differencing and the original U-MUSCL scheme
%------------------------------------------------------------------------------%
\begin{equation}
  \begin{aligned}
    q_L &= q_1 + (1-\kappa)\left[ \phi\left( \pd{q_1}{x}dx + \pd{q_1}{y}dy +
    \pd{q_1}{z}dz \right) \right] + \frac{\kappa}{2}\left( q_2 - q_1 \right) \\
    q_R &= q_2 + (1-\kappa)\left[ \phi\left( \pd{q_2}{x}dx + \pd{q_2}{y}dy +
    \pd{q_2}{z}dz \right) \right] + \frac{\kappa}{2}\left( q_1 - q_2 \right)
  \end{aligned}
  \label{u-muscl}
\end{equation}
%------------------------------------------------------------------------------%
where $\kappa = 0.5$ is usually used on unstructured grids with mixed elements.
\fref{fig:edge-recons} shows that $q_{L,R}$ are the primitive variables at the
left and right sides of the edge midpoint, where the flux is evaluated,
$q_{1,2}$ are the primitive variables at nodes 1 and 2.
%------------------------------------------------------------------------------%
\begin{figure}[h]
  \centering
  \includegraphics[width=0.5\textwidth]{figures/edge_reconstruction.png}
  \caption{Edge reconstruction.}
  \label{fig:edge-recons}
\end{figure}
%------------------------------------------------------------------------------%
The variable $\phi$ is the result of the scalar flux limiter function, that is
required to preserve monotonicity in the second order reconstruction near
discontinuities.  FUN3D supports a larger variety of flux limiters that fall
into two main categories: edge-based limiters, and stencil-based limiters.  The
edge based limiters are evaluated for two nodes at each edge.  Different values
of $\phi$ can exist at each node for edge-based limiter, and the results are not
``freezable'' at each node in FUN3D. Stencil-based limiters are evaluated at
each node, based on the gradients that are computed there.  For these limiters,
the value of $\phi$ is unique to each node and can be stored with the flow
solution as an additional variable.  The frozen state of the limiter is only
re-evaluated if the reconstruction forces a non-physical state at the dual
volume interface.

For this study, smooth Van Albada\cite{van1997comparative}, Van
Leer\cite{vatsa2009calibration}, and Minmod\cite{roe1986characteristic} flux
limiter functions are used.  Each of these limiters is augmented with a
heuristic pressure limiter by Park\cite{park2008anisotropic}. The choice of
these limiters impacts solution convergence and accuracy.  It is important to
note that the smooth Van Albada averaging function is given as
%------------------------------------------------------------------------------%
\begin{equation}
  \phi\left( a, b \right) =
  \frac{(b^2 + \varepsilon^2)a + (a^2 + \varepsilon^2)b}
  {a^2 + b^2 + 2\varepsilon^2}
  \label{van-albada-avg}
\end{equation}
%------------------------------------------------------------------------------%
where $a$ and $b$ are the left and right node gradients, and $\varepsilon$ is
the ``smoothing coefficient''.  This smoothing coefficient is used to tune the
flux limiter to a variety of the problems, and it is advised to be the
reciprocal of the mean aerodynamic chord (MAC) in grid units for
FUN3D\cite{biedron2016fun3d}.  The choice of $\varepsilon$ is critical to some
hypersonic applications involving chemistry, and will be discussed later.



%%---------------------------------------------------------------------------%%
%%  Bibliography 

\bibliography{KyleThompson-thesis}{}
\bibliographystyle{plain}

%%---------------------------------------------------------------------------%%
% Appendices
\appendix

\chapter{Derivations}

\section{Decoupled Flux Derivation}

For the Roe flux difference splitting scheme, the species mass fluxes are given by
%
\begin{equation}
	F_{\rho_s} = \frac{\rho_s^L\overline{U}^L+\rho_s^R\overline{U}^R}{2}
	-\frac{\tilde{c}_s(\lambda_1 dv_1 + \lambda_2 dv_2)+\lambda_3 dv_{3_s}}{2} \label{species_mass} \\
\end{equation}
\begin{align}	
		dv_1 &= \frac{p^R-p^L+\tilde{\rho} \tilde{a} (\overline{U}^R-\overline{U}^L)}{\tilde{a}^2} \\
		dv_2 &= \frac{p^R-p^L-\tilde{\rho} \tilde{a} (\overline{U}^R-\overline{U}^L)}{\tilde{a}^2} \\
		dv_{3_s} &= \frac{\tilde{a}^2 (\rho_s^R-\rho_s^L)- \tilde{c}_s (p^R-p^L)}{\tilde{a}^2}
\end{align}
\begin{align}
	\lambda_1 = \mid\mathbf{\overline{U}}+\tilde{a} \mid,\quad 
	\lambda_2 = \mid \mathbf{\overline{U}}-\tilde{a} \mid,\quad 
	\lambda_3 =  \mid \mathbf{\overline{U}} \mid
\end{align}
%
where the $\tilde{}$ notation signifies a Roe-averaged quantity, given by:
%
\begin{gather}
	\mathbf{\tilde{U}} =w\mathbf{\tilde{U}}^L+(1-w)\mathbf{\tilde{U}}^R \\
	w = \frac{\tilde{\rho}}{\tilde{\rho}+\rho^R} \\
	\tilde{\rho} = \sqrt{\rho^R\rho^L}
\end{gather}
%
The species mass fluxes must sum to the total mass flux; thus, the total mixture mass flux is given as
%
\begin{equation}
\label{total_mass}
	F_\rho = \sum\limits_{s}{F_{\rho_s}} = \frac{\rho^L\overline{U}^L+\rho^R\overline{U}^R}{2}
	-\frac{\tilde{c}_s(\lambda_1 dv_1 + \lambda_2 dv_2)+\lambda_3 dv_3}{2}
\end{equation}
\begin{equation}
	dv_3 = \frac{\tilde{a}^2 (\rho^R-\rho^L)-(p^R-p^L)}{\tilde{a}^2}
\end{equation}
%
Multiplying Eq.~(\ref{total_mass}) by the Roe-averaged mass fraction and
substituting it into Eq.~(\ref{species_mass}) results in:
%
\begin{equation}
\label{unsimp_sp_flux}
	F_{\rho_s} =\tilde{c}_s F_\rho + \frac{(c_s^L-\tilde{c}_s)\rho^L(\overline{U}^L+\mid \tilde{U}\mid)}{2}
	+ \frac{(c_s^R-\tilde{c}_s)\rho^R(\overline{U}^R-\mid \tilde{U}\mid)}{2}
\end{equation}
%
It should be noted here that the Roe-averaged normal velocity, $\tilde{U}$,
requires an entropy correction in the presence of strong shocks\cite{harten}.
This correction has no dependence on the species mass fractions; therefore,
it does not change the form of the Jacobian for this decoupled scheme. The
notation can be further simplified by defining the normal velocities as follows:
%
\begin{equation} \label{lambda_pm} \lambda^+ = \frac{\overline{U}^L+\mid
  \tilde{U}\mid}{2}, \quad \lambda^- = \frac{\overline{U}^R-\mid
  \tilde{U}\mid}{2} \end{equation}
%
Finally, substituting Eq.~(\ref{lambda_pm}) into Eq.~(\ref{unsimp_sp_flux})
yields the final result for calculating the species flux in the decoupled
system:
%
\begin{equation} \label{sp_flux} F_{\rho_s} =\tilde{c}_s F_\rho +
  (c_s^L-\tilde{c}_s)\rho^L\lambda^+ + (c_s^R-\tilde{c}_s)\rho^R\lambda^-
\end{equation}
%
Forming the convective contributions to the Jacobians is straightforward.
Because the $\mathbf{U}'$ level variables are constant, only the left, right,
and Roe-averaged state mass fractions vary.  Differentiating Eq.~(\ref{sp_flux})
with respect to the mass fraction, $c_s$, the left and right state contributions
are
%
\begin{gather} \frac{\partial F_{\rho_s}}{\partial c^L_s} =
  wF_\rho+(1-w)\rho^L\lambda^+ - w\rho^R\lambda^- \\ \frac{\partial
    F_{\rho_s}}{\partial c^R_s} = (1-w)F_\rho+(w-1)\rho^L\lambda^+ +
    w\rho^R\lambda^- \end{gather}
%
Because there is no dependence between species in decoupled convective
formulation, the Jacobian block elements are purely diagonal for the convective
contributions, of the form
%
\begin{equation} \begin{pmatrix} \frac{\partial F_{\rho_1}}{\partial c_1} & & 0
    \\ & \ddots &  \\ 0 & & \frac{\partial F_{\rho_{ns}}}{\partial c_{ns}}
  \end{pmatrix} \end{equation}


%%---------------------------------------------------------------------------%%
\backmatter

\end{document}
