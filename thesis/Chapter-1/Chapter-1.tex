\chapter{Introduction}
\label{chapter-one}

In the last decades, computational fluid dynamics (CFD) codes have matured to
the point that it is possible to obtain high-fidelity design sensitivities,
which can be coupled with optimization packages to enable design optimization
for a variety of inputs\cite{baysal1992aerodynamic, balagangadhar2001design}.
Typically, CFD codes are coupled with a numerical optimization
package\cite{SNOPT-alg, KSOPT, fletcher1963rapidly, npsol-manual}, where the CFD
analysis evaluates an aerodynamic ``cost function'' of interest, based on a set
of design variables, and passes this cost function to the optimizer.  The
optimizer then returns a new set of inputs to the CFD code to improve the
previous design.  These optimization procedures also require the sensitivity
derivatives of the cost function with respect to the design variables.  

The most straightforward approach to computing design sensitivities is a
finite-difference approach.  The finite-difference approach effectively turns a
CFD code into ``black box'', where each design variables is perturbed and used
as an input to a CFD code.  This requires very little effort to implement, and
can be attractive if the number of cost functions is large in comparison to the
number of design variables, since the sensitivity of all cost functions to a
single design variable can be computed from a single finite-difference
evaluation.  The finite-difference approach, however, requires at least one
analysis from the CFD code per design variable, making it ill-suited for design
optimization problems with a large number of design inputs.

In recent years, an adjoint-based approach to computing sensitivities has been
implemented in many compressible CFD codes\cite{mavriplis-2006,
nemec-aftosmis-adjoint, nielsen2002recent, reuther1999constrained}. This
adjoint-based approach requires a single flow and ``adjoint'' solution to
compute the sensitivity of all design variables to a given cost function.
In essence, the adjoint-based approach requires an additional solution per cost
function, whereas the finite-difference approach requires an additional solution
per design variable.  Because the number of design variables is, usually, much
larger than the number of cost functions in CFD applications, the adjoint-based
approach is much more efficient than the finite-difference approach for
computing design sensitivities.


Reacting gas CFD codes have lagged in adopting this adjoint-based approach, but
interest has increased in recent years\cite{Copeland, Barcelona,
lockwood2010parameter}.  The relatively small number of reacting gas CFD codes
that employ adjoint-based sensitivity analysis is likely due to the significant
jump in complexity of the linearizations required, especially in regard to the
chemical source term and the dissipation term in the Roe flux difference
splitting (FDS) scheme\cite{roe}.  There is also a split in the reacting gas
solver community over the methodology used to compute adjoint-based sensitivity
derivatives, specifically whether to implement a continuous or discrete version
of the adjoint equations.  In a continuous formulation of the adjoint equations,
the flow solver equations are linearized and then discretized, whereas in
discrete formulation the flow solver equations are discretized and then
linearized.  Copeland et al.\cite{Copeland} have adopted a continuous
formulation of the adjoint equations for flows in chemical and thermal
non-equilibrium, and have implemented this formulation in the CFD code
SU2\cite{palacios2013stanford} from Stanford University.  The reasoning cited
for choosing a continuous formulation of the adjoint equations, was that, among
other things, the adjoint system was not well suited to being solved by the same
methods used for solving the flow solver equations and the resulting adjoint
solution could lead to non-physical oscillations that fail to capture the
continuous adjoint solution.  Lockwood et al.\cite{lockwood2010uncertainty,
lockwood2010parameter}, on the other hand, have published favorable results for
a discrete adjoint formulation for both parameter sensitivity and uncertainty
quantification.

For the work presented here, a discrete adjoint formulation is chosen over a
continuous one.  The reason for this decision is that the merits a of continuous
adjoint formulation do not outweigh the testing and practical benefits of a
discrete adjoint formulation.  Because the discretized flow solver equations are
linearized in a discrete adjoint solver, the adjoint solution can be verified
using finite-difference derivatives.  This is not possible with a continuous
formulation, as the solution to the continuous problem will not exactly match
the discretization; therefore, from a practical standpoint, the verification of
a continuous adjoint solution is much more ambiguous than for a discrete adjoint
solution.  The claim that non-equilibrium flow solvers are not guaranteed to
solve the adjoint system is disputed here, as Nielsen et
al.\cite{nielsen2004implicit} proved that a discrete adjoint scheme can be made
dual exact, where the discrete adjoint approach matches the direct
differentiation approach exactly for each time step.

An additional obstacle that may prevent adjoint-based sensitivity analysis in
reacting gas solvers is the extreme problem size associated with high energy
physics.  The additional equations required in reacting gas simulations lead to
large Jacobians that scale quadratically in size to the number of governing
equations.  This scaling leads to a significant increase in the memory required
to store the flux linearizations and the computational cost of the point solver.
As reacting gas CFD solvers are used to solve increasingly more complex
problems, this onerous quadratic scaling of computational cost and Jacobian size
will ultimately surpass the current limits of hardware and time constraints on
achieving a flow solution\cite{fischer}.

To mitigate this scaling issue, Candler et al.\cite{candler} proposed a scheme
to for a modified form of the Steger-Warming flux vector splitting
scheme\cite{MacCormack,Steger}. In that work, it was shown that quadratic
scaling between the cost of solving the implicit system and additional species
mass equations can be reduced from quadratic to linear scaling by decoupling the
species mass equations from the mixture mass, momentum, and energy equations and
solving the two systems sequentially.  This work extends the aforementioned work
from the modified form of the Steger-Warming flux vector splitting method to the
Roe flux difference splitting (FDS) scheme, and, more importantly, extends it
for the first time to the solution of the adjoint formulation.  Candler et
al.\cite{candler2013analysis} later analyzed stability issues associated with
the decoupled formulation of the flow solver in \cite{candler}.  A new chemical
source term scaling procedure is introduced in the present work that mitigates
these stability concerns.

\section{Contributions}

\begin{enumerate}

  \item {\bf Improved stability for solving inviscid flows in chemical
    non-equilibrium using a decoupled, point-implicit approach.}

To mitigate the quadratic scaling in computational cost and memory required with
respect to each additional species continuity equation, Candler et
al.\cite{candler} proposed a scheme to for a modified form of the Steger-Warming
flux vector splitting scheme\cite{MacCormack,Steger}. In that work, it was shown
that the quadratic scaling could be reduced from quadratic to linear scaling by
decoupling the species mass equations from the mixture mass, momentum, and
energy equations and solving the two systems sequentially. Candler et
al.\cite{candler2013analysis} later analyzed stability issues associated with
the decoupled formulation of the flow solver in \cite{candler}. This work
extends the aforementioned work from the modified form of the Steger-Warming
flux vector splitting method to the Roe flux difference splitting (FDS) scheme,
and introduces a new chemical source term scaling procedure to mitigate these
stability concerns.  To the author's knowledge, this decoupled flow solver
procedure, involving chemical source term scaling, is a novel extension to the
work by Candler et al.\cite{candler}, and enables the decoupled approach to be
used on more challenging problems involving very large chemical reaction rates.

  \item {\bf Improved computational cost and memory required scaling with number of
    species continuity equations in adjoint solver.}

A discrete adjoint formulation is derived and implemented in the reacting gas
path of the FUN3D CFD code\cite{biedron2016fun3d} for the first time, and both a
fully coupled point-implicit and decoupled point-implicit solution procedure are
described to solve the adjoint equations.  This decoupled point-implicit
approach to solving the adjoint system of equation is an analog to the method
used by the flow solver decoupled approach, and, to the best of the author's
knowledge, represents the first of its kind in published literature.  This novel
approach to solving the adjoint system of equations significantly improves the
computational cost and memory required by the adjoint solver in the same manner
as the decoupled flow solver approach, and is shown to speedup the time to
solution by more than an order of magnitude in some cases.

  \item {\bf Design optimization off a hypersonic reentry vehicle with a
    retro-firing annular jet.}

An axi-symmetric hypersonic re-entry vehicle with an annular jet is chosen as
the main demonstration problem in this work.  This design was previously
examined by Gnoffo et al.\cite{gnoffo2016tapping} for a perfect gas as a means
of increasing the time-averaged vehicle drag via unsteady interactions between
the annular jet plume and the bow shock.  A steady solution was found for that
geometry and it was postulated that the annular jet plume could be used as an
active cooling mechanism to augment or replace the passive thermal protection
system (TPS) on the vehicle.  This work extends the previous work to inviscid
flow in chemical non-equilibrium, and applies sensitivities computed by the
adjoint solver to minimize the vehicle outer surface temperature and mass flow
rate through the annular jet plenum by changing the conditions at the plenum.

\end{enumerate}
