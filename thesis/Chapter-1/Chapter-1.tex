\chapter{Introduction}
\label{chapter-one}

The usability of hypersonic solvers on complex geometries is often limited by
the extreme problem size associated with high energy physics.  The additional
equations required in reacting gas simulations lead to large Jacobians that
scale quadratically in size to the number of governing equations.  This leads
to a significant increase in the memory required to store the flux
linearizations and the computational cost of the point solver.  As reacting gas
CFD solvers are used to solve increasingly more complex problems, this onerous
quadratic scaling of computational cost and Jacobian size will ultimately
surpass the current limits of hardware and time constraints on achieving a flow
solution\cite{fischer}.

The proposed method is based heavily upon the work of Candler et
al.\cite{candler}. In that work, it was shown that quadratic scaling between the
cost of solving the implicit system and adding species mass equations can be
reduced from quadratic to linear scaling by decoupling the species mass
equations from the mixture mass, momentum, and energy equations and solving the
two systems sequentially.  In the aforementioned work, the scheme was derived
for a modified form of the Steger-Warming flux vector splitting
method\cite{MacCormack}, whereas the work presented here is derived for the Roe
flux difference splitting (FDS) scheme\cite{roe}.

A primary motivator for the presented work is to prepare for the implementation
of an adjoint capability in a reacting gas solver that employs the Roe scheme.
Developing an adjoint implementation for non-equilibrium flows is an extremely
challenging problem, because deriving exact Jacobians for a reacting gas system
is particularly difficult.  Decoupling the system also simplifies the derivation
of exact Jacobians.  If the species mass equations are decoupled, the mixture mass,
momentum, and energy flux Jacobians can be easily derived\cite{Nishikawa}, and a
weighting scheme can be used to correct the non-uniqueness of the pressure
linearization\cite{Shuen}.  For the decoupled species fluxes, we derive an exact
linearization in the presented work.  Future work will include an adjoint-based
error estimation capability that leverages these exact linearizations.

