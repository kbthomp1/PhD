\chapter{Introduction}
\label{chapter-one}

In the last decades, computational fluid dynamics (CFD) codes have matured to
the point that it is possible to obtain high-fidelity design sensitivities,
which can be coupled with optimization packages to enable design optimization
for a variety of inputs\cite{baysal1992aerodynamic, balagangadhar2001design}.
In more recent years, adjoint-based formulations to compute sensitivities have
been implemented in many compressible CFD codes\cite{mavriplis-2006,
nemec-aftosmis-adjoint, nielsen2002recent}. This approach is much attractive
over a finite-difference method, because a single flow and adjoint solution is
required to compute sensitivities all design variables and a given cost
function, whereas finite difference requires an additional flow solution for
each design variable.  Reacting gas CFD codes have lagged in adopting this
adjoint-based approach, but interest has increased in recent
years\cite{Copeland, Barcelona, lockwood2010parameter}.  The relatively small
number of reacting gas CFD codes that employ adjoint-based sensitivity analysis
is likely due to the significant jump in complexity of the linearizations
required, especially in regard to the chemical source term and the dissipation
term in the Roe flux difference splitting (FDS) scheme\cite{roe}.

In the reacting gas solver community, there is a split in the methodology used
to compute adjoint-based sensitivity derivatives, specifically whether to
implement a continuous or discrete version of the adjoint equations.  In a
continuous formulation of the adjoint equations, the flow solver equations are
linearized and then discretized, whereas in discrete formulation the flow solver
equations are discretized and then linearized.  Copeland et al.\cite{Copeland}
have adopted a continuous formulation of the adjoint equations for flows in
chemical and thermal non-equilibrium, and have implemented this formulation in
the CFD code SU2\cite{palacios2013stanford} from Stanford University.  The
reasoning cited for choosing a continuous formulation of the adjoint equations,
was that, among other things, the adjoint system was not wel suited to being
solved by the same methods used for solving the flow solver equations and the
resulting adjoint solution could lead to non-physical oscillations that fail to
capture the continuous adjoint solution.  Lockwood et
al.\cite{lockwood2010uncertainty, lockwood2010parameter}, on the other hand,
have published favorable results for a discrete adjoint formulation for both
parameter sensitivity and uncertainty quantification.

For the work presented here, a discrete adjoint formulation is chosen over a
continuous one.  The reason for this decision is that the merits a of continuous
adjoint formulation do not outweigh the testing and practical benefits of a
discrete adjoint formulation.  Because the discretized flow solver equations are
linearized in a discrete adjoint solver, the adjoint solution can be verified
using finite-difference derivatives.  This is not possible with a continuous
formulation, as the solution to the continuous problem will not exactly match
the discretization; therefore, from a practical standpoint, the verification of
a continuous adjoint solution is much more ambiguous than for a discrete adjoint
solution.  The claim that non-equilibrium flow solvers are not guaranteed to
solve the adjoint system is disputed here, as Nielsen et
al.\cite{nielsen2004implicit} proved that a discrete adjoint scheme can be made
dual exact, where the discrete adjoint approach matches the direct
differentiation approach exactly for each time step.

An additional obstacle that may prevent adjoint-based sensitivity analysis in
reacting gas solvers is the extreme problem size associated with high energy
physics.  The additional equations required in reacting gas simulations lead to
large Jacobians that scale quadratically in size to the number of governing
equations.  This scaling leads to a significant increase in the memory required
to store the flux linearizations and the computational cost of the point solver.
As reacting gas CFD solvers are used to solve increasingly more complex
problems, this onerous quadratic scaling of computational cost and Jacobian size
will ultimately surpass the current limits of hardware and time constraints on
achieving a flow solution\cite{fischer}.

To mitigate this scaling issue, Candler et al.\cite{candler} proposed a scheme
to for a modified form of the Steger-Warming flux vector splitting
scheme\cite{MacCormack,Steger}. In that work, it was shown that quadratic
scaling between the cost of solving the implicit system and additional species
mass equations can be reduced from quadratic to linear scaling by decoupling the
species mass equations from the mixture mass, momentum, and energy equations and
solving the two systems sequentially.  This work extends the aforementioned work
from the modified form of the Steger-Warming flux vector splitting method to the
Roe flux difference splitting (FDS) scheme, and, more importantly, extends it
for the first time to the solution of the adjoint formulation.  Candler et
al.\cite{candler2013analysis} later analyzed stability issues associated with
the decoupled formulation of the flow solver in \cite{candler}.  A new chemical
source term scaling procedure is introduced in the present work that mitigates
these stability concerns.

As its primary focus, this work demonstrates that a variation of the decoupled
point-implicit scheme can be applied to both the flow solver and adjoint solver
in the reacting gas path of the FUN3D CFD code, and significantly improve the
efficiency in both computational cost and memory required.  Additionally,
exactly linearizing the Roe FDS scheme is shown to significantly improve
performance and robustness in the flow solver over previous approximate
linearizations used\cite{gnoffo-tp}.

An axi-symmetric hypersonic re-entry vehicle with an annular jet is chosen as
the main demonstration problem in this work.  This design was previously
examined by Gnoffo et al.\cite{gnoffo2016tapping} for a perfect gas as a means
of increasing the time-averaged vehicle drag via unsteady interactions between
the annular jet plume and the bow shock.  A steady solution was found for that
geometry and it was postulated that the annular jet plume could be used as an
active cooling mechanism to augment or replace the passive thermal protection
system (TPS) on the vehicle.  This work extends the previous work to inviscid
flow in chemical non-equilibrium, and applies sensitivities computed by the
adjoint solver to minimize the vehicle outer surface temperature and mass flow
rate through the annular jet plenum by changing the conditions at the plenum.

