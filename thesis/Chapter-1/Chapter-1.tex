\chapter{Introduction}
\label{chapter-one}

In the last decades computational fluid dynamics (CFD) codes have matured to the
point that it is possible to obtain high-fidelity design sensitivities that can
be coupled with optimization packages to enable design optimization for a
variety of design inputs\cite{baysal1992aerodynamic, balagangadhar2001design}.
In recent years, the benefits of using an adjoint-based formulation to compute
sensitivities have been realized and implemented in many compressible CFD
codes\cite{mavriplis-2006, nemec-aftosmis-adjoint, nielsen2002recent}, because
of the ability compute all sensitivities a the cost of a single extra adjoint
solution, instead of an additional flow solution for each design variable.
Reacting gas CFD codes have lagged significantly in adopting this adjoint-based
approach, with only a small number of codes having published
results\cite{Copeland, Barcelona}.  This is likely due to the significant
jump in complexity of the linearizations required, especially in regard to the
chemical source term and the dissipation term in the Roe flux difference
splitting (FDS) scheme\cite{roe}.

An additional obstacle that may prevent adjoint-based sensitivity analysis in
reacting gas solvers is the extreme problem size associated with high energy
physics.  The additional equations required in reacting gas simulations lead to
large Jacobians that scale quadratically in size to the number of governing
equations.  This leads to a significant increase in the memory required to store
the flux linearizations and the computational cost of the point solver.  As
reacting gas CFD solvers are used to solve increasingly more complex problems,
this onerous quadratic scaling of computational cost and Jacobian size will
ultimately surpass the current limits of hardware and time constraints on
achieving a flow solution\cite{fischer}.

To mitigate this scaling issue, Candler et al.\cite{candler} proposed a scheme
to for a modified form of the Steger-Warming flux vector splitting
scheme\cite{MacCormack,Steger}. In that work, it was shown that quadratic
scaling between the cost of solving the implicit system and adding species mass
equations can be reduced from quadratic to linear scaling by decoupling the
species mass equations from the mixture mass, momentum, and energy equations and
solving the two systems sequentially.  This work extends the aforementioned work
from the modified form of the Steger-Warming flux vector splitting
method to the Roe flux difference splitting (FDS) scheme.

The work presented demonstrates that this decoupling can be applied to both the
flow solver and adjoint solver in a reacting gas CFD code, and significantly
improve the efficiency in both computational cost and memory required.
Additionally the implementation of exact linearizations is shown to
significantly improve performance and robustness in the flow solver over common
approximations used when linearizing the Roe FDS scheme.  The formulation and
linearization of the fluxes for the Roe FDS scheme are presented here, and
design optimization for an inviscid reacting flow is conducted for inviscid,
reacting flow around an axi-symmetric hypersonic re-entry vehicle with an
annular jet.

