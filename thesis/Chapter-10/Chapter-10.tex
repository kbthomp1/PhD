\chapter{Concluding Remarks}
\label{chapter-ten}

An implicit, decoupled scheme has been implemented in the reacting gas path of
the FUN3D flow solver, and an adjoint solver employing a fully coupled and
decoupled scheme has also been implemented.  The sensitivity dervatives provided
by the adjoint solver have been verified using a complex-variable approach, and
a sample inverse optimization and direct optimization was completed on a
hypersonic re-entry vehicle geometry that employed a retro-firing annular jet.

The consistency and robustness of decoupling the variable sets in the flow
solver was examined.  It was found that the decoupled scheme only exhibited
stability issues when the chemical source term was very large, due to large
reaction rates.  This stability issue was mitigated by multiplying the chemical
source term uniformly by scaling term, which could be ramped from zero to one
over the course of the simulation.  Provided that the ramping was completed, the
true steady-state solution, done without ramping, was recovered.  The
consistency of the solutions computed by the fully coupled and decoupled methods
was found to match discretely for surface temperature and pressure, and to
within $10^{-4}$ for mass fractions on the stagnation line of a spanwise
cylinder case in a $5 km/s$ flow.  It was also determined that this difference
in mass fraction was reduced with mesh spacing.

The decoupled adjoint scheme demonstrated here is a significant improvement to
the state of the art in the field of sensitivity analysis for reacting flows.
The decreased memory and computational costs afforded by solving a decoupled
dependent variable set in the adjoint solver enable a speedup greater than an
order of magnitude for some cases.  
