\chapter{Concluding Remarks}
\label{chapter-ten}

An implicit, decoupled scheme has been implemented in the reacting gas path of
the FUN3D flow solver, and an adjoint solver employing a fully coupled and
decoupled scheme has also been implemented.  The decoupled adjoint scheme
demonstrated here is a significant improvement to the state of the art in the
field of sensitivity analysis for reacting flows.  The decreased memory and
computational costs afforded by solving a decoupled dependent variable set in
the adjoint solver enabled a speedup greater than an order of magnitude for some
cases.  

The consistency and robustness of decoupling the variable sets in the flow
solver was examined.  It was found that the decoupled scheme only exhibited
stability issues when the chemical source term was very large, due to large
reaction rates.  This stability issue was mitigated by multiplying the chemical
source term uniformly by scaling term, which could be ramped from zero to one
over the course of the simulation.  Provided that the ramping was completed, the
true steady-state solution, done without ramping, was recovered.  The
consistency of the solutions computed by the fully coupled and decoupled methods
was found to match discretely to eight digits for surface temperature and pressure,
and four digits for mass fractions on the stagnation line of a spanwise
cylinder case in a 5 $km/s$ flow.  It was also determined that this difference
in mass fraction was reduced with mesh spacing.

Sample inverse and direct design optimizations were completed on a hypersonic
re-entry vehicle geometry that employed a retro-firing annular jet.  Using the
sensitivity derivatives computed by the adjoint, the annular jet plenum
conditions were modified to improve the surface temperature RMS and mass flow
rate through the annular jet plenum.  The direct design case demonstrated the
ability to tune a design based on favorable engineering attributes, and serves
as a proof-of-concept for future work to develop re-entry vehicles with active
thermal protection systems.

To determine the accuracy of the sensitivity derivatives provided by the adjoint
solver, a complex-variable approach was employed.  By transforming the source
code of FUN3D, complex-variable Frechet derivatives were computed and compared
to those computed by the adjoint solver.  Both of these were found to match to a
high degree of accuracy when both the flow solver and adjoint solver were
sufficiently converged.  Cases involving strong chemical reactions were
typically found to produce the greater differences between the flow and adjoint
solvers; however, the accuracy of the sensitivity derivatives was still
sufficient complete direct and inverse design optimizations.

An analog to the decoupled iterative mechanism in the flow solver was derived
for the adjoint solver, and its consistency and efficiency compared to the fully
coupled scheme was examined.  The adjoint solution was found to discretely match
the fully coupled solution, and numerical tests showed that the decoupled
adjoint scheme was over the 3 times more efficient than the fully coupled
adjoint scheme for the annular jet demonstration problem, if the exact
linearizations were store rather than recomputed at each time step.  The
decoupled adjoint scheme also afforded $20\%$ less memory required than the
fully coupled adjoint scheme, and the decoupled scheme was over 16 times faster
than the fully coupled scheme when the decoupled scheme utilized storing the
exact linearizations and the fully coupled scheme did not.  Future work will
include methods to speedup the adjoint exact linearization computation, as well
as schemes that employ mixed storage and recomputing of these linearizations,
in order to further improve the efficiency of the decoupled adjoint scheme.

The ability to perform high-fidelity sensitivity analysis and design
optimization on reacting flows is significantly helped by the increased
computational efficiency that the decoupled flow and adjoint schemes provide.
The improvement to iterative convergence provided by exact linearizations in the
Jacobians, also compounds the beneficial aspects of the decoupled schemes.  It
is hoped that these improvements will encourage interest in the field of
adjoint-based sensitivity analysis, by enabling solutions to problems that were
intractable due to the high computational and memory cost requirements.
Future work will extend these schemes to viscous flows, incorporate
multi-temperature models, and apply adjoint-based sensitivities to mesh
adaptation and mesh movement strategies.
