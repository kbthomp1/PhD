\chapter{Numerical Solution of Adjoint Equations}
\label{chapter-four}

This section details the derivation of the adjoint equations to be solved in
conjuction with the primal flow equations.  The primary goal of this research is
to compute sensitivities of aerodynamic and aerothermodynamic quantities to
design variables.  To achieve this, the sensitivity to the primal flow equation
formulation must first be solved.  For a discrete adjoint formulation, this
requires a solution of costate variables, relating a change in the flow equation
residual to a change in the function of interest.  Because of the large number
of equations required in a reacting gas solver, the adjoint solver will suffer
from the quadratic scaling in computational cost and memory required similarly
to the primal flow solver.  To mitigate this, a decoupled scheme is derived that
is consistent with the decoupled flow solver.

\section{Discrete Adjoint Derivation}

%The FUN3D adjoint derivation is given in Eric Nielson's PhD Thesis[cite??].
The derivation for the discrete adjoint begins with forming the Lagrangian as
%------------------------------------------------------------------------------%
\begin{equation}
  L(\md,\mq,\mx,\mathbf{\Lambda})=f(\md,\mq,\mx)
  +\mathbf{\Lambda}^T\mr(\md,\mq,\mx)
\end{equation}
%------------------------------------------------------------------------------%
Where $\mr$ is the residual of the flow equations.  Differentiating with respect
to the design variables $\md$ yields
%------------------------------------------------------------------------------%
\begin{equation}
  \frac{\partial L}{\partial \md}=
  \Bigg\{\frac{\partial f}{\partial \md}+\bigg[\frac{\partial \mx}{\partial \md}\bigg]^T \frac{\partial f}{\partial \mx}\Bigg\}
  +\bigg[\frac{\partial \mq}{\partial \md}\bigg]^T
  \Bigg\{\frac{\partial f}{\partial \mq}+\bigg[\frac{\partial \mr}{\partial \mq}\bigg]^T \mathbf{\Lambda}\Bigg\}
  +\Bigg\{\bigg[\frac{\partial \mr}{\partial \md}\bigg]^T
  +\bigg[\frac{\partial \mx}{\partial \md}\bigg]^T\bigg[\frac{\partial \mr}{\partial \mx}\bigg]^T\Bigg\}\mathbf{\Lambda}
  \label{dL}
\end{equation}
%------------------------------------------------------------------------------%
To eliminate the dependence of conserved variables $\mq$ on the design
variables, we solve the adjoint equation
%------------------------------------------------------------------------------%
\begin{equation}
  \bigg[\frac{\partial \mr}{\partial \mq}\bigg]^T\mathbf{\Lambda}=
  -\frac{\partial f}{\partial \mq}
  \label{adjoint-main}
\end{equation}
%------------------------------------------------------------------------------%
Where the Lagrange multipliers (also known as costate variables),
$\mathbf{\Lambda}$ are the cost function dependence on the residual
%------------------------------------------------------------------------------%
\begin{equation}
  \mathbf{\Lambda}=-\frac{\partial f}{\partial \mr}
\end{equation}
%------------------------------------------------------------------------------%
This can ultimately be used to error estimation and sensitivity analysis for
design optimization.  With the second term in eq.~(\ref{dL}) eliminated, the
derivative of the Lagrangian becomes
%------------------------------------------------------------------------------%
\begin{equation}
  \frac{\partial L}{\partial \md}=
  \Bigg\{\frac{\partial f}{\partial \md}+\bigg[\frac{\partial \mx}{\partial
  \md}\bigg]^T \frac{\partial f}{\partial \mx}\Bigg\}
  +\Bigg\{\bigg[\frac{\partial \mr}{\partial \md}\bigg]^T
  +\bigg[\frac{\partial \mx}{\partial \md}\bigg]^T\bigg[\frac{\partial
  \mr}{\partial \mx}\bigg]^T\Bigg\}\mathbf{\Lambda}
  \label{obj_function}
\end{equation}
%------------------------------------------------------------------------------%
By solving the adjoint equation in \eref{adjoint-main}) to obtain the costate
variable vector, $\mathbf{\Lambda}$, we can now use a non-linear optimizer to
determine the optimum set of design variables, $\md^*$. This can be done using
{\bf SNOPT\cite{snopt-manual}}, {\bf KSOPT\cite{KSOPT}}, or {\bf
  NPSOL\cite{npsol-manual}} in FUN3D, as well as a host of other non-linear
optimizers.

\section{Block Jacobi Adjoint Decoupling}

It is possible to decoupled the adjoint equations in a fashion similiar to that
done to the primal flow equations.  In this decoupled adjoint formulation, the
conserved variables are split identically to flow equations, with the
fully-coupled vector of conserved variables
%------------------------------------------------------------------------------%
\begin{equation}
	\vU =
  \begin{pmatrix}
 		\rho_1    \\
		\vdots    \\
		\rho_{ns} \\
    \rho \vu  \\
		\rho E    \\
	\end{pmatrix}
  \label{all-vars}
 \end{equation}
%------------------------------------------------------------------------------%
 split into
%------------------------------------------------------------------------------%
\begin{equation}
	\begin{matrix}
		\mathbf{U}'=\begin{pmatrix}
			\rho \\
			\rho \vu \\
			\rho E
		\end{pmatrix},\quad &
		\mathbf{\hat{U}}=\begin{pmatrix}
			\rho_1 \\
			\vdots \\
			\rho_{ns}
		\end{pmatrix}
	\end{matrix}
  \label{dc-vars}
\end{equation}
%------------------------------------------------------------------------------%
With this splitting, the mixture equations for single point in the global system
of the decoupled flow solve can be written as
%------------------------------------------------------------------------------%
\begin{equation}
  \frac{V}{\Delta t}\mi + 
  \begin{pmatrix}
    \rdiff{\rho}{\rho} & \rdiff{\rho}{\rho \vu} & \rdiff{\rho}{\rho E} \\ \\
    \rdiff{\rho \vu}{\rho} & \rdiff{\rho \vu}{\rho \vu} & \rdiff{\rho \vu}{\rho E} \\ \\
    \rdiff{\rho E}{\rho} & \rdiff{\rho E}{\rho \vu} & \rdiff{\rho E}{\rho E}
  \end{pmatrix}
  \begin{pmatrix}
    \Delta \rho \\ \\
    \Delta \rho \vu \\ \\
    \Delta \rho E
  \end{pmatrix}
  =
  \begin{pmatrix}
    \res{\rho} \\ \\
    \res{\rho \vu} \\ \\
    \res{\rho E}
  \end{pmatrix}
  \label{approx-jac}
\end{equation}
%------------------------------------------------------------------------------%
Likewise, the species mass equations for a single point can be written as
%------------------------------------------------------------------------------%
\begin{equation}
  \frac{V}{\Delta t}\mi + 
  \begin{pmatrix}
    \rdiff{\rho_1}{c_1} & \cdots & \rdiff{\rho_{1}}{c_{ns}} \\ \\
    \vdots & \ddots & \vdots \\ \\
    \rdiff{\rho_{ns}}{c_1} & \cdots & \rdiff{\rho_{ns}}{c_{ns}}
  \end{pmatrix}
  \begin{pmatrix}
    \Delta c_1 \\ \\
    \vdots \\ \\
    \Delta c_{ns}
  \end{pmatrix}
  =
  \begin{pmatrix}
    \res{\rho_1} \\ \\
    \vdots \\ \\
    \res{\rho_{ns}}
  \end{pmatrix}
  \label{approx-jac-dc}
\end{equation}
%------------------------------------------------------------------------------%
Examining \erefs{approx-jac}{approx-jac-dc} shows that there are clearly some
physical dependencies being omitted, namely $\rdiff{\rho_s}{\rho}$,
$\rdiff{\rho}{c_s}$, $\rdiff{\rho \vu}{c_s}$, and $\rdiff{\rho E}{c_s}$.  It has
been found\cite{candler} that omitting this dependencies does not hinder convergence
the primal flow solver; however, because the adjoint requires an exact
linearization of the converged steady-state solution, these must be accounted
for in the decoupled adjoint formulation.

The next step is to reconcile the split conserved variables, $\mathbf{U}'$ and
$\mathbf{\hat{U}}$, with the conserved variable vector $\mq$ in the discrete
adjoint formulation given in \eref{adjoint-main}. The most intuitive and
staightforward way to do this is to forgo solving for the species mass $\rho_s$
in lieu of the species mass fraction $c_s$.  Thus, $\mq$ can be expressed as
%------------------------------------------------------------------------------%
\begin{equation}
  \mq =
  \begin{pmatrix}
  	\rho \\
  	\rho u \\
  	\rho v \\
  	\rho w \\
  	\rho E \\
    c_1 \\
    \vdots \\
    c_{ns}
  \end{pmatrix}
  \label{q-vec}
\end{equation}
%------------------------------------------------------------------------------%
This allows the linearizations in \erefs{approx-jac}{approx-jac-dc} to be used
in the adjoint formulation, by augmenting them with the previously omitted
linearizations.  Replacing $\rdiff{}{\mq}$ with the fully-coupled system, the
adjoint system becomes
%------------------------------------------------------------------------------%
\begin{equation}
  \begin{pmatrix}
    \rtdiff{\rho}{\rho} & \rtdiff{\rho \vu}{\rho} & 
    \rtdiff{\rho E}{\rho} & \rtdiff{\rho_s}{\rho} \\ \\
    \rtdiff{\rho}{\rho \vu} & \rtdiff{\rho \vu}{\rho \vu} & 
    \rtdiff{\rho E}{\rho \vu} & \rtdiff{\rho_s}{\rho \vu} \\ \\
    \rtdiff{\rho}{\rho E} & \rtdiff{\rho \vu}{\rho E} & 
    \rtdiff{\rho E}{\rho E} & \rtdiff{\rho_s}{\rho E} \\ \\
    \rtdiff{\rho}{c_s} & \rtdiff{\rho \vu}{c_s} & 
    \rtdiff{\rho E}{c_s} & \rtdiff{\rho_s}{c_s}
  \end{pmatrix}
  \begin{pmatrix}
    \adjlam{\rho} \\ \\
    \adjlam{\rho \vu} \\ \\
    \adjlam{\rho E} \\ \\
    \adjlam{c_s}
  \end{pmatrix}
  = -
  \begin{pmatrix}
    \pd{f}{\rho} \\ \\
    \pd{f}{\rho \vu} \\ \\
    \pd{f}{\rho E} \\ \\
    \pd{f}{c_s}
  \end{pmatrix}
  \label{full-adjoint}
\end{equation}
%------------------------------------------------------------------------------%
Thus the jacobian in \eref{full-adjoint} is the completed one of
\erefs{approx-jac}{approx-jac-dc}.  While this is useful, the advantage of
decoupling the species equations from the mixture equations was to speed up the
linear solver and save memory.  Solving \eref{full-adjoint} is roughly
equivalent to solving the fully-coupled system of equations, which undermines
both of these goals; so, an alternative solution strategy must be formulated.  If
a block jacobi scheme is employed, the system can be decoupled once again as
%------------------------------------------------------------------------------%
\begin{equation}
  \begin{pmatrix}
    \rtdiff{\rho}{\rho}     & \rtdiff{\rho \vu}{\rho}     & \rtdiff{\rho E}{\rho} \\ \\
    \rtdiff{\rho}{\rho \vu} & \rtdiff{\rho \vu}{\rho \vu} & \rtdiff{\rho E}{\rho \vu} \\ \\
    \rtdiff{\rho}{\rho E}   & \rtdiff{\rho \vu}{\rho E}   & \rtdiff{\rho E}{\rho E}
  \end{pmatrix}
  \begin{pmatrix}
    \adjlam{\rho} \\ \\
    \adjlam{\rho \vu} \\ \\
    \adjlam{\rho E}
  \end{pmatrix}
  = -
  \begin{pmatrix}
    \pd{f}{\rho} \\ \\
    \pd{f}{\rho \vu} \\ \\
    \pd{f}{\rho E}
  \end{pmatrix}
  -
  \begin{pmatrix}
    \rtdiff{\rho_s}{\rho} \\ \\
    \rtdiff{\rho_s}{\rho \vu} \\ \\
    \rtdiff{\rho_s}{\rho E}
  \end{pmatrix}
  \adjlam{c_s}
  \label{dc-adjoint-1}
\end{equation}
%------------------------------------------------------------------------------%
\begin{equation}
  \rtdiff{\rho_s}{c_s}
  \adjlam{c_s}
  = -\pd{f}{c_s}
  - \rtdiff{\rho}{c_s} \adjlam{\rho}
  - \rtdiff{\rho \vu}{c_s} \adjlam{\rho \vu}
  - \rtdiff{\rho E}{c_s} \adjlam{\rho E}
  \label{dc-adjoint-2}
\end{equation}
%------------------------------------------------------------------------------%
Adding a time-like derivative to the adjoint equations, the solution of the
costate variables, $\Lambda$, can be time marched similar to the primal flow
solver
%------------------------------------------------------------------------------%
\begin{equation}
  \left[ \frac{V}{\Delta t} \mi + \rtdiff{}{\mq} \right] \Delta
  \adjlam{}
  = -\pd{f}{\mq} - \rtdiff{}{\mq} \adjlam{}
  \label{adj-dc-time-general}
\end{equation}
%------------------------------------------------------------------------------%
Thus, the first system of equations in \eref{dc-adjoint-1} becomes
%------------------------------------------------------------------------------%
\begin{equation}
  \begin{split}
    \left[
      \frac{V}{\Delta t} \mi +
      \begin{pmatrix}
        \rtdiff{\rho}{\rho} & \rtdiff{\rho \vu}{\rho} & \rtdiff{\rho E}{\rho} \\ \\
        \rtdiff{\rho}{\rho \vu} & \rtdiff{\rho \vu}{\rho \vu} & \rtdiff{\rho E}{\rho \vu} \\ \\
        \rtdiff{\rho}{\rho E} & \rtdiff{\rho \vu}{\rho E} & \rtdiff{\rho E}{\rho E}
      \end{pmatrix}
    \right]
    \begin{pmatrix}
      \Delta \adjlam{\rho} \\ \\
      \Delta \adjlam{\rho \vu} \\ \\
      \Delta \adjlam{\rho E}
    \end{pmatrix}
    &= \\ -
    \begin{pmatrix}
      \pd{f}{\rho} \\ \\
      \pd{f}{\rho \vu} \\ \\
      \pd{f}{\rho E}
    \end{pmatrix}
    -
    \begin{pmatrix}
      \rtdiff{\rho}{\rho} & \rtdiff{\rho \vu}{\rho} & \rtdiff{\rho E}{\rho} \\ \\
      \rtdiff{\rho}{\rho \vu} & \rtdiff{\rho \vu}{\rho \vu} & \rtdiff{\rho E}{\rho \vu} \\ \\
      \rtdiff{\rho}{\rho E} & \rtdiff{\rho \vu}{\rho E} & \rtdiff{\rho E}{\rho E}
    \end{pmatrix}
    &
    \begin{pmatrix}
      \adjlam{\rho} \\ \\
      \adjlam{\rho \vu} \\ \\
      \adjlam{\rho E}
    \end{pmatrix}
    -
    \begin{pmatrix}
      \rtdiff{\rho_s}{\rho} \\ \\
      \rtdiff{\rho_s}{\rho \vu} \\ \\
      \rtdiff{\rho_s}{\rho E}
    \end{pmatrix}
    \adjlam{c_s}
  \end{split}
  \label{adj-dc-time1}
\end{equation}
%------------------------------------------------------------------------------%
and the second system in \eref{dc-adjoint-2} becomes
%------------------------------------------------------------------------------%
\begin{equation}
  \left( \frac{V}{\Delta t} \mi + \rtdiff{\rho_s}{c_s} \right) \Delta
  \adjlam{c_s}
  = \\ -\pd{f}{c_s}
  - \rtdiff{\rho_s}{c_s} \adjlam{c_s}
  - \rtdiff{\rho}{c_s} \adjlam{\rho}
  - \rtdiff{\rho \vu}{c_s} \adjlam{\rho \vu}
  - \rtdiff{\rho E}{c_s} \adjlam{\rho E}
  \label{adj-dc-time2}
\end{equation}
%------------------------------------------------------------------------------%
The LHS of \erefs{adj-dc-time1}{adj-dc-time2} are the same first-order
approximate jacobians that were used to solve the primal flow equations;
therefore, all of the benefits of the diagonal block matricies that are
exploited in the primal flow solver to reduce the linear solver cost and overall
memory now apply to the adjoint.
