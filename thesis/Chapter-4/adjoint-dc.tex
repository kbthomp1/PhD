%------------------------------------------------------------------------------%
% Decoupled adjoint derivation content
%------------------------------------------------------------------------------%

The primal flow equations are solved using the decoupled scheme based on the
work by Candler, et. al.  In doing this the conserved variables are split from
%------------------------------------------------------------------------------%
\begin{equation}
	\vU =
  \begin{pmatrix}
 		\rho_1    \\
		\vdots    \\
		\rho_{ns} \\
		\rho u    \\
		\rho v    \\
		\rho w    \\
		\rho E    \\
	\end{pmatrix}
  \label{all-vars}
 \end{equation}
%------------------------------------------------------------------------------%
 into
%------------------------------------------------------------------------------%
\begin{equation}
	\begin{matrix}
		\mathbf{U}'=\begin{pmatrix}
			\rho \\
			\rho u \\
			\rho v \\
			\rho w \\
			\rho E
		\end{pmatrix},\quad &
		\mathbf{\hat{U}}=\begin{pmatrix}
			\rho_1 \\
			\vdots \\
			\rho_{ns}
		\end{pmatrix}
	\end{matrix}
  \label{dc-vars}
\end{equation}
%------------------------------------------------------------------------------%
By doing this, the flow equations can be solved implicitly using an approximate,
first-order jacobian to drive the flow solution to a converged steady-state.  In
practice, this is done by essentially formulating
%------------------------------------------------------------------------------%
\begin{equation}
  \frac{V}{\Delta t}\mi + 
  \begin{pmatrix}
    \rdiff{\rho}{\rho} & \rdiff{\rho}{\rho \vu} & \rdiff{\rho}{\rho E} \\ \\
    \rdiff{\rho \vu}{\rho} & \rdiff{\rho \vu}{\rho \vu} & \rdiff{\rho \vu}{\rho E} \\ \\
    \rdiff{\rho E}{\rho} & \rdiff{\rho E}{\rho \vu} & \rdiff{\rho E}{\rho E}
  \end{pmatrix}
  \begin{pmatrix}
    \Delta \rho \\ \\
    \Delta \rho \vu \\ \\
    \Delta \rho E
  \end{pmatrix}
  =
  \begin{pmatrix}
    \res{\rho} \\ \\
    \res{\rho \vu} \\ \\
    \res{\rho E}
  \end{pmatrix}
  \label{approx-jac}
\end{equation}
%------------------------------------------------------------------------------%
to solve for the mixture flow variables, and formulating
%------------------------------------------------------------------------------%
\begin{equation}
  \frac{V}{\Delta t}\mi + 
  \begin{pmatrix}
    \rdiff{\rho_1}{c_1} & \cdots & \rdiff{\rho_{1}}{c_{ns}} \\ \\
    \vdots & \ddots & \vdots \\ \\
    \rdiff{\rho_{ns}}{c_1} & \cdots & \rdiff{\rho_{ns}}{c_{ns}}
  \end{pmatrix}
  \begin{pmatrix}
    \Delta c_1 \\ \\
    \vdots \\ \\
    \Delta c_{ns}
  \end{pmatrix}
  =
  \begin{pmatrix}
    \res{\rho_1} \\ \\
    \vdots \\ \\
    \res{\rho_{ns}}
  \end{pmatrix}
  \label{approx-jac-dc}
\end{equation}
%------------------------------------------------------------------------------%
Note that the correction terms required to maintain $\lsum{s=1}{ns}{c_s} = 1$
and $\lsum{s=1}{ns}{\delta c_s} = 0$ are omitted in \eref{approx-jac-dc} only
for brevity in this explanation.  Examining \erefs{approx-jac}{approx-jac-dc}
shows that there are clearly some physical dependencies being omitted, namely
$\rdiff{\rho_s}{\rho}$, $\rdiff{\rho}{c_s}$, $\rdiff{\rho \vu}{c_s}$, and 
$\rdiff{\rho E}{c_s}$.  It has been demonstrated that omitting this dependencies
does not hinder convergence the primal solver; however, the adjoint requires an
exact linearization of the converged steady-state solution.

The discrete adjoint formulation is given as
%------------------------------------------------------------------------------%
\begin{equation}
  \bigg[\rdiff{}{\mq}\bigg]^T\mathbf{\Lambda}= -\pd{f}{\mq}
  \label{adjoint}
\end{equation}
%------------------------------------------------------------------------------%
where $\mq$ is a vector of conserved variables, and $f$ is the cost function (i.e.
lift, drag, etc.). There is now an apparent need to reconcile the two sets of
conserved variables in \eref{dc-vars} with $\mq$.  The most intuitive and
staightforward way to do this is to forgo solving for the species mass $\rho_s$
in lieu of the species mass fraction $c_s$.  Thus, $\mq$ can be expressed as
%------------------------------------------------------------------------------%
\begin{equation}
  \mq =
  \begin{pmatrix}
  	\rho \\
  	\rho u \\
  	\rho v \\
  	\rho w \\
  	\rho E \\
    c_1 \\
    \vdots \\
    c_{ns}
  \end{pmatrix}
  \label{q-vec}
\end{equation}
%------------------------------------------------------------------------------%
This allows the linearizations derived in \erefs{approx-jac}{approx-jac-dc} to
be used in the adjoint formulation, by augmenting them with the previously
omitted linearizations.  Replacing $\rdiff{}{\mq}$ with the fully system, the
adjoint system becomes
%------------------------------------------------------------------------------%
\begin{equation}
  \begin{pmatrix}
    \rtdiff{\rho}{\rho} & \rtdiff{\rho}{\rho \vu} & 
    \rtdiff{\rho}{\rho E} & \rtdiff{\rho}{c_s} \\ \\
    \rtdiff{\rho \vu}{\rho} & \rtdiff{\rho \vu}{\rho \vu} & 
    \rtdiff{\rho \vu}{\rho E} & \rtdiff{\rho \vu}{c_s} \\ \\
    \rtdiff{\rho E}{\rho} & \rtdiff{\rho E}{\rho \vu} & 
    \rtdiff{\rho E}{\rho E} & \rtdiff{\rho E}{c_s} \\ \\
    \rtdiff{\rho_s}{\rho} & \rtdiff{\rho_s}{\rho \vu} & 
    \rtdiff{\rho_s}{\rho E} & \rtdiff{\rho_s}{c_s}
  \end{pmatrix}
  \begin{pmatrix}
    \adjlam{\rho} \\ \\
    \adjlam{\rho \vu} \\ \\
    \adjlam{\rho E} \\ \\
    \adjlam{c_s}
  \end{pmatrix}
  = -
  \begin{pmatrix}
    \pd{f}{\rho} \\ \\
    \pd{f}{\rho \vu} \\ \\
    \pd{f}{\rho E} \\ \\
    \pd{f}{c_s}
  \end{pmatrix}
  \label{full-adjoint}
\end{equation}
%------------------------------------------------------------------------------%
Thus the jacobian in \eref{full-adjoint} is the completed one of
\erefs{approx-jac}{approx-jac-dc}.  While this is useful, the point of
decoupling the species equations from the mixture equations was to speed up the
linear solver and save memory.  Solving \eref{full-adjoint} undermines both of
these goals, so an alternative solution strategy must be formulated.  If a block
jacobi scheme is employed, the system can be decoupled once again as
%------------------------------------------------------------------------------%
\begin{equation}
  \begin{pmatrix}
    \rtdiff{\rho}{\rho} & \rtdiff{\rho}{\rho \vu} & \rtdiff{\rho}{\rho E} \\ \\
    \rtdiff{\rho \vu}{\rho} & \rtdiff{\rho \vu}{\rho \vu} & \rtdiff{\rho \vu}{\rho E} \\ \\
    \rtdiff{\rho E}{\rho} & \rtdiff{\rho E}{\rho \vu} & \rtdiff{\rho E}{\rho E}
  \end{pmatrix}
  \begin{pmatrix}
    \adjlam{\rho} \\ \\
    \adjlam{\rho \vu} \\ \\
    \adjlam{\rho E}
  \end{pmatrix}
  = -
  \begin{pmatrix}
    \pd{f}{\rho} \\ \\
    \pd{f}{\rho \vu} \\ \\
    \pd{f}{\rho E}
  \end{pmatrix}
  -
  \begin{pmatrix}
    \rtdiff{\rho}{c_s} \\ \\
    \rtdiff{\rho \vu}{c_s} \\ \\
    \rtdiff{\rho E}{c_s}
  \end{pmatrix}
  \adjlam{c_s}
  \label{dc-adjoint-1}
\end{equation}
%------------------------------------------------------------------------------%
\begin{equation}
  \rtdiff{\rho_s}{c_s}
  \adjlam{c_s}
  = -\pd{f}{c_s}
  - \rtdiff{\rho_s}{\rho} \adjlam{\rho}
  - \rtdiff{\rho_s}{\rho \vu} \adjlam{\rho \vu}
  - \rtdiff{\rho_s}{\rho E} \adjlam{\rho E}
  \label{dc-adjoint-2}
\end{equation}
%------------------------------------------------------------------------------%
If a time-like derivative is added to the adjoint, the solution of the costate
variables, $\Lambda$, can be time marched similar to the primal flow solver
%------------------------------------------------------------------------------%
\begin{equation}
  \left[ \frac{V}{\Delta t} \mi + \rtdiff{}{\mq} \right] \Delta
  \adjlam{}
  = -\pd{f}{\mq} - \rtdiff{}{\mq} \adjlam{}
  \label{adj-dc-time-general}
\end{equation}
%------------------------------------------------------------------------------%
Thus, the first system in \eref{dc-adjoint-1} becomes
%------------------------------------------------------------------------------%
\begin{equation}
  \begin{split}
    \left[
      \frac{V}{\Delta t} \mi +
      \begin{pmatrix}
        \rtdiff{\rho}{\rho} & \rtdiff{\rho}{\rho \vu} & \rtdiff{\rho}{\rho E} \\ \\
        \rtdiff{\rho \vu}{\rho} & \rtdiff{\rho \vu}{\rho \vu} & \rtdiff{\rho \vu}{\rho E} \\ \\
        \rtdiff{\rho E}{\rho} & \rtdiff{\rho E}{\rho \vu} & \rtdiff{\rho E}{\rho E}
      \end{pmatrix}
    \right]
    \begin{pmatrix}
      \Delta \adjlam{\rho} \\ \\
      \Delta \adjlam{\rho \vu} \\ \\
      \Delta \adjlam{\rho E}
    \end{pmatrix}
    &= \\ -
    \begin{pmatrix}
      \pd{f}{\rho} \\ \\
      \pd{f}{\rho \vu} \\ \\
      \pd{f}{\rho E}
    \end{pmatrix}
    -
    \begin{pmatrix}
      \rtdiff{\rho}{\rho} & \rtdiff{\rho}{\rho \vu} & \rtdiff{\rho}{\rho E} \\ \\
      \rtdiff{\rho \vu}{\rho} & \rtdiff{\rho \vu}{\rho \vu} & \rtdiff{\rho \vu}{\rho E} \\ \\
      \rtdiff{\rho E}{\rho} & \rtdiff{\rho E}{\rho \vu} & \rtdiff{\rho E}{\rho E}
    \end{pmatrix}
    &
    \begin{pmatrix}
      \adjlam{\rho} \\ \\
      \adjlam{\rho \vu} \\ \\
      \adjlam{\rho E}
    \end{pmatrix}
    -
    \begin{pmatrix}
      \rtdiff{\rho}{c_s} \\ \\
      \rtdiff{\rho \vu}{c_s} \\ \\
      \rtdiff{\rho E}{c_s}
    \end{pmatrix}
    \adjlam{c_s}
  \end{split}
  \label{adj-dc-time1}
\end{equation}
%------------------------------------------------------------------------------%
and the second system in \eref{dc-adjoint-2} becomes
%------------------------------------------------------------------------------%
\begin{equation}
  \left( \frac{V}{\Delta t} \mi + \rtdiff{\rho_s}{c_s} \right) \Delta
  \adjlam{c_s}
  = \\ -\pd{f}{c_s}
  - \rtdiff{\rho_s}{c_s} \adjlam{c_s}
  - \rtdiff{\rho_s}{\rho} \adjlam{\rho}
  - \rtdiff{\rho_s}{\rho \vu} \adjlam{\rho \vu}
  - \rtdiff{\rho_s}{\rho E} \adjlam{\rho E}
  \label{adj-dc-time2}
\end{equation}
%------------------------------------------------------------------------------%
This is advantageous, because the jacobians on the left hand side (LHS) can be
the first-order approximate jacobians that were used to solve the primal flow
equations; hence, all of the benefits of the diagonal block matricies that are
exploited to reduce the linear solver cost and overall memory now apply to the
adjoint.
