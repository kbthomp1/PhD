\chapter{Change of Equation Sets}
\label{sec:change-of-equations}

In addition to a change of variables, the decoupled scheme also changes the
equations being solved, effectively solving is one more equation than the
fully-coupled scheme.  The residual vector for the decoupled scheme, $\rv{}$,
and the residual vector for the fully coupled scheme, $\ru{}$, can be written as
%------------------------------------------------------------------------------%
\begin{equation}
  \ru{} =
  \begin{pmatrix}
    \res{\rho_1} \\
    \vdots \\
    \res{\rho_{N_s}} \\
    \res{\rho \vu} \\
    \res{\rho E}
  \end{pmatrix}
  \rightarrow
  \rv{} =
  \begin{pmatrix}
    \res{\rho_1} - c_1 \resrho \\
    \vdots \\
    \res{\rho_{N_s}} - c_{N_s} \resrho \\
    \resrho \\
    \res{\rho \vu} \\
    \res{\rho E}
  \end{pmatrix}
  \label{fc-dc-res}
\end{equation}
%------------------------------------------------------------------------------%
\eref{fc-dc-res} shows that $\rv{}$ is completely comprised of components from
$\ru{}$; therefore, a relationship between these equation sets can be derived in
a similar fashion to Appendix \ref{change-of-var-section}.  Rewriting $\rv{}$ in
terms of $\ru{}$
%------------------------------------------------------------------------------%
\begin{equation}
  \rv{} =
  \begin{pmatrix}
    \ru{s} - c_s \left( \lsum{i=1}{N_s}{\ru{i}} \right) \\
    \lsum{i=1}{N_s}{\ru{i}} \\
    \ru{N_s + 1} \\
    \ru{N_s + 2}
  \end{pmatrix}
  \label{ru-to-rv}
\end{equation}
%------------------------------------------------------------------------------%
it is possible to form the transformation
%------------------------------------------------------------------------------%
\begin{equation}
  \pd{\rv{}}{\ru{}} =
  \begin{pmatrix}
    1 - c_1  & -c_1     & \dots  & -c_1      & 0      & 0      \\
    -c_2     & 1-c_2    & \dots  & -c_2      & 0      & 0      \\
    \vdots   & \vdots   & \ddots & \vdots    & \vdots & \vdots \\
    -c_{N_s} & -c_{N_s} & \dots  & 1-c_{N_s} & 0      & 0      \\
    1        & 1        & \dots  & 1         & 0      & 0      \\
    0        & 0        & \dots  & 0         & 1      & 0      \\
    0        & 0        & \dots  & 0         & 0      & 1      \\
  \end{pmatrix}
  \label{drudrv}
\end{equation}
%------------------------------------------------------------------------------%
Because all transformations performed on the equation set are also performed
on the adjoint costate variables, \eref{drudrv} enables the mapping between the
costate variables associated with the fully coupled and decoupled flow solver
equations
%------------------------------------------------------------------------------%
\begin{equation}
  \adjlam{\mU} = \left(\pd{\rv{}}{\ru{}} \right)^{T} \adjlam{\mv}
  \label{adj-ru-to-rv}
\end{equation}
%------------------------------------------------------------------------------%
where costate variables associated with the fully coupled flow solver equations,
$\adjlam{\mU}$, come from solving
%------------------------------------------------------------------------------%
\begin{equation}
  \drudu \adjlam{\mU} = -\pd{f}{\mU}
  \label{fc-adjoint}
\end{equation}
%------------------------------------------------------------------------------%
and the costate variables associated with the decoupled flow solver equations,
$\adjlam{\mv}$, come from solving
%------------------------------------------------------------------------------%
\begin{equation}
  \drvdv \adjlam{\mv} = -\pd{f}{\mv}
  \label{dc-adjoint}
\end{equation}
%------------------------------------------------------------------------------%
\eref{adj-ru-to-rv} faciliates a right preconditioning of \eref{fc-adjoint} to
transform the adjoint solution as post-processing step.  This also decreases the
length of the solution vector storage required, since the fully coupled system
has one less equation than the decoupled system.

