%% ------------------------------ Abstract ---------------------------------- %%
\begin{abstract}

  An algorithm is described to efficiently compute aerothermodynamic design
sensitivities using a decoupled variable set.  In a conventional approach to
computing design sensitivities for reacting flows, the species continuity
equations are fully coupled to the conservation laws for momentum and energy. In
this algorithm, the species continuity equations are solved separately from the
mixture continuity, momentum, and total energy equations. This decoupling
simplifies the implicit system, so that the flow solver can be made
significantly more efficient, with very little penalty on overall scheme
robustness.  Most importantly, the computational cost of the point implicit
relaxation is shown to scale linearly with the number of species for the
decoupled system, whereas the fully coupled approach scales quadratically. Also,
the decoupled method significantly reduces the cost in wall time and memory in
comparison to the fully coupled approach. 

This decoupled approach for computing design sensitivities with the adjoint
system is demonstrated for inviscid flow in chemical non-equilibrium around a
re-entry vehicle with a retro-firing annular nozzle. The sensitivities of the
surface temperature and mass flow rate through the nozzle plenum are computed
with respect plenum conditions and verified against complex-variable
finite-difference.  The decoupled scheme significantly improved the
computational time and memory required to complete the optimization, making this
an attractive method for high-fidelity design of hypersonic vehicles.

\end{abstract}


%% ---------------------------- Copyright page ------------------------------ %%
%% Comment the next line if you don't want the copyright page included.
\makecopyrightpage

%% -------------------------------- Title page ------------------------------ %%
\maketitlepage

%% -------------------------------- Dedication ------------------------------ %%
\begin{dedication}
 \centering To my parents and friends.
\end{dedication}

%% -------------------------------- Biography ------------------------------- %%
\begin{biography}
The author was born on December 11th, 1990, in Willow Springs, North Carolina.
He recieved a B.S. in Aerospace Engineering from North Carolina State University
in 2012, and subsequently recieved his M.S. in Aerospace Engineering from North
Carolina State University in 2014.  After recieving his M.S., he began work in
the Aerothermodynamics branch of NASA Langley Research Center, via the Pathways
program.  He completed this disseration on adjoint-based optimization while
working at NASA Langley Research Center.
\end{biography}

%% ----------------------------- Acknowledgements --------------------------- %%
\begin{acknowledgements}
First, I would like to thank Dr. Peter Gnoffo, who has been an unending source
of support during my time NASA Langley Research Center.  I would like thank my
parents for all of their encouragement over the years, and their commitment to
seeing I be given the best opportunity to make the most of myself.  I would like
to thank my advisor, Dr. Hassan, for instilling in me a diligence to improve
myself and for mentoring me through many challenges.  I would like to thank Dr.
Jeff White in the Computational Aerosciences Branch at NASA Langley Research
Center for his advice on reacting flow simulation. I would like to thank the
Entry Systems Modeling Project within NASA's Game Changing Development Program
for their funding and support of this research.  Finally, I would like to
recognize the FUN3D team at NASA Langley Research Center, for their support and
availability to discuss many challenging problems I have encountered during the
course of my time at NASA.
\end{acknowledgements}

\thesistableofcontents

\thesislistoftables

\thesislistoffigures
