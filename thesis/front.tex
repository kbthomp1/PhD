%% ------------------------------ Abstract ---------------------------------- %%
\begin{abstract}

  Approach is described to efficiently compute aerothermodynamic design
sensitivities using a decoupled approach.  In this approach, the species
continuity equations are decoupled from the mixture continuity, momentum, and
total energy equations for the Roe flux difference splitting scheme in both the
flow and adjoint solvers. This decoupling simplifies the implicit system, so
that the flow solver can be made significantly more efficient, with very little
penalty on overall scheme robustness.  Most importantly, the computational cost
of the point implicit relaxation is shown to scale linearly with the number of
species for the decoupled system, whereas the fully coupled approach scales
quadratically. Also, the decoupled method significantly reduces the cost in wall
time and memory in comparison to the fully coupled approach. 

A design optimization of a re-entry vehicle with a annular nozzle on the
forebody is completed based on this approach.  The sensitivities of the drag
coefficient and surface temperature with respect to a plenum inboard the vehicle
were computed and verified against complex-variable finite-difference.
\end{abstract}


%% ---------------------------- Copyright page ------------------------------ %%
%% Comment the next line if you don't want the copyright page included.
\makecopyrightpage

%% -------------------------------- Title page ------------------------------ %%
\maketitlepage

%% -------------------------------- Dedication ------------------------------ %%
\begin{dedication}
 \centering To my parents and friends.
\end{dedication}

%% -------------------------------- Biography ------------------------------- %%
\begin{biography}
The author was born on December 11th, 1990, in Willow Springs, North Carolina.
He recieved a BS in Aerospace Engineering from North Carolina State University
in 2012, and subsequently recieved his MS in Aerospace Engineering from North
Carolina State University in 2014.  After recieving he MS, he began work in the
Aerothermodynamics branch of NASA Langley Research Center, via the Pathways
program.  He completed this disseration on adjoint-based optimization while
working at NASA Langley Research Center.
\end{biography}

%% ----------------------------- Acknowledgements --------------------------- %%
\begin{acknowledgements}
I would like thank my parents for all of their encouragement over the years, and
their commitment to seeing I be given the best opportunity to make the most of
myself.  I would like to thank my advisor, Dr. Hassan, for instilling in me a
diligence to improve myself and for mentoring me through many challenges.  I
would like to thank the Entry Systems Modeling Project within NASA's Game
Changing Development Program for their funding and support of this research.
Finally, I would like to recognize the FUN3D team at NASA Langley Research
Center, for their support and availability to discuss many challenging problems
I have encountered during the course of my time at NASA.
\end{acknowledgements}

\thesistableofcontents

\thesislistoftables

\thesislistoffigures
