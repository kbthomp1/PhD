\chapter{Verification of Adjoint Sensitivity Gradients}
\label{chapter-seven}

In this chapter the sensitivity gradients computed by the adjoint-formulation
are verified against derivatives obtained with a complex step, which is also
described.  Additionally, this chapter details the methods by which the gradient
information is computed in the forward-mode and in the reverse-mode.

\section{Forward-mode Sensitivities Using Complex Vari\-ables}

The sensitivities can be computed in the forward-mode by using
finite-difference.  While this approach is unfavorable to use in practice,
because a minimum number of flow solves equivalent to the number of design
variables are required, it is a straight-forward way to verify that
sensitivities computed by the adjoint solver are correct.  To avoid cancellation
errors associated with real-variable finite difference, an approach that uses
complex variables was originally suggested by Squire and Trapp\cite{squire1998}
and evaluated by Newman et al\cite{newman1998}.  This complex-variable approach
can be used to determine the derivative of a real valued function, $f$, by
considering the Taylor series expansion of $f$ using a complex step $ih$
%------------------------------------------------------------------------------%
\begin{equation}
  f(x + ih) = \sum_{k=0}\left( \frac{(ih)^k}{!k}\pnd{f}{x}{k} \right)
            = f(x) + ih\pd{f}{x} - \frac{h^2}{2}\pnd{f}{x}{2}
            - \frac{h^3}{6}\pnd{f}{x}{3} + \cdots
  \label{taylor-exp}
\end{equation}
%------------------------------------------------------------------------------%
taking the imaginary parts of both sides of \eref{taylor-exp} and solving for
the first derivative yields
%------------------------------------------------------------------------------%
\begin{equation}
  \begin{aligned}
    \pd{f}{x} &= \frac{Im\left[ f(x + ih) \right]}{h} -
    \frac{h^2}{6}\pnd{f}{x}{3} \\
              &= \frac{Im\left[ f(x + ih) \right]}{h} - O(h^2)
  \end{aligned}
  \label{complex-fd}
\end{equation}
%------------------------------------------------------------------------------%
Thus, this complex-variable approach provides a means of computing the cost
function derivatives without subtractive cancellation error, giving truly second
order accuracy.  The power of using this approach is that very little additional
code is required to compute the derivative of any function with respect to a
single design variable.  All that is needed is all functions be made to use
complex arithmetic and a complex-valued perturbation be applied.

In practice, this complex-variable approach is implemented by transforming the
source code such that all real variables used in a function evaluation are
re-declared to be complex variables.  A specified step size is then added to the
complex part of the design variable that the cost function is to be linearized
with respect to in the function evaluation.  Provided the ``complexified''
solver converges in the same manner as the real-variable solver, the complex
part of the function of interest is the sensitivity derivative.

As mentioned before, the complex-variable and real-variable solvers must
converge to the same solution for the comparison between the adjoint-computed
sensitivity derivatives and those computed by the complex-variable approach to
be valid.  Section \ref{sec:frozen-limiter} described the need for a frozen
limiter to satisfy the Lagrangian in \eref{lagrangian}, from which the discrete
adjoint equations are derived.  A significant obstacle with using a frozen
limiter is that the real-variable and complex-variable solvers do not freeze the
limiter identically.  Although the solvers are nearly identical, their
floating-point operations will be different since complex arithmetic is involved
in only the complex-variable solver.  The flux limiter formulation is sensitive
to these differences and will result in a different converged state, regardless
of the limiter being frozen at the same iteration between the solvers.  

To overcome the sensitivity of solution to the flux limiter state, the
complex-variable solver must not be allowed to change the value of the flux
limiter.  Enforcing consistency between the real-value and complex-value flow
solver is done by converging the real-value flow solver to machine precision,
and then starting the complex-variable solver with the flow field of the
converged real-variable solution.  The complex step is then added to the design
variable; however, the flux limiter is frozen.  Because the limiter value is
constant, the real solution from the complex-variable and real-variable
solutions will match identically, and the complex part of the solution will be
converged without ever needing to update the flux limiter.  This will ensure
that the sensitivity derivatives from the complex and adjoint solvers match to
high precision for hypersonic cases with a frozen flux limiter.

\section{Verification of 2nd-order Adjoint Linearizations}

To verify the hand-coded linearizations implemented in the adjoint solver, the
derivatives of the drag, surface temperature, and mass flow rate cost functions
components for the annular nozzle geometry were computed using both the adjoint
method and the complex-variable approach.  To facilitate checking the
linearizations of all of these components efficiently, a composite cost function
was formed with three components
%------------------------------------------------------------------------------%
\begin{equation} 
  f = w_1\left( \massflow - \massflow^{*}\right)^{2} 
  + w_2\left( T_{RMS} - T_{RMS}^*\right)^2
  + w_3\left( C_{D} - C_{D}^*\right)^2
  \label{check-cost-func}
\end{equation}
%------------------------------------------------------------------------------%
By combining all components into a single composite function, only one
real-valued flow and one adjoint solution are needed to obtain the sensitivity
derivatives for all design variables, computed by \eref{obj-linearization2}. One
complex-valued flow solution is needed for each design variable.
\tref{tab:pg-deriv-check} shows the relative difference between the sensitivity
derivatives computed by \eref{obj-linearization1} and \eref{complex-fd}, for a
perfect gas annular jet simulation.
%------------------------------------------------------------------------------%
\begin{table}[h]
  \centering 
  \caption{Sensitivity derivative comparison - perfect gas.}
  \begin{tabular}{c|c|c|c} 
    Design Variable & Adjoint & Complex & Relative Difference \\ 
    \hline 
    $P_{p,o}$ & 0.120678601062E-04 & 0.120678601067E-04 & 4.54e-11 \\
    $T_{p,o}$ & 0.366546351179E-03 & 0.366546351181E-03 & 5.67e-12 
  \end{tabular}
  \label{tab:pg-deriv-check}
\end{table}
%------------------------------------------------------------------------------%
The adjoint and complex solvers match within machine zero, which is $\sim
10^{-15}$ for this case.  \tref{tab:frozen-deriv-check} shows the same
comparison as that in \tref{tab:pg-deriv-check}, but with a $H_2$-$N_2$ mixture
ejected in a 5-species freestream air mixture with frozen flow
%------------------------------------------------------------------------------%
\begin{table}[h] 
  \centering 
  \caption{Sensitivity derivative comparison - $H_2$-$N_2$ frozen.}
  \begin{tabular}{c|c|c|c} 
    Design Variable & Adjoint & Complex & Relative Difference\\
    \hline 
    $P_{p,o}$ & -0.18451007644E-06 & -0.184510076442E-06 & 2.27e-11 \\ 
    $T_{p,o}$ &  0.62086963151E-03 &  0.620869631517E-03 & 4.93e-13 \\ 
    $\fa$     & -0.34045335117E-01 & -0.340453351177E-01 & 5.86e-12 
  \end{tabular}
  \label{tab:frozen-deriv-check}
\end{table}
%------------------------------------------------------------------------------%
There is very strong agreement between all the design variable sensitivities
computed by the adjoint and complex solvers, with all matching discretely to
within 11 digits.  \tref{tab:react-deriv-check} shows the same comparison as
those in \tref{tab:frozen-deriv-check}, with reactions allowed to take place.
%------------------------------------------------------------------------------%
\begin{table}[h]
  \centering
  \caption{Sensitivity derivative comparison - $H_2$-$N_2$ reacting.}
  \begin{tabular}{c|c|c|c}
    Design Variable & Adjoint & Complex & Relative Difference\\
    \hline
    $P_{p,o}$ & -0.110813156019E-06 & -0.1105297746595E-06 & 2.56e-03 \\
    $T_{p,o}$ &  0.190899412373E-03 &  0.1908928479338E-03 & 3.44e-05 \\
    $\fa$     & -0.280354090455E-01 & -0.2802517317281E-01 & 3.65e-04
  \end{tabular}
  \label{tab:react-deriv-check}
\end{table}
%------------------------------------------------------------------------------%
It is clear that these do not match as well as those in
\trefs{tab:pg-deriv-check}{tab:frozen-deriv-check}.  The difference is explained
by the relative convergence of the flow solver residuals shown in
\fref{fig:chem-res-comp}.  Because convergence of the flow solver residuals
involving chemistry typically stalls at eight orders of magnitude higher than
without chemistry, it should be likewise be expected that the sensitivity
derivatives also degrade in comparison.  The large condition number increase
from the chemical source term linearizations being added to the Jacobian is
believed to be the cause of the stalled convergence. Despite this poor relative
agreement, \sref{sec:2nd-order-direct-design} demonstrates that the accuracy of
the derivatives in \tref{tab:react-deriv-check} is more than sufficient to
complete design optimizations.
