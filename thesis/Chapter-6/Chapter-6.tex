\chapter{Design Optimization}
\label{chapter-six}

Design optimization is a wide field that encompasses methods generally falling
into two categories: local gradient-based optimization, and heuristic global
optimization.  Local gradient-based optimization techniques focus on determining
an optimality condition by evaluating a function and its gradients.  Provided
certain conditions are met, it can be proven that the optimization procedure
will find a local minimum or maximum on a bounded domain.  Examples of local
gradient-based optimization methods include
steepest-descent\cite{fletcher1963rapidly}, sequential quadratic programming
(SQP)\cite{SNOPT-alg}, as well as an interesting method that converts a
constrained optimization problem into an unconstrained one by employing the
Kreisselmeier-Steinhauser function\cite{wrenn1989indirect}.  A heuristic global
optimization seeks to find the global extrema of a function.  Although these
methods are powerful, they often require a prohibitive number of function
evaluation and are not the focus this research.

In the field of optimization, the function of interest is referred to as the
``cost function'' or ``objective function''.  Optimization methods seek to
minimize this function; therefore, if the intent is to find the maximum value
of the function, it should be formulated as the negative of the original.  This
section focuses on the optimization of the annular jet demonstration problem
that was discussed in \cref{chapter-five}.  Details on the implementation of the
cost function components and design variables used in the optimization are
presented, as well a sample first-order inverse design optimization and
second-order direct design optimization.  The purpose of an inverse design
optimization is to match a target design by perturbing a set of design inputs,
whereas the purpose of a direct design optimization is to improve an initial
design by perturbing those same design inputs.

\section{Integrated Quantities of Interest}

The primary objective of this optimization is to explore the effects of the plume
from the annular jet interacting with the bow shock.  Rather than focus on the
net force contribution of this annular jet geometry to drag, as was previously
investigated by Gnoffo et. al \cite{gnoffo2016tapping}, the primary objective of
this study is to efficiently cool the surface of the vehicle with minimal mass
added to the vehicle.  Since only inviscid flow was done in this study, and the
root-mean-square (RMS) of surface temperature is taken as an analog to the
surface heating rate.  Likewise, the mass flow rate through nozzle plenum
boundary is taken as an analog to the addition of mass to the vehicle.

The RMS of surface temperature is defined as
%------------------------------------------------------------------------------%
\begin{equation}
  T_{RMS} =
  \sqrt{
    \frac{\sum_{i}^{N_{faces}}\left( T_{RMS} A_i \right)^2}
       {\sum_{i}^{N_{faces}}\left( A_i \right)^2}
     }
  \label{tt-rms-def}
\end{equation}
%------------------------------------------------------------------------------%
The area-weighted RMS of surface temperature was chosen over a simple
area-weighted average of surface temperature, because the temperatures on the
vehicle forebody are likely non-uniform, and there are regions of very high
temperature near the stagnation region of a vehicle forebody in hypersonic
flows.  Squaring of temperature in the RMS will give greater weight to these
high-temperature regions in the design, as these are in the most danger of
burn-through.
%------------------------------------------------------------------------------%
\begin{figure}[h]
  \centering
	\begin{subfigure}[b]{0.4\textwidth}
    \centering
    \includegraphics[width=\textwidth]{figures/surface2.png}
    \caption{Surface temperature RMS integrated area}
    \label{fig:t-rms-area}
  \end{subfigure}
	\begin{subfigure}[b]{0.4\textwidth}
    \centering
    \includegraphics[width=\textwidth]{figures/plenum_bc.png}
    \caption{Mass flow rate integrated area}
    \label{fig:plenum-face}
  \end{subfigure}
  \caption{Cost function component integrated areas.}
  \label{fig:cost-func-integrated-areas}
\end{figure}
%------------------------------------------------------------------------------%

The mass flow rate, $\massflow$,
through the area outlined in \fref{fig:plenum-face} is computed as
%------------------------------------------------------------------------------%
\begin{equation}
  \massflow = \sum_{i}^{N_{faces}}\left( \rho_i \overline{U} A \right)
  \label{mass-flow-eqn}
\end{equation}
%------------------------------------------------------------------------------%
and is used as a metric for the amount of propellent that is required to be
carried by the vehicle for blowing.  Again, for the purposes of this
demonstration problem, a lower mass flow rate at the plenum is equated to less
total vehicle mass.

\section{Composite Cost Function Definition and Components}
\label{cost-func-components}

The cost function (or objective function) as formulated in FUN3D is a composite,
weighted function
%------------------------------------------------------------------------------%
\begin{equation}
  f = \sum_{j=1}^{N_{func}}w_j\left( C_j - C_{j^*} \right)^{p_j}
  \label{generic-cost-function}
\end{equation}
%------------------------------------------------------------------------------%
Where $w_j$, $C_{j^*}$, and $p_j$ are the weight, target, and power of cost
function component $j$.  $C_j$ is the component value, which is evaluated at
each flow solution.  For example, an optimization problem that seeks to minimize the
surface temperature RMS without decreasing the drag, the cost function is
defined as 
%------------------------------------------------------------------------------%
\begin{equation}
  f = w_1\left( T_{RMS} \right)^{2} + w_2\left( C_{D} - C_{D}^{*} \right)^2
  \label{cd-tt-cost-function}
\end{equation}
%------------------------------------------------------------------------------%
For this case the component weights must be determined heuristically, to normalize
the changes in drag coefficient, $C_D$, and surface temperature Root-Mean-Square
(RMS) $T_{RMS}$.  The terms in \eref{cd-tt-cost-function} are squared to provide
a convex design space.

\section{Design Variables}

The design variables for the optimization problem are the plenum total pressure,
$P_{p,o}$, plenum total temperature, $T_{p,o}$, and plenum ``fuel-air ratio'',
$\fa$.  These are provided explicitly in the optimization problem, and are used
to directly set the flow conditions on plenum face boundary condition in the
nozzle, shown in \fref{fig:plenum-face}.  For a reacting gas mixture, the
``fuel-air ratio'' specifies the mass fractions for two species leaving the
plenum.  For example, if an $H_2-N_2$ mixture is ejected from the annular
nozzle, the mass fractions of $H_2$ and $N_2$ are given by
%------------------------------------------------------------------------------%
\begin{equation}
  \begin{aligned}
    c_{H_2} &= \fa \\
    c_{N_2} &= 1 - \fa
  \end{aligned}
  \label{fuel-air-def}
\end{equation}
%------------------------------------------------------------------------------%
Thus, the ratio $\fa$ dictates the mass fractions for two species injected into
the domain via the plenum boundary.

\section{Obtaining Sensitivity Gradients for Design Variables}

The sensitivity gradients for the plenum design variables are easily obtained by
manipulating \eref{obj-function} to obtain
%------------------------------------------------------------------------------%
\begin{equation}
  \pd{L}{\md} = \pd{f}{\md} + \rtdiff{}{\md} \mathbf{\Lambda}
  \label{obj-linearization1}
\end{equation}
%------------------------------------------------------------------------------%
For the cost function components listed in \sref{cost-func-components}, there is
no direct dependence on the plenum design variables; thus,
\eref{obj-linearization1} can be reduced to
%------------------------------------------------------------------------------%
\begin{equation}
  \pd{L}{\md} = \rtdiff{}{\md} \mathbf{\Lambda}
  \label{obj-linearization2}
\end{equation}
%------------------------------------------------------------------------------%
Once the adjoint costate variables $\mathbf{\Lambda}$ have been computed by
solving the adjoint equations (\eref{adjoint-main}) the sensitivity derivatives
of the cost function with respect to the plenum design variables are obtained by
evaluating relatively inexpensive matrix-vector products.

\section{First-Order Inverse Design Optimization}
\label{inv-design-opt}

The optimization procedure is done using the \textit{opt\_driver} utility in
FUN3D, which is a wrapper utility that executes the FUN3D flow solver, adjoint
solver, and optimization algorithm sequentially.  These steps are repeated until
a termination criterion is reached.  In practice, the termination of the
optimization occurs when the cost function reaches a tolerance of less that
$10^{-8}$, or when continuing towards the optimal condition would exceed the
prescribed upper or lower bounds of the design variables.  Due to the
difficulties encountered with the flux limiter that were described in
\sref{sec:frozen-limiter}, a first-order inverse design optimization is shown in
this section.  Results of the second-order inverse design optimization failed
due to the high sensitivity to the interaction at which the limiter is frozen.

%------------------------------------------------------------------------------%
\begin{figure}[h]
  \centering
	\begin{subfigure}[b]{0.45\textwidth}
    \centering
    \includegraphics[width=\textwidth]{figures/1st-H2/cost_func.png}
    \caption{Composite value}
    \label{fig:cost-func-1st-H2}
  \end{subfigure}
	\begin{subfigure}[b]{0.45\textwidth}
    \centering
    \includegraphics[width=\textwidth]{figures/1st-H2/func_components.png}
    \caption{Component values}
    \label{fig:components-1st-H2}
  \end{subfigure}
  \caption{Cost function and component history.}
\end{figure}
%------------------------------------------------------------------------------%
To demonstrate the inverse design capability of an adjoint-based design
optimization, targets of 2000 K for the surface temperature RMS and 0.0024
$kg/s$ for the annular nozzle mass flow rate were specified.  These targets were
chosen semi-arbitrarily, and were heuristically determined to be feasible based
the design variable bounds. The design variables specified for this optimization
were the plenum total pressure, $P_{p,o}$ and the plenum ``fuel-air ratio'',
$\fa$.  The plenum total temperature, $T_{p,o}$, was fixed at 500 K.  A species
mixture consisting of $H_2$ and $N_2$ was blown from the plenum, with the mass
fractions dictated by $\fa$ as described in \eref{fuel-air-def}.  For the cost
function, the weights were chosen heuristically, such that 
%------------------------------------------------------------------------------%
\begin{equation}
  \frac{w_1}{w_2} = \frac{\cost{T_{RMS}}^2}{\cost{\dot{m}}^2}
  \label{weight-ratio}
\end{equation}
%------------------------------------------------------------------------------%
This results in a roughly equivalent weighting between the $T_{RMS}$ and
$\dot{m}$, which is desired as both targets should be met at optimality.
%------------------------------------------------------------------------------%
\begin{figure}[h]
  \centering
  \begin{subfigure}[b]{0.45\textwidth}
    \includegraphics[width=\textwidth]{figures/1st-H2/dv_hist.png}
    \caption{Design variable history}
    \label{fig:dv-hist-1st-H2}
  \end{subfigure}
  \begin{subfigure}[b]{0.45\textwidth}
    \includegraphics[width=\textwidth]{figures/1st-H2/fm_hist.png}
    \caption{Design variable history}
    \label{fig:fm-hist-1st-H2}
  \end{subfigure}
  \caption{Inverse design history.}
\end{figure}
%------------------------------------------------------------------------------%
Using the SNOPT optimizer, \fref{fig:cost-func-1st-H2} shows that the target
design was met within 13 function evaluations.  SNOPT explores the entire design
space, as is shown in the second function evaluation, where a spike in the cost
function occurred.  \fref{fig:dv-hist-1st-H2} indicates that the optimizer tried
the upper bound for the plenum pressure design variable, and found that
sensitivity derivatives indicated that the lower pressure was required. This is
common, and is an effective way to insure that there is a local minimum in the
prescribed bound.  The optimization terminated when the cost function value was
less than the tolerance of $10^{-8}$, and \fref{fig:components-1st-H2} shows
that both components of the cost function were within one order of magnitude of
each other during the optimization.  This last point is important, since
non-normalized components can skew the optimization results, where competing
components can cause oscillations in the function evaluations and stall the
optimization procedure.  \fref{fig:fm-hist-1st-H2} shows the history of the
surface temperature RMS and annular nozzle mass flow rate, with the design
targets.  The optimization clearly made significant progress early, with smaller
gains as the solution approached the target.

\section{Second-Order Direct Design Optimization}
\label{sec:2nd-order-direct-design}

A direct design problem is possible with the second-order reconstruction scheme,
by defining the termination of the optimization as the point where changes in the
design variables or cost function are below a supplied threshold.  Using the
same cost function components and design variables as in \sref{inv-design-opt},
a direct design problem was completed to minimize both mass flow rate and
surface temperature RMS.  The cost function was formulated as
%------------------------------------------------------------------------------%
\begin{equation}
  f = w_1\left( \massflow \right)^2 + w_2\left( T_{RMS} \right)^2
  \label{direct-design-cost-func}
\end{equation}
%------------------------------------------------------------------------------%
With the weights chosen in the same manner as the inverse design problem to
insure that the components are normalized to the same order of magnitude
%------------------------------------------------------------------------------%
\begin{equation}
  \frac{w_1}{w_2} = \frac{\left( \massflow \right)^2}{\left( T_{RMS} \right)^2}
  \label{direct-design-weights}
\end{equation}
%------------------------------------------------------------------------------%
The SNOPT optimizer was able to very quickly determine that blowing pure
$H_2$, i.e. $\fa = 1.0$, would yield the lowest surface temperature RMS and mass
flow rate.  \fref{fig:dd-cost-func-value} shows that most of the improvement in
the design was made within the first three function evaluations, and the
subsequent steps were significantly less.  The optimization was terminated by
the ninth function evaluation, as no meaningful improvement to the design was
made after that point.
%------------------------------------------------------------------------------%
\begin{figure}[h]
  \centering
	\begin{subfigure}[b]{0.4\textwidth}
    \centering
    \includegraphics[width=\textwidth]{figures/direct_design/cost-func.png}
    \caption{Composite cost function value}
    \label{fig:dd-cost-func-value}
  \end{subfigure}
	\begin{subfigure}[b]{0.4\textwidth}
    \centering
    \includegraphics[width=\textwidth]{figures/direct_design/components.png}
    \caption{Cost function components}
    \label{fig:dd-components}
  \end{subfigure}
  \caption{Direct design cost function.}
  \label{fig:dd-cost-func}
\end{figure}
%------------------------------------------------------------------------------%
\fref{fig:dd-components} verifies that weights determined by
\eref{direct-design-weights} were indeed sufficient to normalize $\massflow$ and
$T_{RMS}$ contributions to the composite cost function.
%------------------------------------------------------------------------------%
\begin{figure}[h]
  \centering
	\begin{subfigure}[b]{0.4\textwidth}
    \centering
    \includegraphics[width=\textwidth]{figures/direct_design/dv-hist.png}
    \caption{Design variable history.}
    \label{fig:dd-dv-hist}
  \end{subfigure}
	\begin{subfigure}[b]{0.4\textwidth}
    \centering
    \includegraphics[width=\textwidth]{figures/direct_design/mass-tt.png}
    \caption{$\massflow$ and $T_{RMS}$ design history}
    \label{fig:dd-mass-tt}
  \end{subfigure}
  \caption{Direct design history.}
  \label{fig:dd-history}
\end{figure}
%------------------------------------------------------------------------------%
\fref{fig:dd-dv-hist} shows that the optimization was largely dependent on the
plenum pressure, and \tref{tab:design-improvement} shows that larger pressure at
the plenum resulted in a 4.044\% lower surface temperature, and 18.93\% lower
mass flow rate.
%------------------------------------------------------------------------------%
\vspace{0.3cm}
\begin{table}[h]
  \centering
  \caption{Direct design optimization improvement.}
  \begin{tabular}{c|c|c|c}
    Component & Initial & Final & Improvement\\
    \hline
    $\dot{m}$, $kg/s$ & 1.268e-3 & 1.028e-3 & 18.93\% \\
    $T_{RMS}$, $K$    & 1088     & 1044     & 4.044\%
  \end{tabular}
  \label{tab:design-improvement}
\end{table}
%------------------------------------------------------------------------------%
Thus, the optimizer was able to improve both components of the cost function.
This is an excellent problem for a high-fidelity gradient-based
optimization, as the non-linear effects of plenum-shock interaction makes the
optimum plenum condition difficult to determine intuitively.

There is a much stronger dependence on the choice of cost function weights for
this problem than for the inverse design problem in \sref{inv-design-opt}.  For
the inverse design problem, the target mass flow rate and surface temperature
RMS were known a priori; therefore, the weighting was chosen as a purely
normalizing measure to accelerate convergence to the target condition.  For this
direct design problem the target mass flow rate and surface temperature RMS are
not know a priori, and the weights chosen have a direct impact on the optimum
condition.  A ``skewed'' weighting may be advisable from an engineering
perspective when attempting a direct design approach.  For example, if the
surface thermal protection system (TPS) is rated to withstand much higher
surface heating than what is nominally predicted, a higher weight might be given
to the mass flow rate in order to decreased the required vehicle mass.  The
heuristic nature of this approach can be avoided by setting a component target,
or converting a composite cost function component to an explicit constraint.
The latter option is more robust, but comes at the cost of an additional adjoint
solution.
