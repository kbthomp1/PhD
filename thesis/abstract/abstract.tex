%        File: abstract.tex
%     Created: Fri Feb 24 05:00 PM 2017 E
% Last Change: Fri Feb 24 05:00 PM 2017 E
%
\documentclass[a4paper]{article}
\begin{document}

An algorithm is described to efficiently compute aerothermodynamic design
sensitivities using a decoupled variable set.  In a conventional approach to
computing design sensitivities for reacting flows, the species continuity
equations are fully coupled to the conservation laws for momentum and energy. In
this algorithm, the species continuity equations are solved separately from the
mixture continuity, momentum, and total energy equations. This decoupling
simplifies the implicit system, so that the flow solver can be made
significantly more efficient, with very little penalty on overall scheme
robustness.  Most importantly, the computational cost of the point implicit
relaxation is shown to scale linearly with the number of species for the
decoupled system, whereas the fully coupled approach scales quadratically. Also,
the decoupled method significantly reduces the cost in wall time and memory in
comparison to the fully coupled approach. 

This decoupled approach for computing design sensitivities with the adjoint
system is demonstrated for inviscid flow in chemical non-equilibrium around a
re-entry vehicle with a retro-firing annular nozzle. The sensitivities of the
surface temperature and mass flow rate through the nozzle plenum are computed
with respect to plenum conditions and verified against sensitivities computed
using a complex-variable finite-difference approach.  The decoupled scheme
significantly reduces the computational time and memory required to complete the
optimization, making this an attractive method for high-fidelity design of
hypersonic vehicles.

\end{document}


