\chapter{Governing Equations}
\label{chapter-two}

In this section, the conservative equations governing fluid flow for inviscid,
chemically reacting flow are presented.  This research is extendable to
multi-temperature models to account for higher excitation modes than the
translational mode; however, the focus of this research uses a 1-temperature
model, so only a single, total energy equation is used.  The thermodynamic
relations and chemical kinetics models used are also presented in detail here.

\section{Reacting Flow Conservation Equations}

Conservation equations for a fluid mixture that is chemical non-equilibrium and
thermal equilibrium can be written as
%------------------------------------------------------------------------------%
\begin{equation}
  \begin{gathered}
    \text{\bf{Species Conservation}}: \\
     \pd{\rho_s}{t} + \pd{\rho_s u}{x} + \pd{\rho_s v}{y} + \pd{\rho_s w}{z} =
     w_s
  \end{gathered}
  \label{species-cons}
\end{equation}
%------------------------------------------------------------------------------%
\begin{equation}
  \begin{gathered}
    \text{\bf{Mixture Momentum Conservation}}: \\
    \begin{aligned}
    \pd{\rho u}{t} + \pd{\rho u^2}{x} + \pd{\rho u v}{y} + \pd{\rho u w}{z} &= -\pd{p}{x}\\
    \pd{\rho v}{t} + \pd{\rho v u}{x} + \pd{\rho v^2}{y} + \pd{\rho v w}{z} &= -\pd{p}{y}\\
    \pd{\rho w}{t} + \pd{\rho w u}{x} + \pd{\rho w v}{y} + \pd{\rho w^2}{z} &= -\pd{p}{z}
    \end{aligned}
  \end{gathered}
  \label{mix-mom-cons}
\end{equation}
%------------------------------------------------------------------------------%
\begin{equation}
  \begin{gathered}
    \text{\bf{Total Energy Conservation}}: \\
     \pd{\rho E}{t} 
    + \pd{\rho \left( E + p \right) u}{x} 
    + \pd{\rho \left( E + p \right) v}{y}
    + \pd{\rho \left( E + p \right) w}{z} = 0
  \end{gathered}
  \label{tot-energy-cons}
\end{equation}
%------------------------------------------------------------------------------%

\section{Thermodynamic Relationships}

The pressure, $p$, is defined as the sum of the partial pressures of the species
%------------------------------------------------------------------------------%
\begin{equation}
  p = \sum_{s=1}^{N_s} p_s
  \label{pressure-def}
\end{equation}
%------------------------------------------------------------------------------%
and the partial pressure of species $s$, $p_s$, is defined as
%------------------------------------------------------------------------------%
\begin{equation}
  p_s = \frac{\rho_s R_u T}{M_s}
  \label{partial-pressure-def}
\end{equation}
%------------------------------------------------------------------------------%
where $R_u$ is the universal gas constant and $M_s$ is the molecular weight of
species $s$.  The total energy per unit mass $E$, is defined as
%------------------------------------------------------------------------------%
\begin{equation} 
  E = \sum{c_s e_s} + \frac{u^2+v^2+w^2}{2} 
  \label{tot-energy-def}
\end{equation}
%------------------------------------------------------------------------------%
where $c_s$ is the mass fraction of species $s$, defined as
%------------------------------------------------------------------------------%
\begin{equation}
  c_s = \frac{\rho_s}{\rho}
  \label{mass-frac-def}
\end{equation}
%------------------------------------------------------------------------------%
and $e_s$ is the specific internal energy of species $s$, defined as
%------------------------------------------------------------------------------%
\begin{equation}
  e_s = \int_{T_{ref}}^{T} \cvs \ dT + e_{s,o}
  \label{int-energy-def}
\end{equation}
%------------------------------------------------------------------------------%
where $T_ref$ is a reference temperature, with $e_{s,o}$ being the specific
internal energy.  In practice, the specific heat at constant volume for species
$s$, $\cvs$, is not used directly. Instead, the specific heat at constant
pressure for species $s$, $\cps$, is determined via thermodynamic curve
fits and related to $\cvs$ via
%------------------------------------------------------------------------------%
\begin{equation}
  \cvs = \cps - \frac{R_u}{M_s}
  \label{specific-heat-conversion}
\end{equation}
%------------------------------------------------------------------------------%
The thermodynamic properties curve fit tables developed by McBride, Gordon, and
Reno\cite{McBride} were used to compute $\cps$, and quadratic blending function
was used to ensure that $\cps$ and enthalpy were continuous across temperature
ranges.  The details of this blending are given in Appendix \ref{derivations}.

\section{Chemical Kinetics Model}

The production and destruction of species is governed by the source terms,
$w_s$, defined as
%------------------------------------------------------------------------------%
\begin{equation}
  w_s = M_s \sum_{r=1}^{N_r}\left( \sprod - \sreact \right)
        \left( R_{f,r} - R_{b,r} \right)
  \label{source-term}
\end{equation}
%------------------------------------------------------------------------------%
where $N_r$ is the number of reactions, $\sreact$ and $\sprod$ are the
stoichiometric coefficients for the reactants and products, respectively, and
$R_{f,r}$ and $R_{b,r}$ are the forward and backward rates for reaction
$r$, respectively.  The forward and backward reaction rates are defined as
%------------------------------------------------------------------------------%
\begin{align}
  R_{f,r} &= 1000\left[ k_{f,r}\prod_{s=1}^{N_s}\left( 0.001\rho_s/M_s
  \right)\sreact \right] \label{forward-rate} \\
  R_{b,r} &= 1000\left[ k_{b,r}\prod_{s=1}^{N_s}\left( 0.001\rho_s/M_s
  \right)\sprod \right]
  \label{backward-rate}
\end{align}
%------------------------------------------------------------------------------%
where $k_{f,r}$ and $k_{b,r}$ are the forward and backward rate coefficients,
respectively.  It should be noted that all terms on the RHS for
\erefs{forward-rate}{backward-rate} are in cgs units.  The factors 1000 and 0.001 are required
to convert from cgs to mks, so that $R_{f,r}$ and $R_{b,r}$ are in mks units.
The rate coefficients are expressed in the manner defined by Park\cite{park},
but for a one-temperature model; thus, the forward and back rate coefficients
are defined as
%------------------------------------------------------------------------------%
\begin{align}
  k_{f,r} &= C_{f,r}T^{n_{f,r}}exp\left( -E_{f,r}/kT \right)
  \label{forward-rate-coef} \\
  k_{b,r} &= \frac{k_{f,r}}{K_{c,r}}
  \label{backward-rate-coef}
\end{align}
%------------------------------------------------------------------------------%
where $K_{c,r}$ is the equilibrium constant for reaction $r$, and $k$ is the
boltzmann constant.  The preexponential factors $C_{f,r}$ and $n_{f,r}$, as well
as the activation energy, $E_f$, are documented by Gnoffo\cite{gnoffo-tp}.
The equilibrium coefficient is computed according to the curve fit defined by
Park\cite{park1985convergence}
\begin{gather}
  K_{c,r} = exp(\br{1} + \br{2}\ln{Z} + \br{3}Z + \br{4}Z^2 + \br{5}Z^3) \\
  Z = 10000/T
\end{gather}
where the curve fit constants, $\br{i}$, are also documented by
Gnoffo\cite{gnoffo-tp}
