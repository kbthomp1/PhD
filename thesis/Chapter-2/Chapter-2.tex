\chapter{Design Optimization}
\label{chapter-three}

\section{Introduction}

Design optimization is a wide field that encompasses methods generally falling into
two categories: local gradient-based optimization, and heuristic global
optimization.  Local gradient-based optimization techniques focus on the
determining an optimality condition by evaluating a function and its gradients.
Provided certain conditions are met, it can be proven that the optimization
procedure will find a local minimum or maximum on a bounded domain.  Examples of
local gradient-based optimization methods include steepest-descent[cite??],
conjugate-gradient[cite??], and Augmented Lagrangian Method [cite??].  A
heuristic global optimization seeks to find the global extrema of a function.
Although these methods are powerful, because of their heuristic nature they are
not guaranteed to find the absolute optimum condition and are not the focus this
research.

In the field of optimization, the function of interest is referred to as the
``cost function'' or ``objective function''.  Optimization methods seek to
minimize this function; therefore, if the intent is to find the maximum value
of the function, it should be formulated as the negative of the original.

\section{Cost Function Definition}

The cost function (or objective function) as formulated in FUN3D is a composite,
weighted function
%
\begin{equation}
  f = \sum_{j=1}^{N_{func}}w_j\left( C_j - C_{j^*} \right)^{p_j}
  \label{generic_cost_function}
\end{equation}
%
Where $w_j$, $C_{j^*}$, and $p_j$ are the weight, target, and power of cost
function component $j$.  $C_j$ is the component value, which is evaluated at
each flow solution.  Examples of component values include integrated
aerodynamic or aero-thermodynamic quantities.  


