\chapter{Change of Variable Sets}
\label{change-of-var-section}

The decoupled scheme developed by Candler et. al\cite{candler} is based upon the
change of variables
%------------------------------------------------------------------------------%
\begin{equation}
  \mU = \begin{pmatrix}
    \rho_1 \\
    \vdots \\
    \rho_{N_s} \\
    \rho \vu \\
    \rho E
  \end{pmatrix}
  \rightarrow
  \mv = \begin{pmatrix}
    c_1 \\
    \vdots \\
    c_{N_s} \\
    \rho \\
    \rho \vu \\
    \rho E
  \end{pmatrix}
  \label{var-sets}
\end{equation}
%------------------------------------------------------------------------------%
To avoid confusion between variable sets, we re-write the variable vectors,
$\mU$ and $\mv$, in a more generic sense
%------------------------------------------------------------------------------%
\begin{equation}
  \mU = \begin{pmatrix}
    u_1 \\
    \vdots \\
    u_{N_s + 2}
  \end{pmatrix}
  \rightarrow
  \mv = \begin{pmatrix}
    v_1 \\
    \vdots \\
    v_{N_s + 3}
  \end{pmatrix}
  \label{generic-var-sets}
\end{equation}
%------------------------------------------------------------------------------%
For simplicity, consider a system with two species, $\rho_1$ and $\rho_2$.
Using the relationship $\rho_s = c_s \rho$, then the original variable vector,
$\mU$ can be rewritten in terms of the new variables, $\mv$ as
%------------------------------------------------------------------------------%
\begin{equation}
  \mU = \begin{pmatrix}
    u_1 \\
    u_2 \\
    u_3 \\
    u_4
  \end{pmatrix}
  =
  \begin{pmatrix}
    v_1 v_3 \\
    v_2 v_3 \\
    v_4 \\
    v_5
  \end{pmatrix}
  \label{u-to-v}
\end{equation}
%------------------------------------------------------------------------------%
This allows the derivation of the Jacobian 
%------------------------------------------------------------------------------%
\begin{equation}
  \pd{\mU}{\mv} = 
  \begin{pmatrix}
    \pd{u_1}{v_1} & \pd{u_1}{v_2} & \pd{u_1}{v_3} & \pd{u_1}{v_4} & \pd{u_1}{v_5} \\ \\
    \pd{u_2}{v_1} & \pd{u_2}{v_2} & \pd{u_2}{v_3} & \pd{u_2}{v_4} & \pd{u_2}{v_5} \\ \\
    \pd{u_3}{v_1} & \pd{u_3}{v_2} & \pd{u_3}{v_3} & \pd{u_3}{v_4} & \pd{u_3}{v_5} \\ \\
    \pd{u_4}{v_1} & \pd{u_4}{v_2} & \pd{u_4}{v_3} & \pd{u_4}{v_4} & \pd{u_4}{v_5}
  \end{pmatrix}
  =
  \begin{pmatrix}
    v_3 & 0   & v_1 & 0 & 0 \\ \\
    0   & v_3 & v_2 & 0 & 0 \\ \\
    0   & 0   & 0   & 1 & 0 \\ \\
    0   & 0   & 0   & 0 & 1
  \end{pmatrix}
  \label{dudv-jac}
\end{equation}
%------------------------------------------------------------------------------%
At this point, it is important to note that the Jacobian in \eref{dudv-jac} has
two psuedo-inverse matricies, that correspond to the right and left inverse.
The right inverse, $\pdr{\mv}{\mU}{R}$, can be constructed based on the previously
defined steps
%------------------------------------------------------------------------------%
\begin{equation}
  \pdr{\mv}{\mU}{R} = 
  \begin{pmatrix}
    \pd{v_1}{u_1} & \pd{v_1}{u_2} & \pd{v_1}{u_3} & \pd{v_1}{u_4} \\ \\
    \pd{v_2}{u_1} & \pd{v_2}{u_2} & \pd{v_2}{u_3} & \pd{v_2}{u_4} \\ \\
    \pd{v_3}{u_1} & \pd{v_3}{u_2} & \pd{v_3}{u_3} & \pd{v_3}{u_4} \\ \\
    \pd{v_4}{u_1} & \pd{v_4}{u_2} & \pd{v_4}{u_3} & \pd{v_4}{u_4} \\ \\
    \pd{v_5}{u_1} & \pd{v_5}{u_2} & \pd{v_5}{u_3} & \pd{v_5}{u_4} 
  \end{pmatrix}
  =
  \begin{pmatrix}
    \frac{1-v_1}{v_3} & \frac{-v_1}{v_3}  & 0 & 0 \\ \\
    \frac{-v_2}{v_3}  & \frac{1-v_2}{v_3} & 0 & 0 \\ \\
    1                 & 1                 & 0 & 0 \\ \\
    0                 & 0                 & 1 & 0 \\ \\
    0                 & 0                 & 0 & 1
  \end{pmatrix}
  \label{dvdu-jac-right}
\end{equation}
%------------------------------------------------------------------------------%
It is easily verified that the matrix product of
\erefs{dudv-jac}{dvdu-jac-right} produces identity
%------------------------------------------------------------------------------%
\begin{equation}
  \pd{\mU}{\mv} \pdr{\mv}{\mU}{R} = 
  \begin{pmatrix}
    v_3 & 0   & v_1 & 0 & 0 \\ \\
    0   & v_3 & v_2 & 0 & 0 \\ \\
    0   & 0   & 0   & 1 & 0 \\ \\
    0   & 0   & 0   & 0 & 1
  \end{pmatrix}
  \begin{pmatrix}
    \frac{1-v_1}{v_3} & \frac{-v_1}{v_3}  & 0 & 0 \\ \\
    \frac{-v_2}{v_3}  & \frac{1-v_2}{v_3} & 0 & 0 \\ \\
    1                 & 1                 & 0 & 0 \\ \\
    0                 & 0                 & 1 & 0 \\ \\
    0                 & 0                 & 0 & 1
  \end{pmatrix}
  =
  \begin{pmatrix}
    1 & 0 & 0 & 0 \\ \\
    0 & 1 & 0 & 0 \\ \\
    0 & 0 & 1 & 0 \\ \\
    0 & 0 & 0 & 1
  \end{pmatrix}
  \label{dvdu-jac-right-I}
\end{equation}
%------------------------------------------------------------------------------%
however, \eref{dvdu-jac-right-I} is not associative
%------------------------------------------------------------------------------%
\begin{equation}
  \begin{aligned}
    \pdr{\mv}{\mU}{R} \pd{\mU}{\mv} &= 
    \begin{pmatrix}
      \frac{1-v_1}{v_3} & \frac{-v_1}{v_3}  & 0 & 0 \\ \\
      \frac{-v_2}{v_3}  & \frac{1-v_2}{v_3} & 0 & 0 \\ \\
      1                 & 1                 & 0 & 0 \\ \\
      0                 & 0                 & 1 & 0 \\ \\
      0                 & 0                 & 0 & 1
    \end{pmatrix}
    \begin{pmatrix}
      v_3 & 0   & v_1 & 0 & 0 \\ \\
      0   & v_3 & v_2 & 0 & 0 \\ \\
      0   & 0   & 0   & 1 & 0 \\ \\
      0   & 0   & 0   & 0 & 1
    \end{pmatrix} \\
    &= 
    \begin{pmatrix}
      1-v_1 & -v_1  & \frac{-(v_1)^2 - v_1 v_2 + v_1}{v_3} & 0 & 0 \\ \\
      -v_2  & 1-v_2 & \frac{-(v_2)^2 - v_1 v_2 + v_2}{v_3} & 0 & 0 \\ \\
      0     & 0     & 0                                    & 1 & 0 \\ \\
      0     & 0     & 0                                    & 0 & 1
    \end{pmatrix}
  \end{aligned}
  \label{dvdu-jac-right-I-bad}
\end{equation}
%------------------------------------------------------------------------------%
to correctly compute identity, the property of matrix transpose multiplication
is used
%------------------------------------------------------------------------------%
\begin{equation}
  \begin{aligned}
    \left( \pd{\mU}{\mv} \pdr{\mv}{\mU}{R} \right)^T &= 
    {\pdr{\mv}{\mU}{R}}^T {\pd{\mU}{\mv}}^T \\ &=
    \begin{pmatrix}
      \frac{1-v_1}{v_3} & \frac{-v_2}{v_3}  & 1 & 0 & 0 \\ \\
      \frac{-v_1}{v_3}  & \frac{1-v_2}{v_3} & 1 & 0 & 0 \\ \\
      0                 & 1                 & 0 & 1 & 0 \\ \\
      0                 & 0                 & 0 & 0 & 1
    \end{pmatrix}
    \begin{pmatrix}
      v_3 & 0   & 0   & 0 \\ \\
      0   & v_3 & 0   & 0 \\ \\
      v_1 & v_2 & 0   & 0 \\ \\
      0   & 0   & 1   & 0 \\ \\
      0   & 0   & 0   & 1 
    \end{pmatrix} \\
    &= 
    \begin{pmatrix}
      1 & 0 & 0 & 0 \\ \\
      0 & 1 & 0 & 0 \\ \\
      0 & 0 & 1 & 0 \\ \\
      0 & 0 & 0 & 1
    \end{pmatrix}
  \end{aligned}
  \label{dvdu-jac-tranpose}
\end{equation}
%------------------------------------------------------------------------------%
This is critical in understanding the relationships needed to switch transform
variables sets.  For linearizations of the residual, $\mr$, the correct
transformation from the variable set $\mU$ to the variable set $\mv$ is
%------------------------------------------------------------------------------%
\begin{equation}
  \pd{\mr}{\mv} = \pd{\mr}{\mU} \pd{\mU}{\mv}
  \label{r-u-to-v}
\end{equation}
%------------------------------------------------------------------------------%
which is intuitively understood; however, the transformation from the variable
set $\mv$ to the variable set $\mU$ must follow \eref{dvdu-jac-tranpose}
%------------------------------------------------------------------------------%
\begin{equation}
  \pd{\mr}{\mU} = \left( \pd{\mr}{\mv}^{T} \pd{\mv}{\mU}^{T} \right)^{T}
  \label{r-v-to-u}
\end{equation}
%------------------------------------------------------------------------------%
The transposition in \eref{r-u-to-v} is critical, as the linearizations will be
incorrect if the multiplication is done without it.  Fortunately,
\eref{r-u-to-v} is rarely seen in practice, as most linearizations are done for
the fully-coupled system that requires \eref{r-v-to-u} to transform the
linearizations
%------------------------------------------------------------------------------%
\begin{equation}
  \begin{aligned}
    \pd{\mr}{\mv} = \pd{\mr}{\mU} \pd{\mU}{\mv} =
    \begin{pmatrix}
      \rdiff{\rho_1}{\rho_1}    & \dots  & \rdiff{\rho_1}{\rho_{N_s}}    & \rdiff{\rho_1}{\rho \vu}    & \rdiff{\rho_1}{\rho E}    \\ \\
      \vdots                    & \ddots & \vdots                       & \vdots                      & \vdots                    \\ \\
      \rdiff{\rho_{N_s}}{\rho_1} & \dots  & \rdiff{\rho_{N_s}}{\rho_{N_s}} & \rdiff{\rho_{N_s}}{\rho \vu} & \rdiff{\rho_{N_s}}{\rho E} \\ \\
      \rdiff{\rho \vu}{\rho_1}  & \dots  & \rdiff{\rho \vu}{\rho_{N_s}}  & \rdiff{\rho \vu}{\rho \vu}  & \rdiff{\rho \vu}{\rho E}  \\ \\
      \rdiff{\rho E}{\rho_1}    & \dots  & \rdiff{\rho E}{\rho_{N_s}}    & \rdiff{\rho E}{\rho \vu}    & \rdiff{\rho E}{\rho E}
    \end{pmatrix}
    \begin{pmatrix}
      \rho   & \dots  & 0      & c_1     & 0      & 0      \\ \\
      \vdots & \ddots & \vdots & \vdots  & \vdots & \vdots \\ \\
      0      & \dots  &\rho    & c_{N_s}  & 0      & 0      \\ \\
      0      & \dots  &0       & 0       & 1      & 0      \\ \\
      0      & \dots  &0       & 0       & 0      & 1
    \end{pmatrix}
  \end{aligned}
  \label{drdu-to-drdv}
\end{equation}
%------------------------------------------------------------------------------%
likewise, in the adjoint the transformation is applied to the tranpose of the 
Jacobian
%------------------------------------------------------------------------------%
\begin{equation}
  \begin{aligned}
    \pd{\mr}{\mU}^{T} = \pd{\mU}{\mv}^{T} \pd{\mr}{\mU}^{T} =
    \begin{pmatrix}
      \rho   & \dots  & 0      &  0      & 0      \\ \\
      \vdots & \ddots & \vdots &  \vdots & \vdots \\ \\
      0      & \dots  &\rho    &  0      & 0      \\ \\
      c_1    & \dots  & c_{N_s} &  0      & 0      \\ \\
      0      & \dots  & 0      &  1      & 0      \\ \\
      0      & \dots  & 0      &  0      & 1
    \end{pmatrix}
    \begin{pmatrix}
      \rdiff{\rho_1}{\rho_1}    & \dots  & \rdiff{\rho_{N_s}}{\rho_1}    & \rdiff{\rho \vu}{\rho_1}    & \rdiff{\rho E}{\rho_1} \\ \\
      \vdots                    & \ddots & \vdots                       & \vdots                      & \vdots                   \\ \\
      \rdiff{\rho_1}{\rho_{N_s}} & \dots  & \rdiff{\rho_{N_s}}{\rho_{N_s}} & \rdiff{\rho \vu}{\rho_{N_s}} & \rdiff{\rho E}{\rho_{N_s}} \\ \\
      \rdiff{\rho_1}{\rho \vu}  & \dots  & \rdiff{\rho_{N_s}}{\rho \vu}  & \rdiff{\rho \vu}{\rho \vu}  & \rdiff{\rho E}{\rho \vu} \\ \\
      \rdiff{\rho_1}{\rho E}    & \dots  & \rdiff{\rho_{N_s}}{\rho E}    & \rdiff{\rho \vu}{\rho E}    & \rdiff{\rho E}{\rho E}
    \end{pmatrix}
  \end{aligned}
  \label{drdu-to-drdv-t}
\end{equation}
%------------------------------------------------------------------------------%
Since the tranformation is left-multiplied, the matrix vector products of the 
exact Jacobian with costate variables, $\adjlam{}$, in the adjoint linear system
can be done first, and the transformation can then be applied to the system
%------------------------------------------------------------------------------%
\begin{equation}
  \begin{aligned}
    \pd{\mr}{\mU}^{T} \adjlam{} =
    \left(
    \begin{smallmatrix}
      \rho   & \dots  & 0      &  0      & 0      \\ \\
      \vdots & \ddots & \vdots &  \vdots & \vdots \\ \\
      0      & \dots  &\rho    &  0      & 0      \\ \\
      c_1    & \dots  & c_{N_s} &  0      & 0      \\ \\
      0      & \dots  & 0      &  1      & 0      \\ \\
      0      & \dots  & 0      &  0      & 1
    \end{smallmatrix}
    \right)
    \left(
    \begin{smallmatrix}
      \rlprod{\rho_1}{\rho_1}    & \dots  &+& \rlprod{\rho_{N_s}}{\rho_1}    &+& \rlprod{\rho \vu}{\rho_1}    &+& \rlprod{\rho E}{\rho_1} \\ \\
      \vdots                     & \ddots & & \vdots                        & & \vdots                       & & \vdots                   \\ \\
      \rlprod{\rho_1}{\rho_{N_s}} & \dots  &+& \rlprod{\rho_{N_s}}{\rho_{N_s}} &+& \rlprod{\rho \vu}{\rho_{N_s}} &+& \rlprod{\rho E}{\rho_{N_s}} \\ \\
      \rlprod{\rho_1}{\rho \vu}  & \dots  &+& \rlprod{\rho_{N_s}}{\rho \vu}  &+& \rlprod{\rho \vu}{\rho \vu}  &+& \rlprod{\rho E}{\rho \vu} \\ \\
      \rlprod{\rho_1}{\rho E}    & \dots  &+& \rlprod{\rho_{N_s}}{\rho E}    &+& \rlprod{\rho \vu}{\rho E}    &+& \rlprod{\rho E}{\rho E}
    \end{smallmatrix}
    \right)
  \end{aligned}
  \label{adj-drdv}
\end{equation}
%------------------------------------------------------------------------------%
This indicates the important point that the transformation of the adjoint
residual is not dependent on the number of equations solved, but only the number
of dependent variables the equations are linearized with respect to, namely
$\mU$.
