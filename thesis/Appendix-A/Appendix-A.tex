\chapter{Derivations}
\label{derivations}

\section{Decoupled Flux Derivation}

For the Roe flux difference splitting scheme, the species mass fluxes are given by
%
\begin{equation}
	F_{\rho_s} = \frac{\rho_s^L\overline{U}^L+\rho_s^R\overline{U}^R}{2}
	-\frac{\tilde{c}_s(\lambda_1 dv_1 + \lambda_2 dv_2)+\lambda_3 dv_{3_s}}{2} \label{species_mass} \\
\end{equation}
\begin{align}	
		dv_1 &= \frac{p^R-p^L+\tilde{\rho} \tilde{a} (\overline{U}^R-\overline{U}^L)}{\tilde{a}^2} \\
		dv_2 &= \frac{p^R-p^L-\tilde{\rho} \tilde{a} (\overline{U}^R-\overline{U}^L)}{\tilde{a}^2} \\
		dv_{3_s} &= \frac{\tilde{a}^2 (\rho_s^R-\rho_s^L)- \tilde{c}_s (p^R-p^L)}{\tilde{a}^2}
\end{align}
\begin{align}
	\lambda_1 = \mid\mathbf{\overline{U}}+\tilde{a} \mid,\quad 
	\lambda_2 = \mid \mathbf{\overline{U}}-\tilde{a} \mid,\quad 
	\lambda_3 =  \mid \mathbf{\overline{U}} \mid
\end{align}
%
where the $\tilde{}$ notation signifies a Roe-averaged quantity, given by:
%
\begin{gather}
	\mathbf{\tilde{U}} =w\mathbf{\tilde{U}}^L+(1-w)\mathbf{\tilde{U}}^R \\
	w = \frac{\tilde{\rho}}{\tilde{\rho}+\rho^R} \\
	\tilde{\rho} = \sqrt{\rho^R\rho^L}
\end{gather}
%
The species mass fluxes must sum to the total mass flux; thus, the total mixture mass flux is given as
%
\begin{equation}
\label{total_mass}
	F_\rho = \sum\limits_{s}{F_{\rho_s}} = \frac{\rho^L\overline{U}^L+\rho^R\overline{U}^R}{2}
	-\frac{\tilde{c}_s(\lambda_1 dv_1 + \lambda_2 dv_2)+\lambda_3 dv_3}{2}
\end{equation}
\begin{equation}
	dv_3 = \frac{\tilde{a}^2 (\rho^R-\rho^L)-(p^R-p^L)}{\tilde{a}^2}
\end{equation}
%
Multiplying Eq.~(\ref{total_mass}) by the Roe-averaged mass fraction and
substituting it into Eq.~(\ref{species_mass}) results in:
%
\begin{equation}
\label{unsimp_sp_flux}
	F_{\rho_s} =\tilde{c}_s F_\rho + \frac{(c_s^L-\tilde{c}_s)\rho^L(\overline{U}^L+\mid \tilde{U}\mid)}{2}
	+ \frac{(c_s^R-\tilde{c}_s)\rho^R(\overline{U}^R-\mid \tilde{U}\mid)}{2}
\end{equation}
%
It should be noted here that the Roe-averaged normal velocity, $\tilde{U}$,
requires an entropy correction in the presence of strong shocks\cite{harten}.
This correction has no dependence on the species mass fractions; therefore,
it does not change the form of the Jacobian for this decoupled scheme. The
notation can be further simplified by defining the normal velocities as follows:
%
\begin{equation} \label{lambda_pm} \lambda^+ = \frac{\overline{U}^L+\mid
  \tilde{U}\mid}{2}, \quad \lambda^- = \frac{\overline{U}^R-\mid
  \tilde{U}\mid}{2} \end{equation}
%
Finally, substituting Eq.~(\ref{lambda_pm}) into Eq.~(\ref{unsimp_sp_flux})
yields the final result for calculating the species flux in the decoupled
system:
%
\begin{equation} \label{sp_flux} F_{\rho_s} =\tilde{c}_s F_\rho +
  (c_s^L-\tilde{c}_s)\rho^L\lambda^+ + (c_s^R-\tilde{c}_s)\rho^R\lambda^-
\end{equation}
%
Forming the convective contributions to the Jacobians is straightforward.
Because the $\mathbf{U}'$ level variables are constant, only the left, right,
and Roe-averaged state mass fractions vary.  Differentiating Eq.~(\ref{sp_flux})
with respect to the mass fraction, $c_s$, the left and right state contributions
are
%
\begin{gather} \frac{\partial F_{\rho_s}}{\partial c^L_s} =
  wF_\rho+(1-w)\rho^L\lambda^+ - w\rho^R\lambda^- \\ \frac{\partial
    F_{\rho_s}}{\partial c^R_s} = (1-w)F_\rho+(w-1)\rho^L\lambda^+ +
    w\rho^R\lambda^- \end{gather}
%
Because there is no dependence between species in decoupled convective
formulation, the Jacobian block elements are purely diagonal for the convective
contributions, of the form
%
\begin{equation} \begin{pmatrix} \frac{\partial F_{\rho_1}}{\partial c_1} & & 0
    \\ & \ddots &  \\ 0 & & \frac{\partial F_{\rho_{ns}}}{\partial c_{ns}}
  \end{pmatrix} \end{equation}
%

\section{Quadratic Interpolation Between Thermodynamic Curve Fits}

We seek to blend the two thermodynamic curve fits in such a way that we maintain $c_0$ continuity in both specific heat ($C_p$) and enthalpy ($h$).  To accomplish this, a quadratic function must be used, of the form
%------------------------------------------------------------------------------%
\begin{equation}
  a T^2 + b T + c = C_p
  \label{generic_form}
\end{equation}
%------------------------------------------------------------------------------%
The coefficients $a$, $b$, and $c$ are determined by solving the system that results from the boundary value problem
%------------------------------------------------------------------------------%
\begin{equation}
  \begin{cases}
    a {T_1}^{2} + b T_1 +c = C_{p_1} \\
    a {T_2}^{2} + b T_2 +c = C_{p_2} \\
    a \frac{\left( {T_2}^{3} - {T_1}^{3}\right) }{3} + b\frac{ \left( {T_2}^{2} - {T_1}^{2}\right) }{2} + c \left( T_2 - T_1\right) = h_2-h_1
  \end{cases}
\end{equation}
%------------------------------------------------------------------------------%
Where the $x_1$ and $x_2$ subscripts describe the left and right states, respectively.  Solving the linear system, the coefficients are
%------------------------------------------------------------------------------%
\begin{equation}
  \begin{cases}
f   a=\frac{3\left( C_{p_2}+ C_{p_1}\right) }{(T_2-T_1)^{2}} - \frac{6 \left(h_2 - h_1\right)}{(T_2-T_1)^{3}}\\ \\
    b=-\frac{2\left[(C_{p_2} + 2C_{p_1})T_2 + (2C_{p_2} + C_{p_1})T_1\right]}{(T_2-T_1)^{2}} + \frac{6(T_2+T_1)(h_2 - h_1)}{(T_2 - T_1)^3}\\ \\
    c=\frac{C_{p_1} T_2 (T_2 + 2T_1) + C_{p_2} T_1 (T_1 + 2 T_2)}{(T_2-T_1)^2} - \frac{6 T_1 T_2 (h_2 - h_1)}{(T_2 - T_1)^3}
  \end{cases}
\end{equation}
%------------------------------------------------------------------------------%
This can be simplified to
%------------------------------------------------------------------------------%
\begin{gather}
  \begin{cases}
    a=3B - A \\ \\
    b=\frac{-2(C_{p_1} T_2 + C_{p_2}T_2)}{(T_2 - T_1)^2} +(T_2+T_1) (A - 2B) \\ \\
    c=\frac{C_{p_1} {T_2}^2 + C_{p_2} {T_1}^2}{(T_2-T_1)^2} + T_1 T_2 (2B - A)
  \end{cases} \\
  A = \frac{6(h_2 - h_1)}{(T_2 - T_1)^3} \\
  B = \frac{C_{p_2} + C_{p_1}}{(T_2 - T_1)^2}
\end{gather}
%------------------------------------------------------------------------------%
Note that this does not ensure that entropy will be continuous across curve
fits.
