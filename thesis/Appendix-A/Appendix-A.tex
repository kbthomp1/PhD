\chapter{Derivations}
\label{derivations}

\section{Decoupled Flux Derivation}

For the Roe flux difference splitting scheme, the species mass fluxes are given by
%------------------------------------------------------------------------------%
\begin{equation}
	F_{\rho_s} = \frac{\rho_s^L\overline{U}^L+\rho_s^R\overline{U}^R}{2}
	-\frac{\tilde{c}_s(\lambda_1 dv_1 + \lambda_2 dv_2)+\lambda_3 dv_{3_s}}{2} \label{species_mass} \\
\end{equation}
\begin{align}	
		dv_1 &= \frac{p^R-p^L+\tilde{\rho} \tilde{a} (\overline{U}^R-\overline{U}^L)}{\tilde{a}^2} \\
		dv_2 &= \frac{p^R-p^L-\tilde{\rho} \tilde{a} (\overline{U}^R-\overline{U}^L)}{\tilde{a}^2} \\
		dv_{3_s} &= \frac{\tilde{a}^2 (\rho_s^R-\rho_s^L)- \tilde{c}_s (p^R-p^L)}{\tilde{a}^2}
\end{align}
\begin{align}
	\lambda_1 = \mid\mathbf{\overline{U}}+\tilde{a} \mid,\quad 
	\lambda_2 = \mid \mathbf{\overline{U}}-\tilde{a} \mid,\quad 
	\lambda_3 =  \mid \mathbf{\overline{U}} \mid
\end{align}
%------------------------------------------------------------------------------%
where the $\tilde{}$ notation signifies a Roe-averaged quantity, given by:
%------------------------------------------------------------------------------%
\begin{gather}
	\mathbf{\tilde{U}} =w\mathbf{\tilde{U}}^L+(1-w)\mathbf{\tilde{U}}^R \\
	w = \frac{\tilde{\rho}}{\tilde{\rho}+\rho^R} \\
	\tilde{\rho} = \sqrt{\rho^R\rho^L}
\end{gather}
%------------------------------------------------------------------------------%
The species mass fluxes must sum to the total mass flux; thus, the total mixture
mass flux is given as
%------------------------------------------------------------------------------%
\begin{equation}
\label{total_mass}
	F_\rho = \sum\limits_{s}{F_{\rho_s}} = \frac{\rho^L\overline{U}^L+\rho^R\overline{U}^R}{2}
	-\frac{\lambda_1 dv_1 + \lambda_2 dv_2 +\lambda_3 dv_3}{2}
\end{equation}
\begin{equation}
	dv_3 = \frac{\tilde{a}^2 (\rho^R-\rho^L)-(p^R-p^L)}{\tilde{a}^2}
\end{equation}
%------------------------------------------------------------------------------%
Multiplying Eq.~(\ref{total_mass}) by the Roe-averaged mass fraction and
substituting it into Eq.~(\ref{species_mass}) results in:
%------------------------------------------------------------------------------%
\begin{equation}
\label{unsimp_sp_flux}
	F_{\rho_s} =\tilde{c}_s F_\rho + \frac{(c_s^L-\tilde{c}_s)\rho^L(\overline{U}^L+\mid \tilde{U}\mid)}{2}
	+ \frac{(c_s^R-\tilde{c}_s)\rho^R(\overline{U}^R-\mid \tilde{U}\mid)}{2}
\end{equation}
%------------------------------------------------------------------------------%
It should be noted here that the Roe-averaged normal velocity, $\tilde{U}$,
requires an entropy correction in the presence of strong shocks\cite{harten}.
This correction has a dependence on the roe-averaged speed of sound, and
therefore has a dependence on the species mass fractions; however,
through numerical experiments it has been determined that omitting this
dependence does not adversely affect convergence.  The notation can be further
simplified by defining the normal velocities as follows:
%------------------------------------------------------------------------------%
\begin{equation} \label{lambda_pm} \lambda^+ = \frac{\overline{U}^L+\mid
  \tilde{U}\mid}{2}, \quad \lambda^- = \frac{\overline{U}^R-\mid
  \tilde{U}\mid}{2} \end{equation}
%------------------------------------------------------------------------------%
Finally, substituting Eq.~(\ref{lambda_pm}) into Eq.~(\ref{unsimp_sp_flux})
yields the final result for calculating the species flux in the decoupled
system:
%------------------------------------------------------------------------------%
\begin{equation} \label{sp_flux} F_{\rho_s} =\tilde{c}_s F_\rho +
  (c_s^L-\tilde{c}_s)\rho^L\lambda^+ + (c_s^R-\tilde{c}_s)\rho^R\lambda^-
\end{equation}
%------------------------------------------------------------------------------%
Forming the convective contributions to the Jacobians is straightforward.
Because the $\mathbf{U}'$ level variables are constant, only the left, right,
and Roe-averaged state mass fractions vary.  Differentiating Eq.~(\ref{sp_flux})
with respect to the mass fraction, $c_s$, the left and right state contributions
are
%------------------------------------------------------------------------------%
\begin{gather} \frac{\partial F_{\rho_s}}{\partial c^L_s} =
  wF_\rho+(1-w)\rho^L\lambda^+ - w\rho^R\lambda^- \\ \frac{\partial
    F_{\rho_s}}{\partial c^R_s} = (1-w)F_\rho+(w-1)\rho^L\lambda^+ +
    w\rho^R\lambda^- \end{gather}
%------------------------------------------------------------------------------%
Because there is no dependence between species in decoupled convective
formulation, the Jacobian block elements are purely diagonal for the convective
contributions, of the form
%------------------------------------------------------------------------------%
\begin{equation} \begin{pmatrix} \frac{\partial F_{\rho_1}}{\partial c_1} & & 0
    \\ & \ddots &  \\ 0 & & \frac{\partial F_{\rho_{ns}}}{\partial c_{ns}}
  \end{pmatrix} \end{equation}
%------------------------------------------------------------------------------%

\section{Quadratic Interpolation Between Thermodynamic Curve Fits}

We seek to blend the two thermodynamic curve fits in such a way that we maintain $c_0$ continuity in both specific heat ($C_p$) and enthalpy ($h$).  To accomplish this, a quadratic function must be used, of the form
%------------------------------------------------------------------------------%
\begin{equation}
  a T^2 + b T + c = C_p
  \label{generic_form}
\end{equation}
%------------------------------------------------------------------------------%
The coefficients $a$, $b$, and $c$ are determined by solving the system that results from the boundary value problem
%------------------------------------------------------------------------------%
\begin{equation}
  \begin{cases}
    a {T_1}^{2} + b T_1 +c = C_{p_1} \\
    a {T_2}^{2} + b T_2 +c = C_{p_2} \\
    a \frac{\left( {T_2}^{3} - {T_1}^{3}\right) }{3} + b\frac{ \left( {T_2}^{2} - {T_1}^{2}\right) }{2} + c \left( T_2 - T_1\right) = h_2-h_1
  \end{cases}
\end{equation}
%------------------------------------------------------------------------------%
Where the $x_1$ and $x_2$ subscripts describe the left and right states, respectively.  Solving the linear system, the coefficients are
%------------------------------------------------------------------------------%
\begin{equation}
  \begin{cases}
f   a=\frac{3\left( C_{p_2}+ C_{p_1}\right) }{(T_2-T_1)^{2}} - \frac{6 \left(h_2 - h_1\right)}{(T_2-T_1)^{3}}\\ \\
    b=-\frac{2\left[(C_{p_2} + 2C_{p_1})T_2 + (2C_{p_2} + C_{p_1})T_1\right]}{(T_2-T_1)^{2}} + \frac{6(T_2+T_1)(h_2 - h_1)}{(T_2 - T_1)^3}\\ \\
    c=\frac{C_{p_1} T_2 (T_2 + 2T_1) + C_{p_2} T_1 (T_1 + 2 T_2)}{(T_2-T_1)^2} - \frac{6 T_1 T_2 (h_2 - h_1)}{(T_2 - T_1)^3}
  \end{cases}
\end{equation}
%------------------------------------------------------------------------------%
This can be simplified to
%------------------------------------------------------------------------------%
\begin{gather}
  \begin{cases}
    a=3B - A \\ \\
    b=\frac{-2(C_{p_1} T_2 + C_{p_2}T_2)}{(T_2 - T_1)^2} +(T_2+T_1) (A - 2B) \\ \\
    c=\frac{C_{p_1} {T_2}^2 + C_{p_2} {T_1}^2}{(T_2-T_1)^2} + T_1 T_2 (2B - A)
  \end{cases} \\
  A = \frac{6(h_2 - h_1)}{(T_2 - T_1)^3} \\
  B = \frac{C_{p_2} + C_{p_1}}{(T_2 - T_1)^2}
\end{gather}
%------------------------------------------------------------------------------%
Note that this does not ensure that entropy will be continuous across curve
fits.

\section{Temperature Linearizations and Non-Dimensionalization}
\label{temperature-jacobian-conditioning}

Following the thermodynamic state relations detailed by Gnoffo et. al.
\cite{gnoffo-tp}, the derivatives of temperature with respect to the conserved
variable vector
%------------------------------------------------------------------------------%
\begin{equation}
  \pd{T}{Q} =
  \begin{pmatrix}
    \pd{T}{\rho_1} \\
    \vdots \\
    \pd{T}{\rho_{ns}} \\
    \pd{T}{\rho u} \\
    \pd{T}{\rho v} \\
    \pd{T}{\rho w} \\
    \pd{T}{\rho E}
  \end{pmatrix}
  \label{tjac}
\end{equation}
%------------------------------------------------------------------------------%
The derivation of \eref{tjac} begins with the definition of the mixture internal
energy, $e$
%------------------------------------------------------------------------------%
\begin{align}
  e &= \ssum \left( c_s e_s \right) 
  = \ssum \left[ c_s \left( \int_{T_{ref}}^{T} \cvs dT + e_{s,o} \right)\right]
  \label{int-energy} \\
  de &= \ssum \left( dc_s e_s \right) + \ssum \left( c_s \cvs \right) dT
  \label{dint-energy}
\end{align}
%------------------------------------------------------------------------------%
solving \eref{dint-energy} for $dT$
%------------------------------------------------------------------------------%
\begin{equation}
  dT = \frac{de - \ssum \left( dc_s e_s \right)}{C_v}
  \label{dt-de}
\end{equation}
%------------------------------------------------------------------------------%
where $C_v$ is the specific heat at constant volume of the mixture, defined as
%------------------------------------------------------------------------------%
\begin{equation}
  C_v = \ssum \left( c_s \cvs \right)
  \label{cv-mix}
\end{equation}
%------------------------------------------------------------------------------%
to transform \eref{dt-de} into a relation with conserved variables, the
definition of mixture total energy is re-arranged as
%------------------------------------------------------------------------------%
\begin{align}
  \rho E &= \rho e + \frac{1}{2}\left( \rho u^2 + \rho v^2 + \rho w^2 \right) \\
  e &= \frac{\rho E - \frac{1}{2} 
       \left( \rho u^2 + \rho v^2 + \rho w^2 \right)}{\rho} \\
  de &= \frac{
        - \left( e \right) d\rho 
        - \left( u \right) d\rho u
        - \left( v \right) d\rho v
        - \left( w \right) d\rho w
        + d \rho E
         }{\rho}
  \label{rhoe-def}
\end{align}
%------------------------------------------------------------------------------%
substituting \eref{rhoe-def} into \eref{dt-de}, \eref{tjac} can be rewritten as
%------------------------------------------------------------------------------%
\begin{equation}
  \pd{T}{Q} =
  \frac{1}{\rho C_v}
  \begin{pmatrix}
    \frac{\left( \qs \right)}{2} - e_1 \\
    \vdots \\
    \frac{\left( \qs \right)}{2} - e_{ns} \\
    -u \\
    -v \\
    -w \\
    1
  \end{pmatrix}
  \label{tjac-complete}
\end{equation}
%------------------------------------------------------------------------------%

The non-dimensionalization of variables in the generic gas path of FUN3D are
based on freestream dimensional quantities
%------------------------------------------------------------------------------%
\begin{equation}
  \begin{aligned}
    \rho &= \rho^{'} / \rho^{'}_{\infty} \\
    u &= u^{'} / V^{'}_{\infty} \\
    v &= v^{'} / V^{'}_{\infty} \\
    w &= w^{'} / V^{'}_{\infty} \\
    e &= e^{'} / \left( V^{'}_{\infty} \right)^2 \\
    T &= T^{'}
  \end{aligned}
  \label{nondim-gg}
\end{equation}
%------------------------------------------------------------------------------%
where $^{'}$ denotes a dimensional quantity. Note that temperature, $T$, is not
nondimensionalized in FUN3D, due to the large number of table look-ups that have
temperature in dimensional units of Kelvin.  This is an important point when
non-dimensionalizing the mixture specific heat at constant volume, $C_v$.  In
the MKS system, $C_v$ has units $J/kg \cdot K$; therefore the proper
nondimensionalization using the system in \eref{nondim-gg} is
%------------------------------------------------------------------------------%
\begin{equation}
  C_v = {C_v}^{'}/\left( V^{'}_{\infty} \right)^2
  \label{cv-nondim}
\end{equation}
%------------------------------------------------------------------------------%
Using \erefs{nondim-gg}{cv-nondim}, \eref{tjac-complete} can be rewritten in
terms of dimensional quantities.
%------------------------------------------------------------------------------%
\begin{equation}
  \pd{T}{Q} =
  \frac{\rho^{'}_{\infty}}{\rho^{'} {C_v}^{'}}
  \begin{pmatrix}
    \frac{\left( \qsp \right)}{2} - e^{'}_1 \\
    \vdots \\
    \frac{\left( \qsp \right)}{2} - e^{'}_{ns} \\
    -u^{'}(V^{'}_{\infty}) \\
    -v^{'}(V^{'}_{\infty}) \\
    -w^{'}(V^{'}_{\infty}) \\
    (V^{'}_{\infty})^2
  \end{pmatrix}
  \label{tjac-dim}
\end{equation}
%------------------------------------------------------------------------------%
As the reference velocity becomes large, as is the case in hypersonic problems,
the quadratic scaling of the final linearization in \eref{tjac-dim},
$\pd{T}{\rho E}$ can lead to poor conditioning of the jacobian matrix.  This is
particular true for the chemical source term, since small changes in temperature
can lead to very large changes in species densities.

\section{Change of Variable Sets}
\label{change-of-var-section}

The decoupled scheme developed by Candler et. al is based upon the change of
variables
%------------------------------------------------------------------------------%
\begin{equation}
  \mU = \begin{pmatrix}
    \rho_1 \\
    \vdots \\
    \rho_{ns} \\
    \rho \vu \\
    \rho E
  \end{pmatrix}
  \rightarrow
  \mv = \begin{pmatrix}
    c_1 \\
    \vdots \\
    c_{ns} \\
    \rho \\
    \rho \vu \\
    \rho E
  \end{pmatrix}
  \label{var-sets}
\end{equation}
%------------------------------------------------------------------------------%
To avoid confusion between variable sets, we re-write the variable vectors,
$\mU$ and $\mv$, in a more generic sense
%------------------------------------------------------------------------------%
\begin{equation}
  \mU = \begin{pmatrix}
    u_1 \\
    \vdots \\
    u_{ns + 2}
  \end{pmatrix}
  \rightarrow
  \mv = \begin{pmatrix}
    v_1 \\
    \vdots \\
    v_{ns + 3}
  \end{pmatrix}
  \label{generic-var-sets}
\end{equation}
%------------------------------------------------------------------------------%
For simplicity, consider a system with two species, $\rho_1$ and $\rho_2$.
Using the relationship $\rho_s = c_s \rho$, then the original variable vector,
$\mU$ can be rewritten in terms of the new variables, $\mv$ as
%------------------------------------------------------------------------------%
\begin{equation}
  \mU = \begin{pmatrix}
    u_1 \\
    u_2 \\
    u_3 \\
    u_4
  \end{pmatrix}
  =
  \begin{pmatrix}
    v_1 v_3 \\
    v_2 v_3 \\
    v_4 \\
    v_5
  \end{pmatrix}
  \label{u-to-v}
\end{equation}
%------------------------------------------------------------------------------%
This allows the derivation of the jacobian 
%------------------------------------------------------------------------------%
\begin{equation}
  \pd{\mU}{\mv} = 
  \begin{pmatrix}
    \pd{u_1}{v_1} & \pd{u_1}{v_2} & \pd{u_1}{v_3} & \pd{u_1}{v_4} & \pd{u_1}{v_5} \\ \\
    \pd{u_2}{v_1} & \pd{u_2}{v_2} & \pd{u_2}{v_3} & \pd{u_2}{v_4} & \pd{u_2}{v_5} \\ \\
    \pd{u_3}{v_1} & \pd{u_3}{v_2} & \pd{u_3}{v_3} & \pd{u_3}{v_4} & \pd{u_3}{v_5} \\ \\
    \pd{u_4}{v_1} & \pd{u_4}{v_2} & \pd{u_4}{v_3} & \pd{u_4}{v_4} & \pd{u_4}{v_5}
  \end{pmatrix}
  =
  \begin{pmatrix}
    v_3 & 0   & v_1 & 0 & 0 \\ \\
    0   & v_3 & v_2 & 0 & 0 \\ \\
    0   & 0   & 0   & 1 & 0 \\ \\
    0   & 0   & 0   & 0 & 1
  \end{pmatrix}
  \label{dudv-jac}
\end{equation}
%------------------------------------------------------------------------------%
At this point, it is important to note that the jacobian in \eref{dudv-jac} has
two psuedo-inverse matricies, that correspond to the right and left inverse.
The right inverse, $\pdr{\mv}{\mU}{R}$, can be constructed based on the previously
defined steps
%------------------------------------------------------------------------------%
\begin{equation}
  \pdr{\mv}{\mU}{R} = 
  \begin{pmatrix}
    \pd{v_1}{u_1} & \pd{v_1}{u_2} & \pd{v_1}{u_3} & \pd{v_1}{u_4} \\ \\
    \pd{v_2}{u_1} & \pd{v_2}{u_2} & \pd{v_2}{u_3} & \pd{v_2}{u_4} \\ \\
    \pd{v_3}{u_1} & \pd{v_3}{u_2} & \pd{v_3}{u_3} & \pd{v_3}{u_4} \\ \\
    \pd{v_4}{u_1} & \pd{v_4}{u_2} & \pd{v_4}{u_3} & \pd{v_4}{u_4} \\ \\
    \pd{v_5}{u_1} & \pd{v_5}{u_2} & \pd{v_5}{u_3} & \pd{v_5}{u_4} 
  \end{pmatrix}
  =
  \begin{pmatrix}
    \frac{1-v_1}{v_3} & \frac{-v_1}{v_3}  & 0 & 0 \\ \\
    \frac{-v_2}{v_3}  & \frac{1-v_2}{v_3} & 0 & 0 \\ \\
    1                 & 1                 & 0 & 0 \\ \\
    0                 & 0                 & 1 & 0 \\ \\
    0                 & 0                 & 0 & 1
  \end{pmatrix}
  \label{dvdu-jac-right}
\end{equation}
%------------------------------------------------------------------------------%
It is easily verified that the matrix product of
\erefs{dudv-jac}{dvdu-jac-right} produces identity
%------------------------------------------------------------------------------%
\begin{equation}
  \pd{\mU}{\mv} \pdr{\mv}{\mU}{R} = 
  \begin{pmatrix}
    v_3 & 0   & v_1 & 0 & 0 \\ \\
    0   & v_3 & v_2 & 0 & 0 \\ \\
    0   & 0   & 0   & 1 & 0 \\ \\
    0   & 0   & 0   & 0 & 1
  \end{pmatrix}
  \begin{pmatrix}
    \frac{1-v_1}{v_3} & \frac{-v_1}{v_3}  & 0 & 0 \\ \\
    \frac{-v_2}{v_3}  & \frac{1-v_2}{v_3} & 0 & 0 \\ \\
    1                 & 1                 & 0 & 0 \\ \\
    0                 & 0                 & 1 & 0 \\ \\
    0                 & 0                 & 0 & 1
  \end{pmatrix}
  =
  \begin{pmatrix}
    1 & 0 & 0 & 0 \\ \\
    0 & 1 & 0 & 0 \\ \\
    0 & 0 & 1 & 0 \\ \\
    0 & 0 & 0 & 1
  \end{pmatrix}
  \label{dvdu-jac-right-I}
\end{equation}
%------------------------------------------------------------------------------%
however, \eref{dvdu-jac-right-I} is not associative
%------------------------------------------------------------------------------%
\begin{equation}
  \begin{aligned}
    \pdr{\mv}{\mU}{R} \pd{\mU}{\mv} &= 
    \begin{pmatrix}
      \frac{1-v_1}{v_3} & \frac{-v_1}{v_3}  & 0 & 0 \\ \\
      \frac{-v_2}{v_3}  & \frac{1-v_2}{v_3} & 0 & 0 \\ \\
      1                 & 1                 & 0 & 0 \\ \\
      0                 & 0                 & 1 & 0 \\ \\
      0                 & 0                 & 0 & 1
    \end{pmatrix}
    \begin{pmatrix}
      v_3 & 0   & v_1 & 0 & 0 \\ \\
      0   & v_3 & v_2 & 0 & 0 \\ \\
      0   & 0   & 0   & 1 & 0 \\ \\
      0   & 0   & 0   & 0 & 1
    \end{pmatrix} \\
    &= 
    \begin{pmatrix}
      1-v_1 & -v_1  & \frac{-(v_1)^2 - v_1 v_2 + v_1}{v_3} & 0 & 0 \\ \\
      -v_2  & 1-v_2 & \frac{-(v_2)^2 - v_1 v_2 + v_2}{v_3} & 0 & 0 \\ \\
      0     & 0     & 0                                    & 1 & 0 \\ \\
      0     & 0     & 0                                    & 0 & 1
    \end{pmatrix}
  \end{aligned}
  \label{dvdu-jac-right-I-bad}
\end{equation}
%------------------------------------------------------------------------------%
to correctly compute identity, the property of matrix transpose multiplication
is used
%------------------------------------------------------------------------------%
\begin{equation}
  \begin{aligned}
    \left( \pd{\mU}{\mv} \pdr{\mv}{\mU}{R} \right)^T &= 
    {\pdr{\mv}{\mU}{R}}^T {\pd{\mU}{\mv}}^T \\ &=
    \begin{pmatrix}
      \frac{1-v_1}{v_3} & \frac{-v_2}{v_3}  & 1 & 0 & 0 \\ \\
      \frac{-v_1}{v_3}  & \frac{1-v_2}{v_3} & 1 & 0 & 0 \\ \\
      0                 & 1                 & 0 & 1 & 0 \\ \\
      0                 & 0                 & 0 & 0 & 1
    \end{pmatrix}
    \begin{pmatrix}
      v_3 & 0   & 0   & 0 \\ \\
      0   & v_3 & 0   & 0 \\ \\
      v_1 & v_2 & 0   & 0 \\ \\
      0   & 0   & 1   & 0 \\ \\
      0   & 0   & 0   & 1 
    \end{pmatrix} \\
    &= 
    \begin{pmatrix}
      1 & 0 & 0 & 0 \\ \\
      0 & 1 & 0 & 0 \\ \\
      0 & 0 & 1 & 0 \\ \\
      0 & 0 & 0 & 1
    \end{pmatrix}
  \end{aligned}
  \label{dvdu-jac-tranpose}
\end{equation}
%------------------------------------------------------------------------------%
This is critical in understanding the relationships needed to switch transform
variables sets.  For linearizations of the residual, $\mr$, the correct
transformation from the variable set $\mU$ to the variable set $\mv$ is
%------------------------------------------------------------------------------%
\begin{equation}
  \pd{\mr}{\mv} = \pd{\mr}{\mU} \pd{\mU}{\mv}
  \label{r-u-to-v}
\end{equation}
%------------------------------------------------------------------------------%
which is intuitively understood; however, the transformation from the variable
set $\mv$ to the variable set $\mU$ must follow \eref{dvdu-jac-tranpose}
%------------------------------------------------------------------------------%
\begin{equation}
  \pd{\mr}{\mU} = \left( \pd{\mr}{\mv}^{T} \pd{\mv}{\mU}^{T} \right)^{T}
  \label{r-v-to-u}
\end{equation}
%------------------------------------------------------------------------------%
The transposition in \eref{r-u-to-v} is critical, as the linearizations will be
incorrect if the multiplication is done without it.  Fortunately,
\eref{r-u-to-v} is rarely seen in practice, as most linearizations are done for
the fully-coupled system that requires \eref{r-v-to-u} to transform the
linearizations
%------------------------------------------------------------------------------%
\begin{equation}
  \begin{aligned}
    \pd{\mr}{\mv} = \pd{\mr}{\mU} \pd{\mU}{\mv} =
    \begin{pmatrix}
      \rdiff{\rho_1}{\rho_1}    & \dots  & \rdiff{\rho_1}{\rho_{ns}}    & \rdiff{\rho_1}{\rho \vu}    & \rdiff{\rho_1}{\rho E}    \\ \\
      \vdots                    & \ddots & \vdots                       & \vdots                      & \vdots                    \\ \\
      \rdiff{\rho_{ns}}{\rho_1} & \dots  & \rdiff{\rho_{ns}}{\rho_{ns}} & \rdiff{\rho_{ns}}{\rho \vu} & \rdiff{\rho_{ns}}{\rho E} \\ \\
      \rdiff{\rho \vu}{\rho_1}  & \dots  & \rdiff{\rho \vu}{\rho_{ns}}  & \rdiff{\rho \vu}{\rho \vu}  & \rdiff{\rho \vu}{\rho E}  \\ \\
      \rdiff{\rho E}{\rho_1}    & \dots  & \rdiff{\rho E}{\rho_{ns}}    & \rdiff{\rho E}{\rho \vu}    & \rdiff{\rho E}{\rho E}
    \end{pmatrix}
    \begin{pmatrix}
      \rho   & \dots  & 0      & c_1     & 0      & 0      \\ \\
      \vdots & \ddots & \vdots & \vdots  & \vdots & \vdots \\ \\
      0      & \dots  &\rho    & c_{ns}  & 0      & 0      \\ \\
      0      & \dots  &0       & 0       & 1      & 0      \\ \\
      0      & \dots  &0       & 0       & 0      & 1
    \end{pmatrix}
  \end{aligned}
  \label{drdu-to-drdv}
\end{equation}
%------------------------------------------------------------------------------%
likewise, in the adjoint the transformation is applied to the tranpose of the 
jacobian
%------------------------------------------------------------------------------%
\begin{equation}
  \begin{aligned}
    \pd{\mr}{\mU}^{T} = \pd{\mU}{\mv}^{T} \pd{\mr}{\mU}^{T} =
    \begin{pmatrix}
      \rho   & \dots  & 0      &  0      & 0      \\ \\
      \vdots & \ddots & \vdots &  \vdots & \vdots \\ \\
      0      & \dots  &\rho    &  0      & 0      \\ \\
      c_1    & \dots  & c_{ns} &  0      & 0      \\ \\
      0      & \dots  & 0      &  1      & 0      \\ \\
      0      & \dots  & 0      &  0      & 1
    \end{pmatrix}
    \begin{pmatrix}
      \rdiff{\rho_1}{\rho_1}    & \dots  & \rdiff{\rho_{ns}}{\rho_1}    & \rdiff{\rho \vu}{\rho_1}    & \rdiff{\rho E}{\rho_1} \\ \\
      \vdots                    & \ddots & \vdots                       & \vdots                      & \vdots                   \\ \\
      \rdiff{\rho_1}{\rho_{ns}} & \dots  & \rdiff{\rho_{ns}}{\rho_{ns}} & \rdiff{\rho \vu}{\rho_{ns}} & \rdiff{\rho E}{\rho_{ns}} \\ \\
      \rdiff{\rho_1}{\rho \vu}  & \dots  & \rdiff{\rho_{ns}}{\rho \vu}  & \rdiff{\rho \vu}{\rho \vu}  & \rdiff{\rho E}{\rho \vu} \\ \\
      \rdiff{\rho_1}{\rho E}    & \dots  & \rdiff{\rho_{ns}}{\rho E}    & \rdiff{\rho \vu}{\rho E}    & \rdiff{\rho E}{\rho E}
    \end{pmatrix}
  \end{aligned}
  \label{drdu-to-drdv-t}
\end{equation}
%------------------------------------------------------------------------------%
Since the tranformation is right-multiplied, the matrix vector products of the 
exact jacobian with costate variables, $\adjlam{}$, in the adjoint linear system
can be done first, and the transformation can then be applied to the system
%------------------------------------------------------------------------------%
\begin{equation}
  \begin{aligned}
    \pd{\mr}{\mU}^{T} \adjlam{} =
    \begin{pmatrix}
      \rho   & \dots  & 0      &  0      & 0      \\ \\
      \vdots & \ddots & \vdots &  \vdots & \vdots \\ \\
      0      & \dots  &\rho    &  0      & 0      \\ \\
      c_1    & \dots  & c_{ns} &  0      & 0      \\ \\
      0      & \dots  & 0      &  1      & 0      \\ \\
      0      & \dots  & 0      &  0      & 1
    \end{pmatrix}
    \begin{pmatrix}
      \rlprod{\rho_1}{\rho_1}    & \dots  &+& \rlprod{\rho_{ns}}{\rho_1}    &+& \rlprod{\rho \vu}{\rho_1}    &+& \rlprod{\rho E}{\rho_1} \\ \\
      \vdots                     & \ddots & & \vdots                        & & \vdots                       & & \vdots                   \\ \\
      \rlprod{\rho_1}{\rho_{ns}} & \dots  &+& \rlprod{\rho_{ns}}{\rho_{ns}} &+& \rlprod{\rho \vu}{\rho_{ns}} &+& \rlprod{\rho E}{\rho_{ns}} \\ \\
      \rlprod{\rho_1}{\rho \vu}  & \dots  &+& \rlprod{\rho_{ns}}{\rho \vu}  &+& \rlprod{\rho \vu}{\rho \vu}  &+& \rlprod{\rho E}{\rho \vu} \\ \\
      \rlprod{\rho_1}{\rho E}    & \dots  &+& \rlprod{\rho_{ns}}{\rho E}    &+& \rlprod{\rho \vu}{\rho E}    &+& \rlprod{\rho E}{\rho E}
    \end{pmatrix}
  \end{aligned}
  \label{adj-drdv}
\end{equation}
%------------------------------------------------------------------------------%
This indicates the important point that the transformation of the adjoint
residual is not dependent on the number of equations solved, but only the number
of dependent variables the equations are linearized with respect to, namely
$\mU$.

In the decoupled scheme the number of equations effectively solved is one more
than the fully-coupled scheme.  The residual vector, $R$ for the decoupled
scheme can be written as
%------------------------------------------------------------------------------%
\begin{equation}
  \res{} =
  \begin{pmatrix}
    \res{\rho_1} - c_1 \resrho \\ \\
    \vdots \\ \\
    \res{\rho_{N_s}} - c_{N_s} \resrho \\ \\
    \resrho \\ \\
    \res{\rho \vu} \\ \\
    \res{\rho E}
  \end{pmatrix}
  \label{dc-res}
\end{equation}
%------------------------------------------------------------------------------%
The residual vector in \eref{dc-res} is composed entirely of components from
the fully coupled system; there for the linearizations for the fully-coupled
system can be re-used to construct the decoupled adjoint residual

\section{Relationship to Adjoint Equation}

The flow solver equations can be constructed by the integration of the governing
equations.  In semi-discrete form, this is 
%------------------------------------------------------------------------------%
\begin{equation}
  \pd{\mU}{t} + \frac{1}{V} \sum_{i}^{N_{nodes}}
  \left( \vF_i \cdot \norm_i \right) = \mw
  \label{semi-discrete}
\end{equation}
%------------------------------------------------------------------------------%
where
%------------------------------------------------------------------------------%
\begin{equation}
  \vF = 
  \begin{pmatrix}
    \rho_1 \vu \\
    \vdots \\
    \rho_{N_s} \\
    \rho \vu^2 + p\norm \\
    (E + p) \vu
  \end{pmatrix}
  \label{f-vec}
\end{equation}
%------------------------------------------------------------------------------%
the next time level $n+1$ can be determined from the current time level
$n$ if the equations are linearized by the approximations
%------------------------------------------------------------------------------%
\begin{equation}
  \begin{aligned}
    \vF^{n+1} &\approx \vF^{n} + \pd{\vF^{n}}{\mU} d \mU^{n} \\
    \mw^{n+1} &\approx \mw^{n} + \pd{\mw^{n}}{\mU} d \mU^{n}
  \end{aligned}
  \label{linearization-flux}
\end{equation}
%------------------------------------------------------------------------------%
This creates the linear system of equations which may be solved by a
quasi-Newton method
%------------------------------------------------------------------------------%
\begin{equation}
  \left[ 
    \frac{V}{\Delta t} \mi + 
    \begin{pmatrix}
      \tdiff{\rho_1}{\rho_1}    & \dots  & \tdiff{\rho_{ns}}{\rho_1}    & \tdiff{\rho \vu}{\rho_1}    & \tdiff{\rho E}{\rho_1} \\ \\
      \vdots                    & \ddots & \vdots                       & \vdots                      & \vdots                   \\ \\
      \tdiff{\rho_1}{\rho_{ns}} & \dots  & \tdiff{\rho_{ns}}{\rho_{ns}} & \tdiff{\rho \vu}{\rho_{ns}} & \tdiff{\rho E}{\rho_{ns}} \\ \\
      \tdiff{\rho_1}{\rho \vu}  & \dots  & \tdiff{\rho_{ns}}{\rho \vu}  & \tdiff{\rho \vu}{\rho \vu}  & \tdiff{\rho E}{\rho \vu} \\ \\
      \tdiff{\rho_1}{\rho E}    & \dots  & \tdiff{\rho_{ns}}{\rho E}    & \tdiff{\rho \vu}{\rho E}    & \tdiff{\rho E}{\rho E}
    \end{pmatrix}
  \right]
  \begin{pmatrix}
    d \rho_1     \\ \\
    \vdots       \\ \\
    d \rho_{N_s} \\ \\
    d \rho \vu   \\ \\
    d \rho E
  \end{pmatrix}
  =
  \begin{pmatrix}
    \res{\rho_1}     \\ \\
    \vdots           \\ \\
    \res{\rho_{N_s}} \\ \\
    \res{\rho \vu}   \\ \\
    \res{\rho E}
  \end{pmatrix}
  \label{fc-flow}
\end{equation}
%------------------------------------------------------------------------------%
assuming all linearizations are exact, the adjoint system of equations that
results from the fully coupled formulation in \eref{fc-flow} is
%------------------------------------------------------------------------------%
\begin{equation}
  \begin{pmatrix}
    \rdiff{\rho_1}{\rho_1}    & \dots  & \rdiff{\rho_{ns}}{\rho_1}    & \rdiff{\rho \vu}{\rho_1}    & \rdiff{\rho E}{\rho_1}    \\ \\
    \vdots                    & \ddots & \vdots                       & \vdots                      & \vdots                    \\ \\
    \rdiff{\rho_1}{\rho_{ns}} & \dots  & \rdiff{\rho_{ns}}{\rho_{ns}} & \rdiff{\rho \vu}{\rho_{ns}} & \rdiff{\rho E}{\rho_{ns}} \\ \\
    \rdiff{\rho_1}{\rho \vu}  & \dots  & \rdiff{\rho_{ns}}{\rho \vu}  & \rdiff{\rho \vu}{\rho \vu}  & \rdiff{\rho E}{\rho \vu}  \\ \\
    \rdiff{\rho_1}{\rho E}    & \dots  & \rdiff{\rho_{ns}}{\rho E}    & \rdiff{\rho \vu}{\rho E}    & \rdiff{\rho E}{\rho E}
  \end{pmatrix}
  \begin{pmatrix}
    \adjlam{\rho_1}     \\ \\
    \vdots              \\ \\
    \adjlam{\rho_{N_s}} \\ \\
    \adjlam{\rho \vu}   \\ \\
    \adjlam{\rho E}
  \end{pmatrix}
  = -
  \begin{pmatrix}
    \pd{f}{\rho_1}     \\ \\
    \vdots           \\ \\
    \pd{f}{\rho_{N_s}} \\ \\
    \pd{f}{\rho \vu}   \\ \\
    \pd{f}{\rho E}
  \end{pmatrix}
  \label{fc-adj}
\end{equation}
%------------------------------------------------------------------------------%
In the new variable set, $\mv$, \eref{fc-flow} is re-written as
%------------------------------------------------------------------------------%
\begin{equation}
  \left[ 
    \frac{V}{\Delta t} \mi + 
    \begin{pmatrix}
      \tdiff{\rho_1}{\rho_1}    & \dots  & \tdiff{\rho_{ns}}{\rho_1}    & \tdiff{\rho}{\rho_1}    & \tdiff{\rho \vu}{\rho_1}    & \tdiff{\rho E}{\rho_1} \\ \\
      \vdots                    & \ddots & \vdots                       & \vdots                  & \vdots                      & \vdots                   \\ \\
      \tdiff{\rho_1}{\rho_{ns}} & \dots  & \tdiff{\rho_{ns}}{\rho_{ns}} & \tdiff{\rho}{\rho_{ns}} & \tdiff{\rho \vu}{\rho_{ns}} & \tdiff{\rho E}{\rho_{ns}} \\ \\
      \tdiff{\rho_1}{\rho}      & \dots  & \tdiff{\rho_{ns}}{\rho}      & \tdiff{\rho}{\rho}      & \tdiff{\rho \vu}{\rho}      & \tdiff{\rho E}{\rho}      \\ \\
      \tdiff{\rho_1}{\rho \vu}  & \dots  & \tdiff{\rho_{ns}}{\rho \vu}  & \tdiff{\rho}{\rho \vu}  & \tdiff{\rho \vu}{\rho \vu}  & \tdiff{\rho E}{\rho \vu} \\ \\
      \tdiff{\rho_1}{\rho E}    & \dots  & \tdiff{\rho_{ns}}{\rho E}    & \tdiff{\rho}{\rho E}    & \tdiff{\rho \vu}{\rho E}    & \tdiff{\rho E}{\rho E}
    \end{pmatrix}
  \right]
  \begin{pmatrix}
    d c_1      \\ \\
    \vdots     \\ \\
    d c_{N_s}  \\ \\
    d \rho     \\ \\
    d \rho \vu \\ \\
    d \rho E
  \end{pmatrix}
  =
  \begin{pmatrix}
    \res{\rho_1}^{'}     \\ \\
    \vdots               \\ \\
    \res{\rho_{N_s}}^{'} \\ \\
    \res{\rho}           \\ \\
    \res{\rho \vu}       \\ \\
    \res{\rho E}
  \end{pmatrix}
  \label{dc-flow}
\end{equation}
%------------------------------------------------------------------------------%
Where the new species density equations $\res{\rho_s}^{'}$ and mixture density
equation $\res{\rho}$ are defined as
%------------------------------------------------------------------------------%
\begin{align}
  \res{\rho_s}^{'} &= \res{\rho_s} - c_s \res{\rho}
  \label{res-rhos-def} \\
  \res{\rho} &= \sum_{i=1}^{N_s}\left( \res{\rho_i} \right)
  \label{res-rho-def}
\end{align}
%------------------------------------------------------------------------------%
\erefs{res-rhos-def}{res-rho-def} are critical to the adjoint formulation of
\eref{dc-flow}, as primal flow equations have been altered to enforce the
constraint that
%------------------------------------------------------------------------------%
\begin{equation}
  \sum_{i=1}^{N_s}\left( c_i \right) = 1, \quad
  \sum_{i=1}^{N_s}\left( d c_i \right) = 0, \ 
  \label{mass-frac-sum}
\end{equation}
%------------------------------------------------------------------------------%
Just as relationships were derived for the variables set $\mU$ and $\mv$ in
\sref{change-of-var-section}, there are also relationships between the
equations, which we denote as $\ru{}$ and $\rv{}$ for the variable sets
$\mU{}$ and $\mv{}$, respectively.  the equation set $\ru{}$ can be rewritten in
terms of the equation set $\rv{}$, to form the jacobian
%------------------------------------------------------------------------------%
\begin{gather}
  \ru{} =
  \begin{pmatrix}
    \rv{i} + c_i \left(\rv{N_s+1} \right) \\
    \rv{N_s+2} \\
    \rv{N_s+3}
  \end{pmatrix}
  \label{ru-to-rv} \\[12pt]
  \pd{\ru{}}{\rv{}} =
  \begin{pmatrix}
    1 & 0 & c_i & 0 & 0 \\ \\
    0 & 1 & c_i & 0 & 0 \\ \\
    0 & 0 & 0   & 1 & 0 \\ \\
    0 & 0 & 0   & 0 & 1 \\ \\
  \end{pmatrix}
  \label{drudrv}
\end{gather}
%------------------------------------------------------------------------------%
Likewise, the tranformation can be made from $\rv{}$ to $\ru{}$
%------------------------------------------------------------------------------%
\begin{gather}
  \rv{} =
  \begin{pmatrix}
    \ru{i} + c_i \lsum{k=1}{N_s}{\left( \ru{k} \right)} \\ \\
    \lsum{k=1}{N_s}{\left( \ru{k} \right)} \\ \\
    \ru{N_s+1} \\ \\
    \ru{N_s+2}
  \end{pmatrix}
  \label{rv-to-ru} \\[12pt]
  \pd{\rv{}}{\ru{}} =
  \begin{pmatrix}
    1 - c_i & -c_i    & 0 & 0 \\ \\
    -c_i    & 1 - c_i & 0 & 0 \\ \\
    1       & 1       & 0 & 0 \\ \\
    0       & 0       & 1 & 0 \\ \\
    0       & 0       & 0 & 1
  \end{pmatrix}
  \label{drvdru}
\end{gather}
%------------------------------------------------------------------------------%
The change of equations sets in \erefs{ru-to-rv}{drudrv} and the change of
dependent variable in \eref{u-to-v} can be used to rewrite the fully coupled
system in \eref{fc-flow} into the decoupled system from \eref{dc-flow}

%------------------------------------------------------------------------------%

%------------------------------------------------------------------------------%

Based on \eref{adj-u-to-v}, it is possible to reuse the exact jacobian of the
fully coupled scheme, $\ma_{\mU}$, instead of computing the exact jacobian of
the decoupled system, $\ma_{\mv}$.  This is very attractive, since the
implementation of the fully coupled scheme does not need to be changed at the
low-level linearizations.  Instead, the residual of the adjoint can be formed in
the exact same fashion as the fully coupled scheme, and a series of matrix
operations can then be performed to transform the equations and dependent
variables into those used by the decoupled scheme.

The exact same preconditioners used by the flow solver can be transposed and
used to solve the linear system of equations in the adjoint.  This is done in
two steps and in reverse order of the iterative mechanism used in the flow
solver. First, the adjoint costate variables associated with the species mass
equations, $\adjlam{\rho_s}$, are solved for
%------------------------------------------------------------------------------%
\begin{equation}
  \left( \frac{V}{\Delta t} \mi + \ma_d \right) d \adjlam{\rho_s}
  = -
  \begin{pmatrix}
    \lsum{i=1}{N_s}{\rlpprod{\rho_i}{c_s}} 
    + \rlprod{\rho}{c_s}
    + \rlprod{\rho \vu}{c_s}
    + \rlprod{\rho E}{c_s}
    + \pd{f}{c_s}
  \end{pmatrix}
  \label{dc-solve-adj}
\end{equation}
%------------------------------------------------------------------------------%
Followed by the adjoint costate variables associated with the mixture equations,
$\adjlam{\rho}$, $\adjlam{\rho \vu}$, and $\adjlam{\rho E}$
%------------------------------------------------------------------------------%
\begin{equation}
  \left( \frac{V}{\Delta t} \mi + \ma_m \right) 
  \begin{pmatrix}
    d \adjlam{\rho}     \\ \\
    d \adjlam{\rho \vu} \\ \\
    d \adjlam{\rho E}
  \end{pmatrix}
  = -
  \begin{pmatrix}
    \lsum{i=1}{N_s}{\rlpprod{\rho_i}{\rho}} + \rlprod{\rho}{\rho} + \rlprod{\rho \vu}{\rho} + \rlprod{\rho E}{\rho} + \pd{f}{\rho} \\ \\
    \lsum{i=1}{N_s}{\rlpprod{\rho_i}{\rho \vu}} + \rlprod{\rho}{\rho \vu} + \rlprod{\rho \vu}{\rho \vu} + \rlprod{\rho E}{\rho \vu} + \pd{f}{\rho \vu} \\ \\
    \lsum{i=1}{N_s}{\rlpprod{\rho_i}{\rho E}} + \rlprod{\rho}{\rho E} + \rlprod{\rho \vu}{\rho E} + \rlprod{\rho E}{\rho E} + \pd{f}{\rho E}
  \end{pmatrix}
  \label{mean-solve-adj}
\end{equation}
%------------------------------------------------------------------------------%

