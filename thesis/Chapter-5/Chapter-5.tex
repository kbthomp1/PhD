\chapter{Design Optimization}
\label{chapter-five}

Design optimization is a wide field that encompasses methods generally falling
into two categories: local gradient-based optimization, and heuristic global
optimization.  Local gradient-based optimization techniques focus on the
determining an optimality condition by evaluating a function and its gradients.
Provided certain conditions are met, it can be proven that the optimization
procedure will find a local minimum or maximum on a bounded domain.  Examples of
local gradient-based optimization methods include
steepest-descent\cite{fletcher1963rapidly}, sequential quadratic programming
(SQP)\cite{SNOPT-alg}, as well as an interesting method that converts a
constrained optimization problem into an unconstrained on by employing the
Kreisselmeier-Steinhauser function\cite{wrenn1989indirect}.  A heuristic global
optimization seeks to find the global extrema of a function.  Although these
methods are powerful, because of their heuristic nature they are not guaranteed
to find the absolute optimum condition and are not the focus this research.

In the field of optimization, the function of interest is referred to as the
``cost function'' or ``objective function''.  Optimization methods seek to
minimize this function; therefore, if the intent is to find the maximum value
of the function, it should be formulated as the negative of the original.  This
section focuses on geometry and test conditions for the optimization, the
implementation of the cost function components and design variables used in the
optimization, as well as the interface to the optimizer that is used.

\section{Annular Jet Configuration and Test Conditions}

This design optimization is intended to showcase the adjoint-based formulation
used to obtain sensitivity information.  The geometry chosen is a hypersonic
re-entry vehicle with an annular nozzle, as shown in
\fref{fig:annular-jet-side}.
%------------------------------------------------------------------------------%
\begin{figure}[h]
  \centering
	\begin{subfigure}[b]{0.4\textwidth}
    \centering
    \includegraphics[width=\textwidth]{figures/all_iso.png}
  \end{subfigure}
	\begin{subfigure}[b]{0.4\textwidth}
    \centering
    \includegraphics[width=\textwidth]{figures/all_side.png}
  \end{subfigure}
  \caption{Annular Jet Geometry}
  \label{fig:annular-jet-side}
\end{figure}
%------------------------------------------------------------------------------%
This geometry was originally investigated by Gnoffo et
al\cite{gnoffo2016tapping} to obtain increased drag from ``pulsing'' the annular
jet to obtain a beneficial effect from the unsteady shock interaction with the
plume of the jet.  For this work, the optimization is conducted for purely steady
flow, with the intent of altering the plenum conditions to achieve the optimum
condition of maximizing drag, while minimizing surface temperature.

The geometry was generated with the parameters shown in \tref{tab:annular-geom},
with the mesh originally created as structured grid and then converted to an
unstructured grid of hexahedra elements.
%------------------------------------------------------------------------------%
\begin{table}[h]
  \centering
  \begin{tabular}{c|c|c}
    Parameter & Description & Value \\
    \hline
    $r_{throat}$       &   nozzle throat radius, $m$           & 0.02 \\
    $r_{plenum}$       &   nozzle radius at plenum face, $m$   & 0.05 \\
    $r_{exit,inner}$   &   inside nozzle radius at exit, $m$   & 0.03 \\
    $r_{exit,outer}$   &   outside nozzle radius at exit, $m$  & 0.05 \\
    $l_{conv}$         &   distance from plenum to throat, $m$  & 0.05 \\
    $\theta_c$         &   cone half angle, deg                & 70.0
  \end{tabular}
  \caption{Annular Nozzle Geometry Inputs}
  \label{tab:annular-geom}
\end{table}
%------------------------------------------------------------------------------%
The flow conditions for the optimization are shown in \tref{tab:flow-conditions}
%------------------------------------------------------------------------------%
\begin{table}[!h]
  \centering
  \begin{tabular}{c|c|c}
    Flow Condition & Description & Value \\
    \hline
    $V_{\infty}$    & freestream velocity, $m/s$        & 5686.24 \\
    $\rho_{\infty}$ & freestream density, $kg/m^3$      & 0.001 \\
    $T_{\infty}$    & freestream temperature, $K$       & 200.0 \\
    $M_{\infty}$    & freestream Mach number (derived)  & 20.0
  \end{tabular}
  \caption{Flow Conditions}
  \label{tab:flow-conditions}
\end{table}
%------------------------------------------------------------------------------%

\section{Integrated Quantities of Interest}

The quantities of interest for this annular jet geometry are surface temperature
RMS, the outer surface drag coefficient, and the mass flow rate through the
nozzle plenum.  The drag coefficient is defined as
%------------------------------------------------------------------------------%
\begin{equation}
  C_D = \sum_{i}^{N_{faces}}
        \frac{2\left( p_i - p_\infty \right)n_{x_{i}}}
        {\rho_{\infty} V_{\infty}S_{ref}}
  \label{drag-coef-def}
\end{equation}
%------------------------------------------------------------------------------%
where $p_i$ is the average pressure at face $i$.  RMS of surface temperature is
defined as
%------------------------------------------------------------------------------%
\begin{equation}
  \sqrt{
    \frac{\sum_{i}^{N_{faces}}\left( T_{RMS} A_i \right)^2}
       {\sum_{i}^{N_{faces}}\left( A_i \right)^2}
     }
  \label{tt-rms-def}
\end{equation}
%------------------------------------------------------------------------------%
The area-weighted RMS of surface temperature was chosen over a simple
area-weighted average of surface temperature, because the stagnation temperature
is generally much higher than temperatures elsewhere on a vehicle forebody in
hypersonic flows; therefore, the squaring of temperature in the RMS will give
greater weight to the stagnation temperature in the design.

%------------------------------------------------------------------------------%
\begin{figure}[h]
  \centering
	\begin{subfigure}[b]{0.4\textwidth}
    \centering
    \includegraphics[width=\textwidth]{figures/surface2.png}
    \caption{$C_D$ and $T_{RMS}$ Integrated Area}
    \label{fig:cd-t-rms-area}
  \end{subfigure}
	\begin{subfigure}[b]{0.4\textwidth}
    \centering
    \includegraphics[width=\textwidth]{figures/plenum_bc.png}
    \caption{Plenum Face Boundary Condition}
    \label{fig:plenum-face}
  \end{subfigure}
\end{figure}
%------------------------------------------------------------------------------%

A primary objective of this optimization is to explore the effects of the plume
from the annular jet interacting with the bow shock; thus, the thrust effects
and, therefore, forces inside the nozzle are ignored.  To accomplish this, only
the area shown in \fref{fig:cd-t-rms-area} is integrated to compute the drag
coefficient and surface temperature RMS.

The mass flow rate, $\massflow$,
through the area outlined in \fref{fig:plenum-face} is computed as
%------------------------------------------------------------------------------%
\begin{equation}
  \massflow = \sum_{i}^{N_{faces}}\left( \rho_i \overline{U} A \right)
  \label{mass-flow-eqn}
\end{equation}
%------------------------------------------------------------------------------%
and is used a metric for the amount of propellent that is required to be carried
by the vehicle for blowing.  For the purposes of this demonstration problem, a
lower mass flow rate at the plenum is equated to less total vehicle mass.

\section{Composite Cost Function Definition and Components}
\label{cost-func-components}

The cost function (or objective function) as formulated in FUN3D is a composite,
weighted function
%------------------------------------------------------------------------------%
\begin{equation}
  f = \sum_{j=1}^{N_{func}}w_j\left( C_j - C_{j^*} \right)^{p_j}
  \label{generic-cost-function}
\end{equation}
%------------------------------------------------------------------------------%
Where $w_j$, $C_{j^*}$, and $p_j$ are the weight, target, and power of cost
function component $j$.  $C_j$ is the component value, which is evaluated at
each flow solution.  For example, an optimization problem that seeks to minimize the
surface temperature RMS without decreasing the drag, the cost function is
defined as 
%------------------------------------------------------------------------------%
\begin{equation}
  f = w_1\left( T_{RMS} \right)^{2} + w_2\left( C_{D} - C_{D}^{*} \right)^2
  \label{cd-tt-cost-function}
\end{equation}
%------------------------------------------------------------------------------%
For this case the component weights must be determined heuristically, to normalize
the changes in drag coefficient, $C_D$, and surface temperature Root-Mean-Square
(RMS) $T_{RMS}$.  The terms in \eref{cd-tt-cost-function} are squared to provide
a convex design space.

\section{Design Variables}

The design variables for the optimization problem are the plenum total pressure,
$P_{p,o}$, plenum total temperature, $T_{p,o}$, and plenum ``fuel-air ratio'',
$\fa$.  These are provided explicitly in the optimization problem, and are used
to directly set the flow conditions on plenum face boundary condition in the
nozzle, shown in \fref{fig:plenum-face}.  For a reacting gas mixture, the
``fuel-air ratio'' specifies the mass fractions for two species leaving the
plenum.  For example, if an $H_2-N_2$ mixture is ejected from the annular
nozzle, the mass fractions of $H_2$ and $N_2$ are given by
%------------------------------------------------------------------------------%
\begin{equation}
  \begin{aligned}
    c_{H_2} &= \fa \\
    c_{N_2} &= 1 - \fa
  \end{aligned}
  \label{fuel-air-def}
\end{equation}
%------------------------------------------------------------------------------%
Thus, the ratio $\fa$ dictates the mass fractions for two species injected into
the domain via the plenum boundary.

\section{Obtaining Sensitivity Gradients for Design Variables}

The sensitivity gradients for the plenum design variables are easily obtained by
manipulating \eref{obj-function} to obtain
%------------------------------------------------------------------------------%
\begin{equation}
  \pd{L}{\md} = \pd{f}{\md} + \rtdiff{}{\md} \mathbf{\Lambda}
  \label{obj-linearization1}
\end{equation}
%------------------------------------------------------------------------------%
For the cost function components listed in \sref{cost-func-components}, there is
no direct dependence on the plenum design variables; thus,
\eref{obj-linearization1} can be reduced to
%------------------------------------------------------------------------------%
\begin{equation}
  \pd{L}{\md} = \rtdiff{}{\md} \mathbf{\Lambda}
  \label{obj-linearization2}
\end{equation}
%------------------------------------------------------------------------------%
Once the adjoint co-state variables $\mathbf{\Lambda}$ have been computed by
solving the adjoint equations (\eref{adjoint-main}) the sensitivity derivatives
of the cost function with respect to the plenum design variables are obtained by
evaluating relatively inexpensive matrix-vector products.

\section{Inverse Design Optimization}

The optimization procedure is done using the \textit{opt\_driver} utility in
FUN3D, which is a wrapper utility that executes the FUN3D flow solver, adjoint
solver, and optimization algorithm sequentially.  These steps are repeated until
a termination criterion is reached.  In practice, the termination of the
optimization occurs when the cost function reaches a tolerance of less that
$10^{-8}$, or when continuing towards the optimal condition would exceed the
prescribed upper or lower bounds of the design variables.

To demonstrate the inverse design capability of an adjoint-based design
optimization, targets of 2000 K for the surface temperature RMS and 0.0024
$kg/s$ for the annular nozzle mass flow rate were specified.  These targets were
chosen semi-arbitarily, and were heuristically determined to be feasible based
the the design variable bounds. The design variables specified for this
optimization were the plenum total pressure, $P_{p,o}$ and the plenum ``fuel-air
ratio'', $\fa$.  A species mixture consisting of $H_2$ and $N_2$ was blown from
the plenum, with the mass fractions dictated by $\fa$ as described in
\eref{fuel-air-def}.  For the cost function, the weights were chosen
heuristically, such that 
%------------------------------------------------------------------------------%
\begin{equation}
  \frac{w_1}{w_2} = \frac{\cost{T_{RMS}}^2}{\cost{\dot{m}}^2}
  \label{weight-ratio}
\end{equation}
%------------------------------------------------------------------------------%
This results in a roughly equivalent weighting beween the $T_{RMS}$ and
$\dot{m}$, which is desired as both targets should be met at optimality.
%------------------------------------------------------------------------------%
\begin{figure}[h]
  \centering
	\begin{subfigure}[b]{0.45\textwidth}
    \centering
    \includegraphics[width=\textwidth]{figures/1st-H2/cost_func.png}
    \caption{Composite Value}
    \label{fig:cost-func-1st-H2}
  \end{subfigure}
	\begin{subfigure}[b]{0.45\textwidth}
    \centering
    \includegraphics[width=\textwidth]{figures/1st-H2/func_components.png}
    \caption{Component Values}
    \label{fig:components-1st-H2}
  \end{subfigure}
  \caption{Cost Function and Component History}
\end{figure}
%------------------------------------------------------------------------------%
Using the SNOPT optimizer, \fref{fig:cost-func-1st-H2} shows that the target
design was met within 13 function evaluations.  SNOPT explores the entire design
space, as is show in the second function evaluation, where a spike in the cost
function occurred.
%------------------------------------------------------------------------------%
\begin{figure}[h]
  \centering
	\begin{subfigure}[b]{0.45\textwidth}
    \includegraphics[width=\textwidth]{figures/1st-H2/dv_hist.png}
    \caption{Design Variable History}
    \label{fig:dv-hist-1st-H2}
  \end{subfigure}
	\begin{subfigure}[b]{0.45\textwidth}
    \includegraphics[width=\textwidth]{figures/1st-H2/fm_hist.png}
    \caption{Design Variable History}
    \label{fig:fm-hist-1st-H2}
  \end{subfigure}
\end{figure}
%------------------------------------------------------------------------------%
\fref{fig:dv-hist-1st-H2} indicates that the optimizer tried the upper
bound for the plenum pressure design variable, and found that sensitivity
derivatives indicated that the lower pressure was required. This is common, and
is an effective way to insure that there is a local minimum in the prescribed
bound.  The optimization terminated when the cost function value was less than
the tolerance of $10^-8$, and \fref{fig:components-1st-H2} shows that both
components of the cost function were within one order of magnitude of each other
during the optimization.  This last point is important, since non-normalized
components can skew the optimization results, where competing components can
cause oscillations in the function evaluations and stall the optimization
procedure.  \fref{fig:fm-hist-1st-H2} shows the history of the surface
temperature RMS and annular nozzle mass flow rate, with the design targets.  The
optimization clearly made significant progress early, with smaller gains as the
solution approached the target.
