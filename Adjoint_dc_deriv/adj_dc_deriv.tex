%        File: adj_dc_deriv.tex
%     Created: Fri Sep 11 04:00 PM 2015 E
% Last Change: Fri Sep 11 04:00 PM 2015 E
%
\documentclass[10pt]{article}
\usepackage{amsmath}
\usepackage{fullpage,amsmath}

\begin{document}
\centerline{Species Decoupled Costate Variable Relationship Derivation}

\vspace{.2in}

The purpose of this is to show the relationship between the costate variable for total density $\lambda_\rho$ to the costate variables for species densities $\lambda_{\rho_s}$. Beginning with the definition of the Adjoint Equations:
\begin{equation}
  \left(\frac{\partial R}{\partial Q}\right)^T\lambda = \frac{\partial F}{\partial Q}
  \label{adj_eqn}
\end{equation}
Where the $R$ is the residual of the governing equations, $Q$ is the vector of conserved variables, and $F$ is the cost function (i.e. lift, drag, etc.). Note that the first term is simply the transpose of the jacobian multiplied by the costate variable vector $\lambda$, and can be written as:
\begin{equation}
  \left(\frac{\partial R}{\partial Q}\right)_i^T \lambda 
  = \sum_{j=1}^{N_{eq}}{
    \left(\frac{\partial R_j}{\partial Q_i}\right) \lambda_j}
  \label{lhs_sum}
\end{equation}
Suppose we define the system of equations in two different ways.  The first system, which we'll call the ``meanflow system'', consists of 5 equations:
\begin{equation}
  R = \begin{pmatrix} 
        R_{\rho} \\ R_{\rho u} \\ R_{\rho v} \\ R_{\rho w} \\ R_{\rho E}
      \end{pmatrix}, \quad
      \lambda = \begin{pmatrix}
        \lambda_\rho \\ \lambda_{\rho u} \\ \lambda_{\rho v} \\ \lambda_{\rho w} \\
        \lambda_{\rho E}
      \end{pmatrix}
  \label{5x5}
\end{equation}
The second system consists of the full system of equations:
\begin{equation}
  R = \begin{pmatrix} 
        R_{\rho_1} \\ \vdots \\ R_{\rho_s} \\ R_{\rho u} \\
        R_{\rho v} \\ R_{\rho w} \\ R_{\rho E}
      \end{pmatrix}, \quad
      \lambda = \begin{pmatrix}
        \lambda_{\rho_1} \\ \vdots \\ \lambda_{\rho_s} \\
        \lambda_{\rho u} \\ \lambda_{\rho v} \\ \lambda_{\rho w} \\
        \lambda_{\rho E}
      \end{pmatrix}
  \label{full_sys}
\end{equation}
By making the approximation that the mass fraction $c_s$ is constant, we can show that the full system of equations reduces to the meanflow system.  By this approximation the derivatives with respect to species density become:
\begin{equation}
  \frac{\partial R}{\partial \rho} =
  \frac{\partial R}{\partial \rho_s} 
  \frac{\partial \rho_s}{\partial \rho} =
  c_s\left(\frac{\partial R}{\partial \rho_s}\right)
  \label{mass_frac_approx}
\end{equation}
Thus, for a single row of the full system:
\begin{equation}
  \left(\frac{\partial R}{\partial Q}\right)_{\rho_s}^T \lambda 
  = \sum_{j=1}^{N_{full}}{
    \left(\frac{\partial R_j}{\partial \rho}\right) \frac{\lambda_j}{c_s}}
    = \frac{1}{c_s}\left(\frac{\partial F}{\partial \rho}\right)
  \label{full_reduction}
\end{equation}
After cancelling the mass fractions, this allows the first row of the full system to be equated to the first row of the meanflow system:
\begin{equation}
  \sum_{j=1}^{N_{full}}{
    \left(\frac{\partial R_j}{\partial \rho}\right) \lambda_j}
  = \sum_{j=1}^{N_{meanflow}}{
    \left(\frac{\partial R_j}{\partial \rho}\right) \lambda_j}
  \label{eq_mean_full}
\end{equation}
Expanding this out, it becomes clear many terms cancel:
\begin{multline}
  \frac{\partial R_{\rho_1}}{\partial \rho}\lambda_{\rho_1} +
  \dots +
  \frac{\partial R_{\rho_s}}{\partial \rho}\lambda_{\rho_s} +
  \frac{\partial R_{\rho u}}{\partial \rho}\lambda_{\rho u} +
  \frac{\partial R_{\rho v}}{\partial \rho}\lambda_{\rho v} +
  \frac{\partial R_{\rho w}}{\partial \rho}\lambda_{\rho w} +
  \frac{\partial R_{\rho E}}{\partial \rho}\lambda_{\rho E} = \\
  \frac{\partial R_{\rho}}{\partial \rho}\lambda_{\rho} +
  \frac{\partial R_{\rho u}}{\partial \rho}\lambda_{\rho u} +
  \frac{\partial R_{\rho v}}{\partial \rho}\lambda_{\rho v} +
  \frac{\partial R_{\rho w}}{\partial \rho}\lambda_{\rho w} +
  \frac{\partial R_{\rho E}}{\partial \rho}\lambda_{\rho E}
  \label{expand_row}
\end{multline}
\begin{equation}
  \frac{\partial R_{\rho_1}}{\partial \rho}\lambda_{\rho_1} +
  \dots +
  \frac{\partial R_{\rho_s}}{\partial \rho}\lambda_{\rho_s} =
  \frac{\partial R_{\rho}}{\partial \rho}\lambda_{\rho}
  \label{simpl_ex_row}
\end{equation}
Finally, because the individual species mass fluxes must sum to the total mass flux:
\begin{equation}
  \sum_{s=1}^{N_{species}}{R_{\rho_s}} = R_{\rho}
  \label{sp_sum}
\end{equation}
Eqn (\ref{simpl_ex_row}) can be rewritten as:
\begin{equation}
  \frac{\partial R_{\rho_1}}{\partial \rho}\lambda_{\rho_1} +
  \dots +
  \frac{\partial R_{\rho_s}}{\partial \rho}\lambda_{\rho_s} =
  \frac{\partial R_{\rho_1}}{\partial \rho}\lambda_{\rho} +
  \dots +
  \frac{\partial R_{\rho_s}}{\partial \rho}\lambda_{\rho}
  \label{near_final}
\end{equation}
Which implies that the species mass costate variables are all equal to the total mass costate variable, yielding:
\begin{align}
  \lambda_{\rho} &= \lambda_{\rho_s} \\
  d \lambda_{\rho} &= d \lambda_{\rho_s}
  \label{final_result}
\end{align}

\end{document}


